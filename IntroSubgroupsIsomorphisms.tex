\chapter{An Introduction to Subgroups and Isomorphisms}
\label{chapter:intro_subgroups_isomorphisms}
\thispagestyle{empty}

In this chapter, we'll continue to utilize our intuitive definition of a group.  That is, a group $G$ is a set of actions that satisfies the following rules.

\begin{description}
\item[Rule 1.] There is a predefined list of actions that never changes.
\item[Rule 2.] Every action is reversible.
\item[Rule 3.] Every action is deterministic.
\item[Rule 4.] Any sequence of consecutive actions is also an action.
\end{description}

In the previous chapter, we constructed lots of Cayley diagrams for various groups.  To construct a Cayley diagram for a group $G$, we need to first identify a set of generators, say $S$.  Recall that our choice of generators is important as changing the generators can result in a different Cayley diagram.  

In the Cayley diagram for $G$ using $S$, all the actions of $G$ are represented by the vertices of the graph.  Each vertex corresponds to a unique action.  This does not imply that there is a unique way to obtain a given action from the generators.  Each of the generators determines an arrow type in the diagram.  One way to distinguish the different arrow types is by using different colors.  An arrow of a particular color always represents the same generator.

One of the vertices in the diagram is labeled by the do-nothing action, often denoted by $e$.  Each of the other vertices are labeled by words that correspond to following arrows (forwards or backwards) from $e$ to a given vertex.  There may be many ways to do this as each sequence of arrows corresponds to a unique word.  So, a vertex could be potentially labeled by many words.  However, we'll often just pick one.  Also, one potentially confusing item is that we read our words from right to left.  That is, the first arrow we follow out of $e$ is the rightmost generator in the word.

\begin{section}{Subgroups}

\begin{exercise}
Recall the definition of ``subset."  What do you think ``subgroup" means?  Try to come up with a potential definition.  Try not to read any further before doing this.
\end{exercise}

Before continuing, gather up the following Cayley diagrams:
\begin{itemize}
\item $\Spin_{1\times 2}$. There are 3 of these.  I drew one for you in Chapter~\ref{chapter:cayley_diagrams} and you discovered two more in Exercise~\ref{exer:minimal_Cayley_Spin1by2}.
\item $S_2$.  See Exercise~\ref{exer:introducing_S2}.
\item $R_4$.  See Exercise~\ref{exer:introducing_R4}.
\item $D_3$.  There are two of these.  See exercises \ref{exer:introducing_D3} and \ref{exer:alternate_D3}.
\item $D_4$.  See Exercise~\ref{exer:introducing_D4}.
\end{itemize}

\begin{exercise}
Examine your Cayley diagrams for $D_4$ and $R_4$.  Make some observations.  How are they similar and how are they different?  Can you reconcile the similarities and differences by thinking about the actions of each group?
\end{exercise}

Hopefully, one of the things you noticed in the previous exercise is that we can ``see" $R_4$ inside of $D_4$ (and hopefully you didn't just read that before completing the exercise).  You may have used different colors in each case and maybe even labeled the vertices with different words, but the overall structure of $R_4$ is there nonetheless.

\begin{exercise}\label{exer:R4_subgroup_D_4}
If you just pay attention to the configuration of arrows, it appears that there are two copies of the Cayley diagram for $R_4$ in the Cayley diagram for $D_4$.  Isolate these two copies by ignoring the edges that correspond to the generator $s$.  Paying close attention to the words that label the vertices from the original Cayley diagram for $D_4$, are either of these groups in their own right?
\end{exercise}

Recall that the do-nothing action must always be one of the actions included in a group.  If this didn't occur to you when doing the previous exercise, you might want to go back and rethink your answer.  Just like in the previous exercise, we can often ``see" smaller groups living inside larger groups.  These smaller groups are called \textbf{subgroups}.

\begin{intuitivedef}
Let $G$ be a group of actions and let $H\subseteq G$.  We say that $H$ is a \textbf{subgroup} if and only if $H$ is a group in its own right.  In this case, we write $H\leq G$.
\end{intuitivedef}

In light of Exercise~\ref{exer:R4_subgroup_D_4}, we would write $R_4\leq D_4$.  The second sub-diagram of $D_4$ that resembles $R_4$ cannot be a subgroup because it does not contain the do-nothing action.  However, since it looks a lot like $R_4$, we call it a \textbf{clone} of $R_4$.

The next theorem\footnote{Perhaps we should call this an ``Intuitive Theorem" since we are using an intuitive definition of a group.} tells us that if we already have a subset of a group, we only need to check two of our rules instead of four.

\begin{theorem}
Let $G$ be a group of actions and let $H\subseteq G$. Then $H$ is a subgroup if and only if $H$ satisfies Rules 2 and 4.%Note: I should probably discuss how the predefined list of generators in Rule 1 is potentially modified.
\end{theorem}

There is one subgroup that every group has.

\begin{theorem}
Let $G$ be a group of actions.  Then $\{e\}\leq G$.
\end{theorem}

\begin{exercise}
Let $G$ be a group and let $e\in G$.  What does the Cayley diagram for the subgroup $\{e\}$ look like?
\end{exercise}

Earlier, we referred to subgroups as being ``smaller."  However, our definition does not imply that this has to be the case.

\begin{theorem}
Let $G$ be a group of actions.  Then $G\leq G$.
\end{theorem}

We refer to subgroups that are strictly smaller than the whole group as \textbf{proper subgroups}.

Lots of groups have been given formal names (e.g., $D_4$, $R_4$, etc.).  However, not every group or subgroup has a name.  In this case, it's useful to have notation to refer to specific subgroups.

\begin{definition}\label{def:subgroup_gen_by}
Let $G$ be a group of actions and let $g_1,\ldots, g_n$ be distinct actions from $G$.  We define $\langle g_1,\ldots, g_n\rangle$ to be the smallest subgroup containing $g_1,\ldots, g_n$.  In this case, we call $\langle g_1,\ldots, g_n\rangle$ the \textbf{subgroup generated by} $g_1,\ldots, g_n$.
\end{definition}

For example, consider $r, s, s'\in D_3$ (as defined in exercises~\ref{exer:introducing_D3} and \ref{exer:alternate_D3}).  Then $\langle r,s\rangle=\langle s, s'\rangle=D_3$.  Recall that $R_4$ was the subgroup of rotations of the square.  Similarly, the group of rotations of an equilateral triangle is called $R_3$.  Then using the $r$ from $D_3$, we have $\langle r\rangle = R_3$, which is a subgroup of $D_3$.

Note that in Definition~\ref{def:subgroup_gen_by}, we used a finite number of generators.  There's no reason we have to do this.  That is, we can consider groups/subgroups generated by infinitely many elements.

\begin{exercise}
Consider $\Spin_{1\times 2}$.  
\begin{enumerate}
\item[(a)] Can you find the Cayley diagram for $\langle t_1\rangle$ as a subgroup of $\Spin_{1\times 2}$?
\item[(b)] Write down all the actions of the subgroup $\langle t_1, t_2\rangle$. Write them as words in $t_1$ and $t_2$.  Can you find the Cayley diagram for $\langle t_1, t_2\rangle$ as a subgroup of $\Spin_{1\times 2}$?  Can you find a clone for $\langle t_1, t_2\rangle$?
\end{enumerate}
\end{exercise}

One of the benefits of Cayley diagrams is that they are usual for visualizing subgroups.  However, recall that if we change our set of generators, we might get a very different looking Cayley diagram.  The upshot of this is that we may be able to see a subgroup in one Cayley diagram for a given group, but not be able to see if Cayley diagram with a different set of arrows.

\begin{exercise}
We currently have two different Cayley diagrams for $D_3$ (see exercises \ref{exer:introducing_D3} and \ref{exer:alternate_D3}).  
\begin{enumerate}
\item[(a)] Can you find the Cayley diagram for $\langle e\rangle$ as a subgroup of $D_3$?  Can you see it in both Cayley diagrams for $D_3$?  Can you find all the clones?
\item[(b)] Can you find the Cayley diagram for $\langle r\rangle =R_3$ as a subgroup of $D_3$?  Can you see it in both Cayley diagrams?  Can you find all the clones?
\item[(c)] Find the Cayley diagrams for $\langle s\rangle$ and $\langle s'\rangle$ as subgroups of $D_3$.  Can you see them in both Cayley diagrams for $D_3$?  Can you find all the clones?
\end{enumerate}
\end{exercise}

\begin{exercise}
Consider $D_4$.  Let $h$ be the action that reflects (i.e., flips over) the square over the horizontal midline and let $v$ be the action that reflects the square over the vertical midline.  Also, we'll use $r^2$ as shorthand for the action $rr$ that does $r$ twice in a row.  Which of the following are subgroups of $D_4$?  In each case, justify your answer.  If a subset is a subgroup, try to find a minimal set of generators.  Also, determine whether you can see the subgroups in our Cayley diagram for $D_4$.
\begin{enumerate}
\item[(a)] $\{e, r^2\}$
\item[(b)] $\{e,h\}$
\item[(c)] $\{e, h, v\}$
\item[(d)] $\{e, h, v, r^2\}$
\end{enumerate}
\end{exercise}

The last subgroup above is often referred to as the \textbf{Klein four-group} and is denoted by $V_4$.

Let's introduce a group we haven't seen yet.  We define the \textbf{quaternion group} to be the group $Q_8=\{1,-1,i,-i,j,-j,k,-k\}$ having the following Cayley diagram with generators $i, j, -1$.  In this case, 1 is the do-nothing action.

\tikzstyle{vert} = [circle, draw, fill=grey,inner sep=0pt, minimum size=6mm]
\tikzstyle{b} = [draw,very thick,blue,-stealth]
\tikzstyle{r} = [draw, very thick, red,-stealth]
\tikzstyle{g} = [draw, very thick, green, stealth-stealth]

\begin{center}
\begin{tikzpicture}[scale=1.5,auto]
\node (1) at (135:2) [vert] {{\scriptsize $1$}};
\node (i) at (45:2) [vert] {\scriptsize {$i$}};
\node (k) at (-45:2) [vert] {{\scriptsize $k$}};
\node (j) at (-135:2) [vert] {{\scriptsize $j$}};
\node (-1) at (135:1) [vert] {{\scriptsize $-1$}};
\node (-i) at (45:1) [vert] {{\scriptsize $-i$}};
\node (-k) at (-45:1) [vert] {{\scriptsize $-k$}};
\node (-j) at (-135:1) [vert] {{\scriptsize $-j$}};

\path[b] (1) to (i);
\path[b] (i) to (-1);
\path[b] (-1) to (-i);
\path[b] (-i) to (1);

\path[b] (-j) to (-k);
\path[b] (-k) to (j);
\path[b] (j) to (k);
\path[b] (k) to (-j);

\path[r] (-k) to (-i);
\path[r] (-i) to (k);
\path[r] (k) to (i);
\path[r] (i) to (-k);

\path[r] (1) to (j);
\path[r] (j) to (-1);
\path[r] (-1) to (-j);
\path[r] (-j) to (1);

\path[g] (1) to (-1);
\path[g] (j) to (-j);
\path[g] (i) to (-i);
\path[g] (k) to (-k);

\end{tikzpicture}
\end{center}

Notice that I didn't mention what the actions actually do.  For now, let's not worry about that.  The relationship between the arrows and vertices tells us everything we need to know.  Also, let's take it for granted that $Q_8$ actually is a group.

\begin{exercise}
Consider $Q_8$.
\begin{enumerate}
\item[(a)] Which arrows correspond to which generators in our Cayley diagram for $Q_8$?
\item[(b)] What is $i^2$ equal to?  That is, what element of $\{1,-1,i,-i,j,-j,k,-k\}$ is $i^2$ equal to?  How about $i^3$, $i^4$, and $i^5$?
\item[(c)] What are $j^2$, $j^3$, $j^4$, and $j^5$ equal to?
\item[(d)] What is $(-1)^2$ equal to?
\item[(e)] What is $ij$ equal to?  How about $ji$?
\item[(f)] Can you determine what $k^2$ and $ik$ are equal to?
\item[(g)] Can you identify a generating set consisting of only two elements?  Can you find more than one?
\item[(h)] What subgroups of $Q_8$ can you see in the Cayley diagram (with generators $i, j, -1$)?
\item[(i)] Find a subgroup of $Q_8$ that you cannot see in the Cayley diagram.
\end{enumerate}
\end{exercise}

\end{section}

\begin{section}{Isomorphisms}

Coming soon...
\end{section}