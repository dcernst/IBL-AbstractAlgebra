\chapter{Families of Groups}
\label{chapter:families}
\thispagestyle{empty}

In this chapter we will explore a few families of groups.

\begin{section}{Cyclic Groups}

Recall that if $(G,*)$ is a group and $a\in G$, then the subgroup generated by $a$ is given by
\[
\langle a\rangle =\{a^n:n\in\mathbb{Z}\}.
\]
According to Theorem~\ref{thm:smallest_subgroup_containing_S}, $\langle a\rangle$ is the smallest subgroup containing $a$.  

\begin{exercise}\label{exer:finite_pos_exps}
Notice that in the definition for $\langle a\rangle$, we allow the exponent to be negative.  Explain why we only need to use positive exponents when $\langle a\rangle$ is a finite group.  What about when $\langle a\rangle$ is infinite?
\end{exercise}

\begin{exercise}
Suppose $\langle a\rangle$ is a finite group.  Since $\langle a\rangle$ is a group in its own right, we can draw a Cayley diagram for this group.  Using the generator $a$, what does the Cayley diagram for $\langle a\rangle$ look like?
\end{exercise}

Even when $\langle a\rangle$ infinite, we call $\langle a\rangle$ the \textbf{cyclic group generated by $a$}.  In the finite case, the Cayley diagram with generator $a$ gives us a good indication where the word cyclic comes from.

\begin{definition}
Suppose $(G,*)$ is a group and let $a\in G$.  We define the \textbf{order} of $a$, written $|a|$, to be the order of $\langle a\rangle$.  That is,
\[
|a|=|\langle a\rangle|.
\]
\end{definition}

\begin{exercise}
What is the order of the identity in any group?
\end{exercise}

\begin{exercise}\label{exer:computing_orders}
Find the orders of each of the elements in each of the following groups.
\begin{multicols}{2}
\begin{enumerate}
\item[(a)] $S_2$
\item[(b)] $R_3$
\item[(c)] $R_4$
\item[(d)] $V_4$
\item[(e)] $R_5$
\item[(f)] $R_6$
\item[(g)] $D_3$
\item[(h)] $R_7$
\item[(i)] $R_8$
\item[(j)] $D_4$
\item[(k)] $Q_8$
\end{enumerate}
\end{multicols}
\end{exercise}

\begin{exercise}
Consider the group $(\mathbb{Z},+)$.  What is the order of 1?  Are there any elements in $\mathbb{Z}$ with finite order?
\end{exercise}

\begin{exercise}
Consider the group of invertible $2\times 2$ matrices with real number entries under the operation of matrix multiplication.  This group is denoted $\mathrm{GL}_2(\mathbb{R})$.  Find the order of each of the following elements in this group.
\begin{multicols}{3}
\begin{enumerate}
\item[(a)] $\begin{bmatrix} 0 & -1\\ -1 & 0\end{bmatrix}$
\item[(b)] $\begin{bmatrix} 1 & 1\\ 0 & 1\end{bmatrix}$
\item[(c)] $\begin{bmatrix} 3 & 0\\ 0 & 2\end{bmatrix}$
\end{enumerate}
\end{multicols}
\end{exercise}

The next theorem is related to Exercise~\ref{exer:finite_pos_exps}.

\begin{theorem}
Suppose $(G,*)$ is a finite group and let $a\in G$.  Then there exists a positive integer $m$ such that $a^m=e$, where $e$ is the identity in $G$. 
\end{theorem}

In fact, we can say something even stronger.  You likely noticed the following fact while exploring Exercise~\ref{exer:computing_orders}.

\begin{theorem}
Suppose $(G,*)$ is a finite group and let $a\in G$.  Then the order of $a$ is the smallest positive integer $n$ such that $a^n=e$.
\end{theorem}

\begin{problem}
Suppose $(G,*)$ is a group $a\in G$ with $|a|=n$.  For what other exponents $k$ will it be true that $a^k=e$?
\end{problem}

We are finally ready to introduce our family of interest for this section.

\begin{definition}
Suppose $(G,*)$ is a group.  Then we say that $G$ is a \textbf{cyclic group} if and only if there exists $a\in G$ such that $\langle a\rangle =G$.
\end{definition}

It is clear that if $G$ is cyclic with generator $a$, then $|G|=|a|$.  In fact, if $a\in G$, the converse is true, as well.

\begin{exercise}
Determine which of the groups from Exercise~\ref{exer:computing_orders} are cyclic.  If the group is cyclic, find at least one generator.
\end{exercise}

\begin{exercise}
Determine whether each of the following groups are cyclic.  If the group is cyclic, find at least one generator.
\begin{multicols}{2}
\begin{enumerate}
\item[(a)] $(\mathbb{Z},+)$
\item[(b)] $(\mathbb{R},+)$
\item[(c)] $(\mathbb{R}^+,\cdot)$
\item[(d)] $(\{6^n\mid n\in\mathbb{Z}\},\cdot)$
\end{enumerate}
\end{multicols}
\begin{enumerate}
\item[(e)] $\textrm{GL}_2(\mathbb{R})$ under matrix multiplication
\item[(f)] $\{(\cos(\pi/4) +i\sin(\pi/4))^n\mid n\in \mathbb{Z}\}$ under multiplication of complex numbers
\end{enumerate}
\end{exercise}

\begin{theorem}
If $(G,*)$ is a cyclic group, then $G$ is abelian.
\end{theorem}

\begin{exercise}
Provide an example of a finite group that is abelian but not cyclic.
\end{exercise}

\begin{exercise}
Provide an example of an infinite group that is abelian but not cyclic.
\end{exercise}

\begin{theorem}
Suppose $(G,*)$ is a cyclic group with generator $a$.  Then $a^{-1}$ is also a generator for $G$.
\end{theorem}

\begin{theorem}
Suppose $(G,*)$ is a cyclic group such that $G$ has exactly one element that generates all of $G$.  Then the order of $G$ is at most order 2.   
\end{theorem}

\begin{theorem}
Suppose $(G,*)$ is a group such that $G$ has no proper nontrivial subgroups.  Then $G$ is cyclic.
\end{theorem}

\begin{theorem}\label{thm:infinite_cyclic_groups}
Suppose $(G,*)$ is an infinite cyclic group.  Then $G$ is isomorphic to $\mathbb{Z}$ (under the operation of addition).
\end{theorem}

Recall that for $n\geq3$, $R_n$ is the group of rotational symmetries of a regular $n$-gon, where the operation is composition of actions.

\begin{theorem}
For all $n\geq 3$, $R_n$ is cyclic.
\end{theorem}

\begin{theorem}\label{thm:finite_cyclic_groups}
Suppose $(G,*)$ is a finite cyclic group of order $n\geq 2$.  Then $G$ is isomorphic to $R_n$ if $n\geq 3$ and $S_2$ if $n=2$.
\end{theorem}

\begin{exercise}
Suppose $(G,*)$ is a finite cyclic group of order $n$ with generator $a$.  If we write down the group table for $G$ using $e, a, a^2, \ldots, a^{n-1}$ as the labels for the rows and columns.  Are there any interesting patterns in the table?
\end{exercise}

The upshot of Theorems~\ref{thm:infinite_cyclic_groups} and \ref{thm:finite_cyclic_groups} is that up to isomorphism, we know exactly what all of the cyclic groups are.

See class notes written on board for the rest of this section.

%Division algorithm
%subgroups of cyclic groups are cyclic
%Z_n (motivation: structure of R_n is really just addition of exponents
% Finite cyclic groups describe the symmetry of objects that only exhibit rotational symmetry.

\end{section}

\begin{section}{Dihedral Groups}
See class notes written on board for this section.
\end{section}

\begin{section}{Symmetric Groups}
Recall the group $S_3$ from Exercise~\ref{exer:S_3}.  This group acts on three coins that are in a row by rearranging their positions (but not flipping them over). This group is an example of a \textbf{symmetric group}.  In general, the symmetric group on $n$ objects is the set of permutations that rearranges the $n$ objects.  The group operation is composition of permutations.  Let's be a little more formal.

\begin{definition}
A \textbf{permutation of a set $A$} is a function $\sigma:A\to A$ that is both one-to-one and onto.
\end{definition}

You should take a moment to convince yourself that the formal definition of a permutation agrees with the notin of rearranging the set of objects.  The do-nothing action is the identity permutation, i.e., $\sigma(a)=a$ for all $a\in A$.  There are many ways to represent a permutation.  One visual way is using \textbf{permutation diagrams}, which we will introduce via examples.

Consider the following diagrams:
\begin{multicols}{2}
\[\alpha=\begin{pdiag}{5}{1}
\put(0.2,2.3){{1}}\put(1.2,2.3){{2}}\put(2.2,2.3){{3}}\put(3.2,2.3){{4}}\put(4.2,2.3){{5}} 
\pdmap{1}{2}\pdmap{2}{3}\pdmap{3}{4}\pdmap{4}{5}\pdmap{5}{1}\pdendmapfill 
\end{pdiag}\]

\bigskip

\[\beta=\begin{pdiag}{5}{1}
\put(0.2,2.3){{1}}\put(1.2,2.3){{2}}\put(2.2,2.3){{3}}\put(3.2,2.3){{4}}\put(4.2,2.3){{5}} 
\pdmap{2}{4}\pdmap{4}{3}\pdmap{3}{2}\pdendmapfill 
\end{pdiag}\]

\[\sigma=\begin{pdiag}{5}{1}
\put(0.2,2.3){{1}}\put(1.2,2.3){{2}}\put(2.2,2.3){{3}}\put(3.2,2.3){{4}}\put(4.2,2.3){{5}} 
\pdtrans{1}{3}\pdmap{2}{5}\pdmap{5}{4}\pdmap{4}{2}\pdendmapfill 
\end{pdiag}\]

\bigskip

\[\gamma=\begin{pdiag}{5}{1}
\put(0.2,2.3){{1}}\put(1.2,2.3){{2}}\put(2.2,2.3){{3}}\put(3.2,2.3){{4}}\put(4.2,2.3){{5}}
\pdtrans{1}{5}\pdendmapfill 
\end{pdiag}\]
\end{multicols}
\noindent Each of these diagrams represents a permutation on five objects.  I've given the permutations the names $\alpha$, $\beta$, $\sigma$, and $\gamma$.  The intention is to read the diagrams from the top down.  The numbers labeling the nodes along the top are identifying position.  Following an edge from the top row of nodes to the bottom row of nodes tells us what position an object moves to.  It is important to remember that the numbers are referring to the position of an object, not the object itself.  For example, $\beta$ is the permutation that sends the object in the second position to the fourth position, the object in the third position to the second position, and the object in the fourth position to the third position.  Moreover, the permutation $\beta$ doesn't do anything to the objects in positions 1 and 5.

\begin{exercise}
Describe in words what the permutations $\sigma$ and $\gamma$ do.
\end{exercise}

\begin{exercise}
Draw the permutation diagram for the do-nothing permutation on 5 objects.  This is called the \textbf{identity permutation}. What does the identity permutation diagram look like in general for arbitrary $n$?
\end{exercise}

\begin{definition}
The set of all permutations on $n$ objects is denoted by $S_n$.
\end{definition}

\begin{exercise}
Draw all the permutation diagrams for the permutations in $S_3$.
\end{exercise}

\begin{exercise}
How many distinct permutations are there in $S_4$?  How about $S_n$ for any $n\in \mathbb{N}$?
\end{exercise}

If $S_n$ is going to be a group, we need to know how to compose permutations.  This is easy to do using the permutation diagrams.  Consider the permutations $\alpha$ and $\beta$ from earlier.  We can represent the composition $\alpha \circ \beta$ via

\bigskip

\[\alpha \circ \beta=\begin{pdiag}{5}{2}
\put(0.2,4.3){{1}}\put(1.2,4.3){{2}}\put(2.2,4.3){{3}}\put(3.2,4.3){{4}}\put(4.2,4.3){{5}} 
\pdname{\tiny \beta}\pdmap{2}{4}\pdmap{4}{3}\pdmap{3}{2}\pdendmapfill 
\pdname{\tiny \alpha}\pdmap{1}{2}\pdmap{2}{3}\pdmap{3}{4}\pdmap{4}{5}\pdmap{5}{1}\pdendmapfill 
\end{pdiag}=
\begin{pdiag}{5}{1}
\put(0.2,2.3){{1}}\put(1.2,2.3){{2}}\put(2.2,2.3){{3}}\put(3.2,2.3){{4}}\put(4.2,2.3){{5}} 
\pdmap{1}{2}\pdmap{2}{5}\pdmap{5}{1}\pdendmapfill 
\end{pdiag}.\]
\noindent As you can see by looking at the figure, to compose two permutations, you stack the one that goes first in the composition (e.g., $\beta$ in the example above) on top of the other and just follow the edges from the top through the middle to the bottom.  If you think about how function composition works, this is very natural.  The resulting permutation is determined by where we begin and where we end in the composition.

We already know that the order of composition matters for functions, and so it should matter for the composition of permutations. To make this crystal clear, let's compose $\alpha$ and $\beta$ in the opposite order.  We see that

\bigskip

\[\beta \circ \alpha=\begin{pdiag}{5}{2}
\put(0.2,4.3){{1}}\put(1.2,4.3){{2}}\put(2.2,4.3){{3}}\put(3.2,4.3){{4}}\put(4.2,4.3){{5}} 
\pdname{\tiny \alpha}\pdmap{1}{2}\pdmap{2}{3}\pdmap{3}{4}\pdmap{4}{5}\pdmap{5}{1}\pdendmapfill 
\pdname{\tiny \beta}\pdmap{2}{4}\pdmap{4}{3}\pdmap{3}{2}\pdendmapfill 
\end{pdiag}=
\begin{pdiag}{5}{1}
\put(0.2,2.3){{1}}\put(1.2,2.3){{2}}\put(2.2,2.3){{3}}\put(3.2,2.3){{4}}\put(4.2,2.3){{5}} 
\pdmap{1}{4}\pdmap{4}{5}\pdmap{5}{1}\pdendmapfill 
\end{pdiag}.\]

\noindent The moral of the story is that composition of permutations does not necessarily commute.

\begin{exercise}
Consider $\alpha$, $\beta$, $\sigma$, and $\gamma$ from earlier.  Can you find a pair of permutations that do commute?  Can you identify any features about your diagrams that indicate why the commuted?
\end{exercise}

\begin{exercise}
Fix $n\in\mathbb{N}$.  Convince yourself that any $\rho\in S_n$ composed with the identity permutation (in either order) equals $\rho$.
\end{exercise}

If $S_n$ is going to be a group, we need to know what the inverse of a permutation is.

\begin{problem}
Given a permutation $\rho\in S_n$, describe a method for constructing $\rho^{-1}$.  Briefly justify that $\rho \circ \rho^{-1}$ will yield the identity permutation.
\end{problem}

At this point, we have all the ingredients we need to prove that $S_n$ forms a group under composition of permutations.

\begin{theorem}
The set of permutations on $n$ objects forms a group under the operation of composition.  That is, $(S_n,\circ)$ is a group.  Moreover, $|S_n|=n!$.
\end{theorem}

Note that it is standard convention to omit the composition symbol when writing down compositions in $S_n$.  For example, we will simply write $\alpha\beta$ to denote $\alpha \circ \beta$.

Permutation diagrams are fun to play with, but we need a more efficient way of encoding information.  One way to do this is using \textbf{cycle notation}.  Consider $\alpha, \beta, \sigma$, and $\gamma$ in $S_{5}$ from the previous examples.  Below I have indicated what each permutation is equal to using cycle notation.

\begin{align*}\alpha=&\begin{pdiag}{5}{1} 
\pdmap{1}{2}\pdmap{2}{3}\pdmap{3}{4}\pdmap{4}{5}\pdmap{5}{1}\pdendmapfill 
\end{pdiag}=\ (1,2,3,4,5)\\
\\
\beta=&\begin{pdiag}{5}{1} 
\pdmap{2}{4}\pdmap{4}{3}\pdmap{3}{2}\pdendmapfill 
\end{pdiag}=\ (2,4,3)\\
\\
\sigma=&\begin{pdiag}{5}{1} 
\pdtrans{1}{3}\pdmap{2}{5}\pdmap{5}{4}\pdmap{4}{2}\pdendmapfill 
\end{pdiag}=\ (1,3)(2,5,4)\\
\\
\gamma=&\begin{pdiag}{5}{1} 
\pdtrans{1}{5}\pdendmapfill 
\end{pdiag}=\ (1,5)
\end{align*}
Each string of numbers enclosed by parentheses is called a \textbf{cycle} and if the string of numbers has length $k$, then we call it a $k$-cycle.  For example, $\alpha$ consists of a single 5-cycle, whereas $\sigma$ consists of one 2-cycle and one 3-cycle.  In the case of $\sigma$, we say that $\sigma$ is the product of two \textbf{disjoint cycles}.  

One observation that you hopefully made is that if an object in position $i$ remains unchanged, then we don't bother listing that number in the cycle notation.  However, if we wanted to, we could use the 1-cycle $(i)$ to denote this.  For example, we could write $\beta=(1)(2,3,4)(5)$.  In particular, we could denote the identity permutation in $S_5$ using $(1)(2)(3)(4)(5)$.  Yet, it is common to simply use $(1)$ to denote the identity in $S_n$ for all $n$.

Notice that the first number we choose to write down for a given cycle is arbitrary.  However, the numbers that follow are not negotiable.  Typically, we would use the smallest possible number first, but this is not necessary.  For example, the cycle $(2,4,7)$ could also be written as $(4,7,2)$ or $(7,2,4)$.

\begin{exercise}\label{exer:S3}
Write down all 6 elements in $S_3$ using cycle notation.
\end{exercise}

\begin{exercise}\label{exer:S4}
Write down all 24 elements in $S_4$ using cycle notation.
\end{exercise}

Suppose $\sigma\in S_n$.  Since $\sigma$ is one-to-one and onto, it is clear that it is possible to write $\sigma$ as a product of disjoint cycles such that each $i\in\{1,2,\ldots, n\}$ appears exactly once.

Let's see if we can figure out how to multiply elements of $S_n$ using cycle notation.  Consider the permutations $\alpha=(1,3,2)$ and $\beta=(3,4)$ in $S_4$.  To compute the composition $\alpha\beta=(1,3,2)(3,4)$, let's explore what happens in each position.  Since we are doing function composition, we should work our way from right to left.  Since 1 does not appear in the cycle notation for $\beta$, we know that $\beta(1)=1$ (i.e., $\beta$ maps 1 to 1).  Now, we see what $\alpha(1)=3$.  Thus, the composition $\alpha\beta$ maps 1 to 3 (since $\alpha\beta(1)=\alpha(\beta(1))=\alpha(1)=3$).  Next, we should return to $\beta$ and see what happens to 3---which is where we ended a moment ago.  We see that $\beta$ maps 3 to 4 and then $\alpha$ maps 4 to 4 (since 4 does not appear in the cycle notation for $\alpha$).  So, $\alpha\beta(3)=4$.  Continuing this way, we see that $\beta$ maps 4 to 3 and $\alpha$ maps 3 to 2, and so $\alpha\beta$ maps 4 to 2.  Lastly, since $\beta(2)=2$ and $\alpha(2)=1$, we have $\alpha\beta(2)=1$.  Putting this altogether, we see that $\alpha\beta=(1,3,4,2)$.  Now, you should try a few.  Things get a little trickier if the composition of two permutations results in a permutation consisting of more than a single cycle.

\begin{exercise}
Consider $\alpha$, $\beta$, $\sigma$, and $\gamma$ for which we drew the permutation diagrams.  Using cycle notation, compute each of the following.
\begin{multicols}{2}
\begin{enumerate}
\item[(a)] $\alpha\gamma$
\item[(b)] $\alpha^2$
\item[(c)] $\alpha^3$
\item[(d)] $\alpha^4$
\item[(e)] $\alpha^5$
\item[(f)] $\sigma\alpha$
\item[(g)] $\alpha^{-1}\sigma^{-1}$
\item[(h)] $\beta^2$
\item[(i)] $\beta^3$
\item[(j)] $\beta\gamma\alpha$
\item[(k)] $\sigma^3$
\item[(l)] $\sigma^6$
\end{enumerate}
\end{multicols}
\end{exercise}

\begin{exercise}
Write down the group table for $S_3$ using cycle notation.
\end{exercise}

In Exercise~\ref{exer:S4}, one of the permutations you should have written down is $(1,2)(3,4)$.  This is a product of two disjoint 2-cycles.  It is worth pointing out that each cycle is a permutation in its own right.  That is, $(1,2)$ and $(3,4)$ are each permutations.  It just so happens that their composition does not ``simplify" any further.  Moreover, these two disjoint 2-cycles commute since $(1,2)(3,4)=(3,4)(1,2)$.  In fact, this phenomenon is always true.

\begin{theorem}
Suppose $\alpha$ and $\beta$ are two disjoint cycles.  Then $\alpha\beta=\beta\alpha$.  That is, products of disjoint cycles commute.
\end{theorem}

Computing the order of a permutation is fairly easy using cycle notation once we figure out how to do it for a single cycle.  In fact, you've probably already guessed at the following theorem.

\begin{theorem}
Suppose $\alpha\in S_n$ such that $\alpha$ consists of a single $k$-cycle.  Then $|\alpha|=k$.
\end{theorem}

\begin{theorem}
Suppose $\alpha\in S_n$ such that $\alpha$ consists of $m$ disjoint cycles of lengths $k_1,\ldots, k_m$.  Then $|\alpha|=\lcm(k_1,\ldots, k_m)$.\footnote{Recall that $\lcm(k_1,\ldots, k_m)$ is the \textbf{least common multiple} of $\{k_1,\ldots, k_m\}$.} 
\end{theorem}

\begin{problem}
Is the previous theorem true if we do not require the cycles to be disjoint?  Justify your answer.
\end{problem}

\begin{exercise}
Compute the orders of all the elements in $S_3$.  See Exercise~\ref{exer:S3}.
\end{exercise}

\begin{exercise}
Compute the orders of all the elements in $S_4$.  See Exercise~\ref{exer:S4}.
\end{exercise}

\begin{exercise}
What is the order of $(1,4,7)(2,5)(3,6,8,9)$?
\end{exercise}

\begin{exercise}
Draw the subgroup lattice for $S_3$.
\end{exercise}

\begin{exercise}
Now, using $(1,2)$ and $(1,2,3)$ as generators, draw the Cayley diagram for $S_3$.  Look familiar?
\end{exercise}

It turns out that the subgroups of symmetric groups play an important role in group theory.

\begin{definition}
Every subgroup of a symmetric group is called a \textbf{permutation group}.
\end{definition}

The proof of the following theorem isn't too bad, but for now we'll take it for granted.

\begin{theorem}[Cayley's Theorem]
Every finite group is isomorphic to some permutation group.  In particular, if $(G,*)$ is a group of order $n$, then $G$ is isomorphic to a subgroup of $S_n$.
\end{theorem}

Cayley's Theorem guarantees that every finite group is isomorphic to a permutation group and it turns out that there is a rather simple algorithm for constructing the corresponding permutation group.  I'll briefly explain an example and then let you try a couple.

Consider the Klein four-group $V_4=\{e,v,h,vh\}$.  Recall that $V_4$ has the following group table.

\begin{center}
\begin{tabu}{c|[2pt]c|c|c|c}
$*$ & $e$ & $v$ & $h$ & $vh$ \\ \tabucline[2pt]{-}
$e$ & $e$ & $v$ & $h$ & $vh$ \\
\hline $v$ & $v$ & $e$ & $vh$ & $h$  \\
\hline $h$ & $h$ & $vh$ & $e$ & $v$\\
\hline $vh$ & $vh$ & $h$ & $v$ & $e$
\end{tabu}
\end{center}

If we number the elements $e,v,h,$ and $vh$ as $1,2,3,$ and $4$, respectively, then we obtain the following table.

\begin{center}
\begin{tabu}{c|[2pt]c|c|c|c}
 & $1$ & $2$ & $3$ & $4$ \\ \tabucline[2pt]{-}
$1$ & $1$ & $2$ & $3$ & $4$ \\
\hline $2$ & $2$ & $1$ & $4$ & $3$  \\
\hline $3$ & $3$ & $4$ & $1$ & $2$\\
\hline $4$ & $4$ & $3$ & $2$ & $1$
\end{tabu}
\end{center}

\noindent Comparing each of the four columns to the leftmost column, we can obtain the corresponding permutations.  In particular, we obtain
\begin{align*}
e&\leftrightarrow (1)\\
v&\leftrightarrow (1,2)(3,4)\\
h&\leftrightarrow (1,3)(2,4)\\
vh&\leftrightarrow(1,4)(2,3). 
\end{align*}
Do you see where these permutations came from?  The claim is that the set of permutations $\{(1),(1,2)(3,4),(1,3)(2,4),(1,4)(2,3)\}$ is isomorphic to $V_4$.  In this particular case, it's fairly clear that this is true.  However, it takes some work to prove that this process will always result in an isomorphic permutation group.  In fact, verifying the algorithm is essentially the proof of Cayley's Theorem. 

Since there are potentially many ways to rearrange the rows and columns of a given table, it should be clear that there are potentially many isomorphisms that could result from the algorithm described above.

Here's another way to obtain a permutation group that is isomorphic to a given group.  Let's consider $V_4$ again.  Recall that $V_4$ is a subset of $D_4$, which is the symmetry group for a square.  Alternatively, $V_4$ is the symmetry group for a non-square rectangle.  Label the corners of the rectangle 1, 2, 3, and 4 by starting in the upper left corner and continuing clockwise.  Recall that $v$ is the action that reflects the rectangle over the vertical midline.  The result of this action is that the corners labeled by 1 and 2 switch places and the corners labeled by 3 and 4 switch places.  Thus, $v$ corresponds to the permutation $(1,2)(3,4)$.  Similarly, $h$ swaps the corners labeled by 1 and 4 and the corners labeled by 2 and 3, and so $h$ corresponds to the permutation $(1,4)(2,3)$.  Notice that this is not the same answer we got earlier and that's okay as there may be many permutation representations for a given group.  Lastly, $vh$ rotates the rectangle $180^{\circ}$ which sends ends up swapping corners labeled 1 and 3 and swapping corners labeled by 2 and 4.  Therefore, $vh$ corresponds to the permutation $(1,3)(2,4)$.

\begin{exercise}
Find a permutation group that is isomorphic to $D_4$.
\end{exercise}

\begin{exercise}
Find a permutation group that is isomorphic to $\mathbb{Z}_6$.
\end{exercise}

More coming soon...

%\begin{exercise}
%Using $(1,2)$ and $(2,3)$ as generators, draw the Cayley diagram for $S_3$.  Look familiar?
%\end{exercise}
%
%\begin{exercise}
%Draw the subgroup lattice for $S_4$.
%\end{exercise}
%
%\begin{exercise}
%Verify that $\{(1,2), (2,3), (3,4)\}$ is a generating set for $S_4$.
%\end{exercise}
%
%The elements in the generating set in the previous exercise are called \textbf{adjacent transpositions}.  Every 2-cycle is also called a \textbf{transposition}.  A transposition is called adjacent if the entries are one unit apart.  For example, $(3,7)$ is a non-adjacent transposition while $(6,7)$ is an adjacent transposition.
%
%It turns out that the set of transpositions in $S_n$ is a generating set

%even and odd permutations...in next section.

\end{section}

\begin{section}{Alternating Groups}

Coming soon...

\end{section}