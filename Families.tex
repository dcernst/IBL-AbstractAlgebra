\chapter{Families of Groups}
\label{chapter:families}
\thispagestyle{empty}

In this chapter we will explore a few families of groups.

\begin{section}{Cyclic Groups}

Recall that if $(G,*)$ is a group and $a\in G$, then the subgroup generated by $a$ is given by
\[
\langle a\rangle =\{a^n:n\in\mathbb{Z}\}.
\]
According to Theorem~\ref{thm:smallest_subgroup_containing_S}, $\langle a\rangle$ is the smallest subgroup containing $a$.  

\begin{exercise}\label{exer:finite_pos_exps}
Notice that in the definition for $\langle a\rangle$, we allow the exponent to be negative.  Explain why we only need to use positive exponents when $\langle a\rangle$ is a finite group.  What about when $\langle a\rangle$ is infinite?
\end{exercise}

\begin{exercise}
Suppose $\langle a\rangle$ is a finite group.  Since $\langle a\rangle$ is a group in its own right, we can draw a Cayley diagram for this group.  Using the generator $a$, what does the Cayley diagram for $\langle a\rangle$ look like?
\end{exercise}

Even when $\langle a\rangle$ infinite, we call $\langle a\rangle$ the \textbf{cyclic group generated by $a$}.  In the finite case, the Cayley diagram with generator $a$ gives us a good indication where the word cyclic comes from.

\begin{definition}
Suppose $(G,*)$ is a group and let $a\in G$.  We define the \textbf{order} of $a$, written $|a|$, to be the order of $\langle a\rangle$.  That is,
\[
|a|=|\langle a\rangle|.
\]
\end{definition}

\begin{exercise}
What is the order of the identity in any group?
\end{exercise}

\begin{exercise}\label{exer:computing_orders}
Find the orders of each of the elements in each of the following groups.
\begin{multicols}{2}
\begin{enumerate}
\item[(a)] $S_2$
\item[(b)] $R_3$
\item[(c)] $R_4$
\item[(d)] $V_4$
\item[(e)] $R_5$
\item[(f)] $R_6$
\item[(g)] $D_3$
\item[(h)] $R_7$
\item[(i)] $R_8$
\item[(j)] $D_4$
\item[(k)] $Q_8$
\end{enumerate}
\end{multicols}
\end{exercise}

\begin{exercise}
Consider the group $(\mathbb{Z},+)$.  What is the order of 1?  Are there any elements in $\mathbb{Z}$ with finite order?
\end{exercise}

\begin{exercise}
Consider the group of invertible $2\times 2$ matrices with real number entries under the operation of matrix multiplication.  This group is denoted $\mathrm{GL}_2(\mathbb{R})$.  Find the order of each of the following elements in this group.
\begin{multicols}{3}
\begin{enumerate}
\item[(a)] $\begin{bmatrix} 0 & -1\\ -1 & 0\end{bmatrix}$
\item[(b)] $\begin{bmatrix} 1 & 1\\ 0 & 1\end{bmatrix}$
\item[(c)] $\begin{bmatrix} 3 & 0\\ 0 & 2\end{bmatrix}$
\end{enumerate}
\end{multicols}
\end{exercise}

The next theorem is related to Exercise~\ref{exer:finite_pos_exps}.

\begin{theorem}
Suppose $(G,*)$ is a finite group and let $a\in G$.  Then there exists a positive integer $m$ such that $a^m=e$, where $e$ is the identity in $G$. 
\end{theorem}

In fact, we can say something even stronger.  You likely noticed the following fact while exploring Exercise~\ref{exer:computing_orders}.

\begin{theorem}
Suppose $(G,*)$ is a finite group and let $a\in G$.  Then the order of $a$ is the smallest positive integer $n$ such that $a^n=e$.
\end{theorem}

\begin{problem}
Suppose $(G,*)$ is a group $a\in G$ with $|a|=n$.  For what other exponents $k$ will it be true that $a^k=e$?
\end{problem}

We are finally ready to introduce our family of interest for this section.

\begin{definition}
Suppose $(G,*)$ is a group.  Then we say that $G$ is a \textbf{cyclic group} if and only if there exists $a\in G$ such that $\langle a\rangle =G$.
\end{definition}

It is clear that if $G$ is cyclic with generator $a$, then $|G|=|a|$.  In fact, if $a\in G$, the converse is true, as well.

\begin{exercise}
Determine which of the groups from Exercise~\ref{exer:computing_orders} are cyclic.  If the group is cyclic, find at least one generator.
\end{exercise}

\begin{exercise}
Determine whether each of the following groups are cyclic.  If the group is cyclic, find at least one generator.
\begin{multicols}{2}
\begin{enumerate}
\item[(a)] $(\mathbb{Z},+)$
\item[(b)] $(\mathbb{R},+)$
\item[(c)] $(\mathbb{R}^+,\cdot)$
\item[(d)] $(\{6^n\mid n\in\mathbb{Z}\},\cdot)$
\end{enumerate}
\end{multicols}
\begin{enumerate}
\item[(e)] $\textrm{GL}_2(\mathbb{R})$ under matrix multiplication
\item[(f)] $\{(\cos(\pi/4) +i\sin(\pi/4))^n\mid n\in \mathbb{Z}\}$ under multiplication of complex numbers
\end{enumerate}
\end{exercise}

\begin{theorem}
If $(G,*)$ is a cyclic group, then $G$ is abelian.
\end{theorem}

\begin{exercise}
Provide an example of a finite group that is abelian but not cyclic.
\end{exercise}

\begin{exercise}
Provide an example of an infinite group that is abelian but not cyclic.
\end{exercise}

\begin{theorem}
Suppose $(G,*)$ is a cyclic group with generator $a$.  Then $a^{-1}$ is also a generator for $G$.
\end{theorem}

\begin{theorem}
Suppose $(G,*)$ is a cyclic group such that $G$ has exactly one element that generates all of $G$.  Then the order of $G$ is at most order 2.   
\end{theorem}

\begin{theorem}
Suppose $(G,*)$ is a group such that $G$ has no proper nontrivial subgroups.  Then $G$ is cyclic.
\end{theorem}

\begin{theorem}\label{thm:infinite_cyclic_groups}
Suppose $(G,*)$ is an infinite cyclic group.  Then $G$ is isomorphic to $\mathbb{Z}$ (under the operation of addition).
\end{theorem}

Recall that for $n\geq3$, $R_n$ is the group of rotational symmetries of a regular $n$-gon, where the operation is composition of actions.

\begin{theorem}
For all $n\geq 3$, $R_n$ is cyclic.
\end{theorem}

\begin{theorem}\label{thm:finite_cyclic_groups}
Suppose $(G,*)$ is a finite cyclic group of order $n$.  Then $G$ is isomorphic to $R_n$.
\end{theorem}

\begin{exercise}
Suppose $(G,*)$ is a finite cyclic group of order $n$ with generator $a$.  If we write down the group table for $G$ using $e, a, a^2, \ldots, a^{n-1}$ as the labels for the rows and columns.  Are there any interesting patterns in the table?
\end{exercise}

The upshot of Theorems~\ref{thm:infinite_cyclic_groups} and \ref{thm:finite_cyclic_groups} is that up to isomorphism, we know exactly what all of the cyclic groups are.

More coming soon...

%Division algorithm
%subgroups of cyclic groups are cyclic
%Z_n (motivation: structure of R_n is really just addition of exponents
% Finite cyclic groups describe the symmetry of objects that only exhibit rotational symmetry.

\end{section}

\begin{section}{Dihedral Groups}
Coming soon...
\end{section}

\begin{section}{Symmetric Groups}
Coming soon...
\end{section}

