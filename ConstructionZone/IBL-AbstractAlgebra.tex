\documentclass[12pt,oneside]{book}

\usepackage[scale=2]{ccicons}
\usepackage{enumitem}
\usepackage{multicol}
\usepackage[labelsep=period]{caption}
\usepackage[labelformat=simple,labelfont={}]{subcaption}
\usepackage{makecell}
\usepackage{pdiag}
\usepackage{xcolor}
\usepackage{colortbl}
\usepackage{tikz}
\usetikzlibrary{arrows,automata,positioning}
\usepackage{rotating}
\usepackage[notextcomp]{kpfonts} 
\usepackage{graphicx}
\usepackage{eurosym}
\usepackage{amsfonts}
\usepackage{amsmath}
\usepackage{amssymb}
\usepackage{stmaryrd}
\usepackage{wasysym}
\usepackage{amsthm}
\usepackage[margin=1in]{geometry}
\usepackage[hang,flushmargin,symbol*]{footmisc}
\usepackage{color}
\definecolor{darkblue}{rgb}{0, 0, .6}
\definecolor{grey}{rgb}{.7, .7, .7}
\usepackage[breaklinks]{hyperref}
\hypersetup{
	colorlinks=true,
	linkcolor=darkblue,
	anchorcolor=darkblue,
	citecolor=darkblue,
	pagecolor=darkblue,
	urlcolor=darkblue,
	pdftitle={},
	pdfauthor={}
}

\usepackage{fancyhdr}
\pagestyle{fancy}
\lhead{\leftmark}
\chead{}
\rhead{}
\lfoot{}
\cfoot{\thepage}
\rfoot{}

\theoremstyle{definition}
\newtheorem{theorem}{Theorem}[chapter]
\newtheorem{acknowledgement}[theorem]{Acknowledgement}
\newtheorem{algorithm}[theorem]{Algorithm}
\newtheorem{axiom}[theorem]{Axiom}
\newtheorem{case}[theorem]{Case}
\newtheorem{claim}[theorem]{Claim}
\newtheorem{conclusion}[theorem]{Conclusion}
\newtheorem{condition}[theorem]{Condition}
\newtheorem{conjecture}[theorem]{Conjecture}
\newtheorem{corollary}[theorem]{Corollary}
\newtheorem{criterion}[theorem]{Criterion}
\newtheorem{definition}[theorem]{Definition}
\newtheorem{example}[theorem]{Example}
\newtheorem{exercise}[theorem]{Exercise}
\newtheorem{journal}[theorem]{Journal}
\newtheorem{lemma}[theorem]{Lemma}
\newtheorem{notation}[theorem]{Notation}
\newtheorem{problem}[theorem]{Problem}
\newtheorem{proposition}[theorem]{Proposition}
\newtheorem{remark}[theorem]{Remark}
\newtheorem{solution}[theorem]{Solution}
\newtheorem{summary}[theorem]{Summary}
\newtheorem{skeleton}[theorem]{Skeleton Proof}
\newtheorem{activity}[theorem]{Activity}
\newtheorem{intuitivedef}[theorem]{Intuitive Definition}

\newsavebox{\savepar}
\newenvironment{textbox}{\noindent\begin{lrbox}{\savepar}\begin{minipage}[c]{.98\textwidth}}{\end{minipage}\end{lrbox}\fcolorbox{black}{white}{\usebox{\savepar}}}

\newcommand{\dom}{\operatorname{Dom}}
\newcommand{\codom}{\operatorname{Codom}}
\newcommand{\range}{\operatorname{Rng}}
\newcommand{\Spin}{\operatorname{Spin}}
\newcommand{\lcm}{\operatorname{lcm}}

\renewcommand\thesubfigure{(\alph{subfigure})}

\begin{document}

\title{An Inquiry-Based Approach to Abstract Algebra}
\author{Dana C.~Ernst, PhD\\
Northern Arizona University}
\date{Fall 2019}

\maketitle

\noindent\copyright{ \the\year\ Dana C.~Ernst.  Some Rights Reserved.\\

\bigskip

\noindent This book is intended to be a task sequence for an undergraduate abstract algebra course that utilizes an inquiry-based learning (IBL) approach.  You can find the most up-to-date version of these notes on GitHub:
\begin{center}
\url{http://dcernst.github.io/IBL-AbstractAlgebra/}
\end{center}
I would be thrilled if you used these notes and improved them. If you make any modifications, you can either make a pull request on GitHub or submit the improvements via email.  You are also welcome to fork the source and modify the notes for your purposes as long as you maintain the license below.

\bigskip

\noindent This work is licensed under the Creative Commons Attribution-Share Alike 4.0 United States License.  You may copy, distribute, display, and perform this copyrighted work, but only if you give credit to Dana C.~Ernst, and all derivative works based upon it must be published under the Creative Commons Attribution-Share Alike 4.0 International License. Please attribute this work to Dana C.~Ernst, Mathematics Faculty at Northern Arizona University, \url{dana.ernst@nau.edu}. To view a copy of this license, visit
\begin{center}
\url{https://creativecommons.org/licenses/by-sa/4.0/}
\end{center}
or send a letter to Creative Commons, 171 Second Street, Suite 300, San Francisco, California, 94105, USA.}

\medskip

\begin{center}
\ccbysa
\end{center}

\noindent Here is a partial list of people (in alphabetical order by last name) that I need to thank for supplying content, advice, and feedback.
\begin{itemize}
\item \href{https://thalestriangles.blogspot.com}{Joshua Bowman} (Pepperdine University). The Preface is a modified version of Joshua's \emph{An IBL preface}.
\item \href{https://faculty.bentley.edu/details.asp?uname=ncarter}{Nathan Carter} (Bentley University). Nathan's excellent book \emph{Visual Group Theory} has had a huge impact on my approach to teaching abstract algebra.
\item \href{http://home.snc.edu/andershendrickson/}{Anders Hendrickson} (St.~Norbert College). Anders is the original author of the content in Appendix~\ref{appendix:elements_of_style}: Elements of Style for Proofs. The current version in Appendix~\ref{appendix:elements_of_style} is a result of modifications made by myself with some suggestions from Dave Richeson.
\item \href{http://users.dickinson.edu/~richesod/}{Dave Richeson} (Dickinson College). Dave is responsible for much of the content in Appendix~\ref{appendix:fancy_math_terms}: Fancy Mathematical Terms and Appendix~\ref{appendix:definitions}: Definitions in Mathematics. 
\item \href{http://webpages.csus.edu/wiscons/}{Josh Wiscons} (CSU Sacramento) and \href{http://emp.byui.edu/woodruffb/}{Ben Woodruff} (BYU Idaho). In the early stages of development, Josh and Ben were instrumental the development of these notes.
\end{itemize}

\tableofcontents

\chapter*{Preface}
\addcontentsline{toc}{chapter}{\protect\numberline{}Preface}

You are the creators. This book is a guide.

This book will not show you how to solve all the problems that are presented, but it should \emph{enable} you to find solutions, on your own and working together. The material you are about to study did not come together fully formed at a single moment in history. It was composed gradually over the course of centuries, with various mathematicians building on the work of others, improving the subject while increasing its breadth and depth.

Mathematics is essentially a human endeavor. Whatever you may believe about the true nature of mathematics---does it exist eternally in a transcendent Platonic realm, or is it contingent upon our shared human consciousness?---our \emph{experience} of mathematics is temporal, personal, and communal. Like music, mathematics that is encountered only as symbols on a page remains inert. Like music, mathematics must be created in the moment, and it takes time and practice to master each piece. The creation of mathematics takes place in writing, in conversations, in explanations, and most profoundly in the mental construction of its edifices on the basis of reason and observation.

To continue the musical analogy, you might think of these notes like a performer's score. Much is included to direct you towards particular ideas, but much is missing that can only be supplied by you: participation in the creative process that will make those ideas come alive. Moreover, your success will depend on the pursuit of both \emph{individual} excellence and \emph{collective} achievement. Like a musician in an orchestra, you should bring your best work and be prepared to blend it with others' contributions.

In any act of creation, there must be room for experimentation, and thus allowance for mistakes, even failure. A key goal of our community is that we support each other---sharpening each other's thinking but also bolstering each other's confidence---so that we can make failure a \emph{productive} experience. Mistakes are inevitable, and they should not be an obstacle to further progress. It's normal to struggle and be confused as you work through new material. Accepting that means you can keep working even while feeling stuck, until you overcome and reach even greater accomplishments.

This book is a guide. You are the creators.
\chapter{Introduction}

\begin{section}{What is Abstract Algebra?}

Abstract algebra is the subject area of mathematics that studies algebraic structures, such as groups, rings, fields, modules, vector spaces, and algebras. This course is an introduction to abstract algebra. We will spend most of our time studying groups. Group theory is the study of symmetry, and is one of the most beautiful areas in all of mathematics. It arises in puzzles, visual arts, music, nature, the physical and life sciences, computer science, cryptography, and of course, throughout mathematics. This course will cover the basic concepts of group theory, and a special effort will be made to emphasize the intuition behind the concepts and motivate the subject matter.  In the last few weeks of the semester, we will also introduce rings and fields.

\end{section}

\begin{section}{An Inquiry-Based Approach}

In a typical course, math or otherwise, you sit and listen to a lecture. (Hopefully) These lectures are polished and well-delivered. You may have often been lured into believing that the instructor has opened up your head and is pouring knowledge into it. I absolutely love lecturing and I do believe there is value in it, but I also believe that in reality most students do not learn by simply listening. You must be active in the learning process. I'm sure each of you have said to yourselves, ``Hmmm, I understood this concept when the professor was going over it, but now that I am alone, I am lost." In order to promote a more active participation in your learning, we will incorporate ideas from an educational philosophy called inquiry-based learning (IBL).

Loosely speaking, IBL is a student-centered method of teaching mathematics that engages students in sense-making activities.  Students are given tasks requiring them to solve problems, conjecture, experiment, explore, create, communicate.  Rather than showing facts or a clear, smooth path to a solution, the instructor guides and mentors students via well-crafted problems through an adventure in mathematical discovery.  Effective IBL courses encourage deep engagement in rich mathematical activities and provide opportunities to collaborate with peers (either through class presentations or group-oriented work).

Perhaps this is sufficiently vague, but I believe that there are two essential elements to IBL.  Students should as much as possible be responsible for:
\begin{enumerate}
\item Guiding the acquisition of knowledge, and
\item Validating the ideas presented.  That is, students should not be looking to the instructor as the sole authority.
\end{enumerate}
\noindent For additional information, check out my blog post, \href{http://maamathedmatters.blogspot.com/2013/05/what-heck-is-ibl.html}{What the Heck is IBL?}

Much of the course will be devoted to students proving theorems on the board and a significant portion of your grade will be determined by how much mathematics you produce. I use the word ``produce" because I believe that the best way to learn mathematics is by doing mathematics. Someone cannot master a musical instrument or a martial art by simply watching, and in a similar fashion, you cannot master mathematics by simply watching; you must do mathematics!

Furthermore, it is important to understand that proving theorems is difficult and takes time. You should not expect to complete a single proof in 10 minutes. Sometimes, you might have to stare at the statement for an hour before even understanding how to get started. 

In this course, everyone will be required to
\begin{itemize}
\item read and interact with course notes on your own;
\item write up quality proofs to assigned problems;
\item present proofs on the board to the rest of the class;
\item participate in discussions centered around a student's presented proof;
\item call upon your own prodigious mental faculties to respond in flexible, thoughtful, and creative ways to problems that may seem unfamiliar on first glance.
\end{itemize}
\noindent As the semester progresses, it should become clear to you what the expectations are. This will be new to many of you and there may be some growing pains associated with it.

Lastly, it is highly important to respect learning and to respect other people's ideas.  Whether you disagree or agree, please praise and encourage your fellow classmates.  Use ideas from others as a starting point rather than something to be judgmental about.  Judgement is not the same as being judgmental.  Helpfulness, encouragement, and compassion are highly valued.

\end{section}

\begin{section}{Rules of the Game}
You should \emph{not} look to resources outside the context of this course for help. That is, you should not be consulting the Internet, other texts, other faculty, or students outside of our course. On the other hand, you may use each other, the course notes, me, and your own intuition.  In this class, earnest failure outweighs counterfeit success; you need not feel pressure to hunt for solutions outside your own creative and intellectual reserves.  For more details, check out the Syllabus.

\end{section}

\begin{section}{Structure of the Notes}

As you read the notes, you will be required to digest the material in a meaningful way.  It is your responsibility to read and understand new definitions and their related concepts.  However, you will be supported in this sometimes difficult endeavor. In addition, you will be asked to complete problems aimed at solidifying your understanding of the material.  Most importantly, you will be asked to make conjectures, produce counterexamples, and prove theorems.

The items labeled as \textbf{Definition} and \textbf{Example} are meant to be read and digested.  However, the items labeled as \textbf{Problem}, \textbf{Theorem}, and \textbf{Corollary} require action on your part.  Items labeled as \textbf{Problem} are sort of a mixed bag. Some Problems are computational in nature and aimed at improving your understanding of a particular concept while others ask you to provide a counterexample for a statement if it is false or to provide a proof if the statement is true. Items with the \textbf{Theorem} and \textbf{Corollary} designation are mathematical facts and the intention is for you to produce a valid proof of the given statement.  The main difference between a \textbf{Theorem} and a \textbf{Corollary} is that corollaries are typically statements that follow quickly from a previous theorem.  In general, you should expect corollaries to have very short proofs.  However, that doesn't mean that you can't produce a more lengthy yet valid proof of a corollary.

It is important to point out that there are very few examples in the notes.  This is intentional.  One of the goals of the items labeled as \textbf{Problem} is for you to produce the examples.

Lastly, there are many situations where you will want to refer to an earlier definition, problem, theorem, or corollary.  In this case, you should reference the statement by number.  For example, you might write something like, ``By Theorem~\ref{thm:order_element_divides_group_order}, we see that\ldots."

\end{section}

\begin{section}{Some Minimal Guidance}
Especially in the opening sections, it won't be clear what facts from your prior experience in mathematics we are ``allowed" to use.  Unfortunately, addressing this issue is difficult and is something we will sort out along the way.  However, in general, here are some minimal guidelines to keep in mind.  

First, there are times when we will need to do some basic algebraic manipulations.  You should feel free to do this whenever the need arises.  But you should show sufficient work along the way.  You do not need to write down justifications for basic algebraic manipulations (e.g., adding 1 to both sides of an equation, adding and subtracting the same amount on the same side of an equation, adding like terms, factoring, basic simplification, etc.).  

On the other hand, you do need to make explicit justification of the logical steps in a proof.  When necessary, you should cite a previous definition, theorem, etc. by number.

Unlike the experience many of you had writing proofs in geometry, our proofs will be written in complete sentences.  You should break sections of a proof into paragraphs and use proper grammar.  There are some pedantic conventions for doing this that I will point out along the way.  Initially, this will be an issue that most students will struggle with, but after a few weeks everyone will get the hang of it.

Ideally, you should rewrite the statements of theorems before you start the proof.  Moreover, for your sake and mine, you should label the statement with the appropriate number.  I will expect you to indicate where the proof begins by writing ``\emph{Proof.}" at the beginning.  Also, we will conclude our proofs with the standard ``proof box" (i.e., $\square$ or $\blacksquare$), which is typically right-justified.

Lastly, every time you write a proof, you need to make sure that you are making your assumptions crystal clear.  Sometimes there will be some implicit assumptions that we can omit, but at least in the beginning, you should get in the habit of stating your assumptions up front.  Typically, these statements will start off ``Assume\ldots" or ``Let\ldots".  

This should get you started.  We will discuss more as the semester progresses.  Now, go have fun and kick some butt!

\end{section}
\chapter{An Introduction to Groups}
\label{chapter:intro_groups}
\thispagestyle{empty}

%%----------------------------%%

One of the major topics of this course is \textbf{groups}.  The area of mathematics that is concerned with groups is called \textbf{group theory}. Loosely speaking, group theory is the study of symmetry, and in my opinion is one of the most beautiful areas in all of mathematics. It arises in puzzles, visual arts, music, nature, the physical and life sciences, computer science, cryptography, and of course, throughout mathematics.

%%----------------------------%%

\begin{section}{A First Example}\label{sec:first_example}

%%----------------------------%%

Let's begin our study by developing some intuition about what groups actually are.  To get started, we will explore the game Spinpossible\texttrademark, which used to be available for iOS and Android devices\footnote{If you'd like to play the game, try going here: \url{https://www.kongregate.com/games/spinpossible}.}.  The game is played on a $3\times 3$ board of scrambled tiles numbered 1 to 9, each of which may be right-side-up or up-side-down. The objective of the game is to return the board to the standard configuration where tiles are arranged in numerical order and right-side-up. This is accomplished by a sequence of ``spins", where a spin consists of rotating an $m\times n$ subrectangle by 180$^\circ$. The goal is to minimize the number of spins used.  The following figure depicts a scrambled board on the left and the solved board on the right.  The sequence of arrows is used to denote some sequence of spins that transforms the scrambled board into the solved board.

\begin{center}
\begin{tabular}{c}\includegraphics[width=1.5in]{scramble1.PNG}\end{tabular}
{\large $\xrightarrow{?} \cdots \xrightarrow{?}$}
\begin{tabular}{c}\includegraphics[width=1.5in]{scramble4.PNG}\end{tabular}
\end{center}

Let's play with an example.  Suppose we start with the following scrambled board.

\begin{center}
\begin{tikzpicture}[every node/.style={minimum size=.65cm}]
  \node [draw] (1) {\rotatebox{180}{$\underline{2}$}};
  \node [draw, right=0cm of 1] (2) {\rotatebox{180}{$\underline{9}$}};
  \node [draw, right=0cm of 2] (3) {\rotatebox{180}{$\underline{1}$}};
  \node [draw, below=0cm of 1] (4) {$\underline{4}$};
  \node [draw, right=0cm of 4] (5) {\rotatebox{180}{$\underline{6}$}};
  \node [draw, right=0cm of 5] (6) {$\underline{5}$};
  \node [draw, below=0cm of 4] (7) {$\underline{7}$};
  \node [draw, right=0cm of 7] (8) {\rotatebox{180}{$\underline{3}$}};
  \node [draw, right=0cm of 8] (9) {$\underline{8}$};
\end{tikzpicture}
\end{center}

\noindent The underlines on the numbers are meant to help us tell whether a tile is right-side-up or up-side-down.  Our goal is to use a sequence of spins to unscramble the board.  Before we get started, let's agree on some conventions.  When we refer to \emph{tile $n$}, we mean the actual tile that is labeled by the number $n$ regardless of its position and orientation on the board.  On the other hand, \emph{position $n$} will refer to the position on the board that tile $n$ is supposed to be in when the board has been unscrambled.  For example, in the board above, tile 1 is in position 3 and tile 7 happens to be in position 7.  

It turns out that there are multiple ways to unscramble this board, but I have one particular sequence in mind.  First, let's spin the rectangle determined by the two rightmost columns.  Here's what we get.  I've shaded the subrectangle that we are spinning.

\begin{center}
\begin{tabular}{c}
\begin{tikzpicture}[every node/.style={minimum size=.65cm}]
  \node [draw] (1) {\rotatebox{180}{$\underline{2}$}};
  \node [draw, fill=blue!40, right=0cm of 1] (2) {\rotatebox{180}{$\underline{9}$}};
  \node [draw, fill=blue!40, right=0cm of 2] (3) {\rotatebox{180}{$\underline{1}$}};
  \node [draw, below=0cm of 1] (4) {$\underline{4}$};
  \node [draw, fill=blue!40, right=0cm of 4] (5) {\rotatebox{180}{$\underline{6}$}};
  \node [draw, fill=blue!40, right=0cm of 5] (6) {$\underline{5}$};
  \node [draw, below=0cm of 4] (7) {$\underline{7}$};
  \node [draw, fill=blue!40, right=0cm of 7] (8) {\rotatebox{180}{$\underline{3}$}};
  \node [draw, fill=blue!40, right=0cm of 8] (9) {$\underline{8}$};
\end{tikzpicture}
\end{tabular}
%
{\large $\rightarrow$}
%
\begin{tabular}{c}
\begin{tikzpicture}[every node/.style={minimum size=.65cm}]
  \node [draw] (1) {\rotatebox{180}{$\underline{2}$}};
  \node [draw, right=0cm of 1] (2) {\rotatebox{180}{$\underline{8}$}};
  \node [draw, right=0cm of 2] (3) {$\underline{3}$};
  \node [draw, below=0cm of 1] (4) {$\underline{4}$};
  \node [draw, right=0cm of 4] (5) {\rotatebox{180}{$\underline{5}$}};
  \node [draw, right=0cm of 5] (6) {$\underline{6}$};
  \node [draw, below=0cm of 4] (7) {$\underline{7}$};
  \node [draw, right=0cm of 7] (8) {$\underline{1}$};
  \node [draw, right=0cm of 8] (9) {$\underline{9}$};
\end{tikzpicture}
\end{tabular}
\end{center}

\noindent Okay, now let's spin the middle column.

\begin{center}
\begin{tabular}{c}
\begin{tikzpicture}[every node/.style={minimum size=.65cm}]
  \node [draw] (1) {\rotatebox{180}{$\underline{2}$}};
  \node [draw, fill=blue!40, right=0cm of 1] (2) {\rotatebox{180}{$\underline{8}$}};
  \node [draw, right=0cm of 2] (3) {$\underline{3}$};
  \node [draw, below=0cm of 1] (4) {$\underline{4}$};
  \node [draw, fill=blue!40, right=0cm of 4] (5) {\rotatebox{180}{$\underline{5}$}};
  \node [draw, right=0cm of 5] (6) {$\underline{6}$};
  \node [draw, below=0cm of 4] (7) {$\underline{7}$};
  \node [draw, fill=blue!40, right=0cm of 7] (8) {$\underline{1}$};
  \node [draw, right=0cm of 8] (9) {$\underline{9}$};
\end{tikzpicture}
\end{tabular}
%
{\large $\rightarrow$}
%
\begin{tabular}{c}
\begin{tikzpicture}[every node/.style={minimum size=.65cm}]
  \node [draw] (1) {\rotatebox{180}{$\underline{2}$}};
  \node [draw, right=0cm of 1] (2) {\rotatebox{180}{$\underline{1}$}};
  \node [draw, right=0cm of 2] (3) {$\underline{3}$};
  \node [draw, below=0cm of 1] (4) {$\underline{4}$};
  \node [draw, right=0cm of 4] (5) {$\underline{5}$};
  \node [draw, right=0cm of 5] (6) {$\underline{6}$};
  \node [draw, below=0cm of 4] (7) {$\underline{7}$};
  \node [draw, right=0cm of 7] (8) {$\underline{8}$};
  \node [draw, right=0cm of 8] (9) {$\underline{9}$};
\end{tikzpicture}
\end{tabular}
\end{center}

\noindent Hopefully, you can see that we are really close to unscrambling the board.  All we need to do is spin the rectangle determined by the tiles in positions 1 and 2.

\begin{center}
\begin{tabular}{c}
\begin{tikzpicture}[every node/.style={minimum size=.65cm}]
  \node [draw, fill=blue!40] (1) {\rotatebox{180}{$\underline{2}$}};
  \node [draw, fill=blue!40, right=0cm of 1] (2) {\rotatebox{180}{$\underline{1}$}};
  \node [draw, right=0cm of 2] (3) {$\underline{3}$};
  \node [draw, below=0cm of 1] (4) {$\underline{4}$};
  \node [draw, right=0cm of 4] (5) {$\underline{5}$};
  \node [draw, right=0cm of 5] (6) {$\underline{6}$};
  \node [draw, below=0cm of 4] (7) {$\underline{7}$};
  \node [draw, right=0cm of 7] (8) {$\underline{8}$};
  \node [draw, right=0cm of 8] (9) {$\underline{9}$};
\end{tikzpicture}
\end{tabular}
%
{\large $\rightarrow$}
%
\begin{tabular}{c}
\begin{tikzpicture}[every node/.style={minimum size=.65cm}]
  \node [draw] (1) {$\underline{1}$};
  \node [draw, right=0cm of 1] (2) {$\underline{2}$};
  \node [draw, right=0cm of 2] (3) {$\underline{3}$};
  \node [draw, below=0cm of 1] (4) {$\underline{4}$};
  \node [draw, right=0cm of 4] (5) {$\underline{5}$};
  \node [draw, right=0cm of 5] (6) {$\underline{6}$};
  \node [draw, below=0cm of 4] (7) {$\underline{7}$};
  \node [draw, right=0cm of 7] (8) {$\underline{8}$};
  \node [draw, right=0cm of 8] (9) {$\underline{9}$};
\end{tikzpicture}
\end{tabular}
\end{center}

\noindent Putting all of our moves together, here is what we have.

\begin{center}
\begin{tabular}{c}
\begin{tikzpicture}[every node/.style={minimum size=.65cm}]
  \node [draw] (1) {\rotatebox{180}{$\underline{2}$}};
  \node [draw, fill=blue!40, right=0cm of 1] (2) {\rotatebox{180}{$\underline{9}$}};
  \node [draw, fill=blue!40, right=0cm of 2] (3) {\rotatebox{180}{$\underline{1}$}};
  \node [draw, below=0cm of 1] (4) {$\underline{4}$};
  \node [draw, fill=blue!40, right=0cm of 4] (5) {\rotatebox{180}{$\underline{6}$}};
  \node [draw, fill=blue!40, right=0cm of 5] (6) {$\underline{5}$};
  \node [draw, below=0cm of 4] (7) {$\underline{7}$};
  \node [draw, fill=blue!40, right=0cm of 7] (8) {\rotatebox{180}{$\underline{3}$}};
  \node [draw, fill=blue!40, right=0cm of 8] (9) {$\underline{8}$};
\end{tikzpicture}
\end{tabular}
%
{\large $\rightarrow$}
%
\begin{tabular}{c}
\begin{tikzpicture}[every node/.style={minimum size=.65cm}]
  \node [draw] (1) {\rotatebox{180}{$\underline{2}$}};
  \node [draw, fill=blue!40, right=0cm of 1] (2) {\rotatebox{180}{$\underline{8}$}};
  \node [draw, right=0cm of 2] (3) {$\underline{3}$};
  \node [draw, below=0cm of 1] (4) {$\underline{4}$};
  \node [draw, fill=blue!40, right=0cm of 4] (5) {\rotatebox{180}{$\underline{5}$}};
  \node [draw, right=0cm of 5] (6) {$\underline{6}$};
  \node [draw, below=0cm of 4] (7) {$\underline{7}$};
  \node [draw, fill=blue!40, right=0cm of 7] (8) {$\underline{1}$};
  \node [draw, right=0cm of 8] (9) {$\underline{9}$};
\end{tikzpicture}
\end{tabular}
%
{\large $\rightarrow$}
%
\begin{tabular}{c}
\begin{tikzpicture}[every node/.style={minimum size=.65cm}]
  \node [draw, fill=blue!40] (1) {\rotatebox{180}{$\underline{2}$}};
  \node [draw, fill=blue!40, right=0cm of 1] (2) {\rotatebox{180}{$\underline{1}$}};
  \node [draw, right=0cm of 2] (3) {$\underline{3}$};
  \node [draw, below=0cm of 1] (4) {$\underline{4}$};
  \node [draw, right=0cm of 4] (5) {$\underline{5}$};
  \node [draw, right=0cm of 5] (6) {$\underline{6}$};
  \node [draw, below=0cm of 4] (7) {$\underline{7}$};
  \node [draw, right=0cm of 7] (8) {$\underline{8}$};
  \node [draw, right=0cm of 8] (9) {$\underline{9}$};
\end{tikzpicture}
\end{tabular}
%
{\large $\rightarrow$}
%
\begin{tabular}{c}
\begin{tikzpicture}[every node/.style={minimum size=.65cm}]
  \node [draw] (1) {$\underline{1}$};
  \node [draw, right=0cm of 1] (2) {$\underline{2}$};
  \node [draw, right=0cm of 2] (3) {$\underline{3}$};
  \node [draw, below=0cm of 1] (4) {$\underline{4}$};
  \node [draw, right=0cm of 4] (5) {$\underline{5}$};
  \node [draw, right=0cm of 5] (6) {$\underline{6}$};
  \node [draw, below=0cm of 4] (7) {$\underline{7}$};
  \node [draw, right=0cm of 7] (8) {$\underline{8}$};
  \node [draw, right=0cm of 8] (9) {$\underline{9}$};
\end{tikzpicture}
\end{tabular}
\end{center}
In this case, we were able to solve the scrambled board in 3 moves.  It's not immediately obvious, but it turns out that there is no way to unscramble the board in fewer than 3 spins.  However, there is at least one other solution that involves exactly 3 spins.

\begin{problem}\label{prob:number_spinpossible_boards}
How many scrambled $3\times 3$ Spinpossible boards are there?  To answer this question, you will need to rely on some counting principles such as factorials. In this context, we want to include the solved board as one of the scrambled boards---it's just not very scrambled.
\end{problem}

\begin{problem}\label{prob:counting_spins}
How many spins are there?
\end{problem}

It's useful to have some notation. Let $s_{ij}$ (with $i\leq j$) denote the spin that rotates the subrectangle that has position $i$ in the upper-left corner and position $j$ in the lower-right corner.  As an example, the sequence of spins that we used above to unscramble our initial scrambled board is
\[
s_{29}\to s_{28} \to s_{12}.
\]
As you noticed in Problem~\ref{prob:counting_spins}, we can also rotate a single tile. Every spin of the form $s_{ii}$ is called a \emph{toggle}. For example, $s_{44}$ toggles the tile in position 4.

We can think of each spin as a function and since we are doing spins on top of spins, every sequence of spins corresponds to a composition of functions. We will follow the standard convention of function composition that says the function on the right goes first.  In this case, our previous sequence of spins becomes $s_{12} \circ s_{28}  \circ s_{29}$, which we abbreviate as $s_{12} s_{28} s_{29}$. This might take some getting used to, but just remember that it is just like function notation---stuff on the right goes first. We will refer to expressions like $s_{12} s_{28} s_{29}$ as \textbf{words} in the alphabet $\{s_{ij}\mid i\leq j\}$.  Our words will always consist of a finite number of spins.

Every word consisting of spins corresponds to a function that takes a scrambled board as input and returns a scrambled board. We say that the words ``act on" the scrambled boards. For each word, there is an associated net action. For example, the word $s_{12} s_{23} s_{12}$ corresponds to swapping the positions but not orientation of the tiles in positions 1 and 3.  You should take the time to verify this for yourself. Sometimes it is difficult to describe what the net action associated to a word is, but there is always some corresponding net action nonetheless.

It is worth pointing out that $s_{12} s_{23} s_{12}$ is not itself a spin.  However, sometimes a composition of spins will yield a spin.  For example, the net action of $s_{12} s_{11} s_{12}$ is toggling the tile in position 2.  That is, $s_{12} s_{11} s_{12}$ and $s_{22}$ are two different words that correspond to the same net action. In this case, we write $s_{12} s_{11} s_{12}=s_{22}$, where the equality is referring to the net action as opposed to the words themselves. The previous example illustrates that multiple words may represent the same net action. 

\begin{problem}\label{prob:3_different_spins}
Find a sequence of 3 spins that is different from the one we described earlier that unscrambles the following board. Write your answer as a word consisting of spins.
\begin{center}
\begin{tikzpicture}[every node/.style={minimum size=.65cm}]
  \node [draw] (1) {\rotatebox{180}{$\underline{2}$}};
  \node [draw, right=0cm of 1] (2) {\rotatebox{180}{$\underline{9}$}};
  \node [draw, right=0cm of 2] (3) {\rotatebox{180}{$\underline{1}$}};
  \node [draw, below=0cm of 1] (4) {$\underline{4}$};
  \node [draw, right=0cm of 4] (5) {\rotatebox{180}{$\underline{6}$}};
  \node [draw, right=0cm of 5] (6) {$\underline{5}$};
  \node [draw, below=0cm of 4] (7) {$\underline{7}$};
  \node [draw, right=0cm of 7] (8) {\rotatebox{180}{$\underline{3}$}};
  \node [draw, right=0cm of 8] (9) {$\underline{8}$};
\end{tikzpicture}
\end{center}
\end{problem}

\begin{problem}
What is the net action that corresponds to the word $s_{23} s_{12} s_{23}$? What can you conclude about $s_{23} s_{12} s_{23}$ compared to 
$s_{12} s_{23} s_{12}$?
\end{problem}

We can also use exponents to abbreviate.  For example, $s_2^2$ is the same as $s_2 s_2$ (which in this case has the net action of doing nothing) and $(s_1 s_2)^2$ is the same as $s_1 s_2 s_1 s_2$.

\begin{problem}\label{prob:braid_relation}
It turns out that there is an even simpler word (i.e., a shorter word) that yields the same net action as $(s_1 s_2)^2$. Can you find one?
\end{problem}

Define $\Spin_{3\times 3}$ to be the collection of net actions that we can obtain from words consisting of spins.  We say that the set of spins \textbf{generates} $\Spin_{3\times 3}$ and we refer to the set of spins as a \textbf{generating set} for $\Spin_{3\times 3}$.  

\begin{problem}
Suppose $s_{x_1}s_{x_2}\cdots s_{x_n}$ and $s_{y_1}s_{y_2}\cdots s_{y_m}$ are both words consisting of spins. Then the corresponding net actions, say $u$ and $v$, respectively, are elements of $\Spin_{3\times 3}$. Prove that the composition of the actions $u$ and $v$ is an element of $\Spin_{3\times 3}$.
\end{problem}

The previous problem tells us that the composition of two net actions from $\Spin_{3\times 3}$ results in another net action in $\Spin_{3\times 3}$. Formally, we say that $\Spin_{3\times 3}$ is \textbf{closed} under composition.

It is clear that we can construct an infinite number of words consisting of spins, but since there are a finite number of ways to rearrange the positions and orientations of the tiles of the $3\times 3$ board, there are only a finite number of net actions arising from these words.  That is, $\Spin_{3\times 3}$ is a finite set of functions.

\begin{problem}
Verify that $\Spin_{3\times 3}$ contains an \textbf{identity} function, i.e., a function whose net action is ``do nothing." What happens if we compose a net action from $\Spin_{3\times 3}$ with the identity?
\end{problem}

A natural question to ask is whether every possible scrambled Spinpossible board can be unscrambled using only spins.  In other words, is $\Spin_{3\times 3}$ sufficient to unscramble every scrambled board? It turns out that the answer is yes.

\begin{problem}\label{prob:kindergarten_algorthim}
Verify that $\Spin_{3\times 3}$ is sufficient to unscramble every scrambled board by describing an algorithm that will always unscramble a scrambled board.  It does not matter whether your algorithm is efficient.  That is, we don't care how many steps it takes to unscramble the board as long as it works in a finite number of steps.  Using your algorithm, what is the maximum number of spins required to unscramble any scrambled board?
\end{problem}

In a 2011 paper, Alex Sutherland and Andrew Sutherland (a father and son team) present a number of interesting results about Spinpossible and list a few open problems. You can find the paper at \url{http://arxiv.org/abs/1110.6645}. As a side note, Alex is one of the developers of the game and his father, Andrew, is a mathematics professor at MIT. Using a brute-force computer algorithm, the Sutherlands verified that every scrambled $3\times 3$ Spinpossible board can be solved in at most 9 moves. However, a human readable mathematical proof of this fact remains elusive.  By the way, mathematics is chock full of open problems and you can often get to the frontier of what is currently known without too much trouble.  Mathematicians are in the business of solving open problems.

Instead of unscrambling boards, we can act on the solved board with an action from $\Spin_{3\times 3}$ to obtain a scrambled board.  Problem~\ref{prob:kindergarten_algorthim} tells us that we can use $\Spin_{3\times 3}$ to get from the solved board to any scrambled board. In fact, starting with the solved board makes it clear that there is a one-to-one correspondence between net actions and scrambled boards.

\begin{problem}
What is the size of $\Spin_{3\times 3}$? That is, how many net actions are in $\Spin_{3\times 3}$?
\end{problem}

Let's make a couple more observations.  First, every spin is reversible. That is, every spin has an \textbf{inverse}.  In the case of Spinpossible, we can just apply the same spin again to undo it.  For example, $s_{12}^2$ is the same as doing nothing. This means that the inverse of $s_{12}$, denoted $s_{12}^{-1}$, is $s_{12}$ itself. Symbolically, we write $s_{12}^{-1}=s_{12}$. Remember that we are exploring the game Spinpossible---it won't always be the case that repeating an action will reverse the action. 

In the same vein, every sequence of spins is reversible. For example, if we apply $s_{12} s_{23}$ (i.e., do $s_{23}$ first followed by $s_{12}$), we could undo the net action by applying $s_{23} s_{12}$ because
\[
(s_{12} s_{23})^{-1}=s_{23}^{-1} s_{12}^{-1}=s_{23} s_{12}
\]
since $s_{23}^{-1}=s_{23}$ and $s_{12}^{-1}=s_{12}$.  Notice that the first equality is an instantiation of the ``socks and shoes theorem", which states that if $f$ and $g$ are functions with compatible domain and codomain, then
\[
(f\circ g)^{-1} = g^{-1}\circ f^{-1}.
\]
The upshot is that the net action that corresponds to a word consisting of spins can be reversed by applying ``socks and shoes" and is itself an action.

\begin{problem}
Imagine we started with the solved board and then you scrambled the board according to some word consisting of spins.  Let's call this word $w$. How could you obtain the solved board from the scrambled board determined by $w$? How is this related to $w^{-1}$?
\end{problem}

There is one detail we have been sweeping under the rug.  Notice that every time we wrote down a word consisting of two or more spins, we didn't bother to group pairs of adjacent spins using parentheses.  Recall that the composition of functions with compatible domains and codomains is \textbf{associative} (see Theorem~\ref{thm:function_comp_associative}).  That is, if $f$, $g$, and $h$ are functions with compatible domains and codomains, then
\[
(f\circ g)\circ h = f\circ (g\circ h).
\]
Since composition of spins is really just function composition, composition of spins is also associative.  And since the spins generate $\Spin_{3\times 3}$, the composition of net actions from $\Spin_{3\times 3}$ is associative, as well.

\begin{problem}
Does the order in which you apply spins matter?  Does it always matter?  Let's be as specific as possible.  If the order in which we apply two spins does not matter, then we say that the spins \textbf{commute}.  However, if the order does matter, then the spins do not commute.  When will two spins commute?  When will they not commute?  Provide some specific examples.
\end{problem}

In the previous problem, you discovered that the composition of two spins may or may not commute.  Since the spins generate $\Spin_{3\times 3}$, the composition of two net actions may or may not commute.  We say that $\Spin_{3\times 3}$ is not commutative.

Let's collect our key observations about $\Spin_{3\times 3}$.
\begin{enumerate}[label=\rm{(\arabic*)}]
\item \textbf{Generating Set:} The set of spins generates $\Spin_{3\times 3}$.  That is, every net action from $\Spin_{3\times 3}$ corresponds to a word consisting of spins.\footnote{The case of Spinpossible is a little misleading. Since each spin is its own inverse, we never need to write words consisting of spins with inverses. However, as we shall see later, there are situations outside the context of Spinpossible where we will need to utilize inverses of elements from a generating set.} 
\item \textbf{Closure:} The composition of any two net actions from $\Spin_{3\times 3}$ results in a net action from $\Spin_{3\times 3}$.
\item \textbf{Associative:} The composition of net actions from $\Spin_{3\times 3}$ is associative.
\item \textbf{Identity:} There is an identity in $\Spin_{3\times 3}$ whose corresponding net action is ``do nothing".
\item \textbf{Inverses:} Every net action from $\Spin_{3\times 3}$ has an inverse net action in $\Spin_{3\times 3}$. Composing a net action and its inverse results in the identity.
\item The composition of two net actions from $\Spin_{3\times 3}$ may or may not commute.
\end{enumerate}

It turns out that $\Spin_{3\times 3}$ is an example of a group. Loosely speaking, a \textbf{group} is a set together with a method for combining two elements together that satisfies conditions (2), (3), (4), and (5) above.  More formally, a group is a nonempty set together with an associative binary operation such that the set contains an identity element and every element in the set has an inverse that is also in the set.  As we shall see, groups can have a variety of generating sets, possibly of different sizes. Also, some groups are commutative and some groups are not.

Before closing out this section, let's tackle a few more interesting problems concerning Spinpossible. We say that a generating set for a group is a \textbf{minimal generating set} if removing any single element from the generating set results in a set that no longer generates the group.

\begin{problem}
Determine whether the set of spins is a minimal generating set for $\Spin_{3\times 3}$.
\end{problem}

It's not too difficult to prove---but we will omit the details---that we can generate $\Spin_{3\times 3}$ with the following subset of 9 spins:
\[
T=\{s_{11}, s_{12}, s_{23}, s_{36}, s_{56}, s_{45}, s_{47}, s_{78}, s_{89}\}.
\]
That is, every net action in $\Spin_{3\times 3}$ corresponds to a word consisting of the spins from $T$.  Try to take a moment to convince yourself that this is at least plausible.  

\begin{problem}
For each of the following spins, find a word consisting of spins from the set $T$ that yields the same net action.
\begin{enumerate}[label=\rm{(\alph*)}]
\item $s_{33}$
\item $s_{13}$
\item $s_{14}$
\end{enumerate}
\end{problem}

\begin{problem}
Taking for granted that $T$ is a generating set for $\Spin_{3\times 3}$, determine whether $T$ is a minimal generating set.
\end{problem}

\end{section}

%%----------------------------%%

\begin{section}{Binary Operations}

%%----------------------------%%

Before beginning our formal study of groups, we need have an understanding of binary operations. After learning to count as a child, you likely learned how to add, subtract, multiply, and divide with natural numbers.  As long as we avoid division by zero, these operations are examples of binary operations since we are combining two objects to obtain a single object.  More formally, we have the following definition.

\begin{definition}
A \textbf{binary operation} $*$ on a set $A$ is a function from $A\times A$ into $A$.  For each $(a,b)\in A\times A$, we denote the element $*(a,b)$ via $a*b$.  If the context is clear, we may abbreviate $a*b$ as $ab$.
\end{definition}

Don't misunderstand the use of $*$ in this context.  We are not implying that $*$ is the ordinary multiplication of real numbers that you are familiar with.  We use $*$ to represent a generic binary operation.  

Notice that since the codomain of a binary operation on a set $A$ is $A$, binary operations require that we yield an element of $A$ when combining two elements of $A$.  In this case, we say that $A$ is \textbf{closed} under $*$.  Binary operations have this closure property by definition.  Also, since binary operations are functions, any attempt to combine two elements from $A$ should result in a \emph{unique} element of $A$.  In this case, we say that $*$ is \textbf{well-defined}.  Moreover, since the domain of $*$ is $A\times A$, it must be the case that $*$ is defined for \emph{all} pairs of elements from $A$.

\begin{example}
Here are some examples of binary operations.
\begin{enumerate}[label=\rm{(\alph*)}]
\item The operations of $+$ (addition), $-$ (subtraction), and $\cdot$ (multiplication) are binary operations on the real numbers.  All three are also binary operations on the integers.  However, while $+$ and $\cdot$ are both binary operations on the set of natural numbers, $-$ is not a binary operation on the natural numbers since $1-2=-1$, which is not a natural number.
\item The operation of $\div$ (division) is not a binary operation on the set of real numbers because all elements of the form $(a,0)$ are not in the domain $\mathbb{R}\times \mathbb{R}$ since we cannot divide by 0.  Yet, $\div$ is a suitable binary operation on $\mathbb{R}\setminus \{0\}$.
\item Let $A$ be a nonempty set and let $F$ be the set of functions from $A$ to $A$.  Then $\circ$ (function composition) is a binary operation on $F$. We utilized this fact when exploring the game Spinpossible.
\item Let $M_{2\times 2}(\mathbb{R})$ be the set of $2\times 2$ matrices with real number entries.  Then matrix multiplication is a binary operation on $M_{2\times 2}(\mathbb{R})$.
\end{enumerate}
\end{example}

\begin{problem}
Let $M(\mathbb{R})$ be the set of matrices (of any size) with real number entries.  Is matrix addition a binary operation on $M(\mathbb{R})$?  How about matrix multiplication? What if you restrict to square matrices of a fixed size $n\times n$?
\end{problem}

\begin{problem}
Let $A$ be a set. Determine whether $\cup$ (union) and $\cap$ (intersection) are binary operations on $\mathcal{P}(A)$ (i.e., the power set of $A$).
\end{problem}

\begin{problem}
Consider the closed interval $[0,1]$ and define $*$ on $[0,1]$ via $a*b=\mathrm{min}\{a,b\}$ (i.e., take the minimum of $a$ and $b$).  Determine whether $*$ is a binary operation on $[0,1]$.
\end{problem}

\begin{problem}\label{prob:introducing_R4}
Consider a square puzzle piece that fits perfectly into a square hole.  Let $R_4$ be the set of net actions consisting of the rotations of the square by an appropriate amount so that it fits back into the hole. Assume we can tell the corners of the square apart from each other so that if the square has been rotated and put back in the hole we can notice the difference. Each net action is called a \textbf{symmetry} of the square. 
\begin{enumerate}[label=\rm{(\alph*)}]
\item Describe all of the distinct symmetries in $R_4$. How many distinct symmetries are in $R_4$?
\item Is composition of symmetries a binary operation on $R_4$?
\end{enumerate}
\end{problem}

The set $R_4$ is called the rotation group for the square. For $n\geq 3$, $R_n$ is the \textbf{rotation group} for the regular $n$-gon and consists of the rotational symmetries for a regular $n$-gon. As we shall see later, every $R_n$ really is a group under composition of symmetries.

\begin{problem}\label{prob:introducing_D3}
Consider a puzzle piece like the one in the previous problem, except this time, let's assume that the piece and the hole are an equilateral triangle.  Let $D_3$ be the full set of symmetries that allow the triangle to fit back in the hole.  In addition to rotations, we will also allow the triangle to be flipped over---called a reflection. 
\begin{enumerate}[label=\rm{(\alph*)}]
\item Describe all of the distinct symmetries in $D_3$. How many distinct symmetries are in $D_3$?
\item Is composition of symmetries a binary operation on $D_3$?
\end{enumerate}
\end{problem}

\begin{problem}\label{prob:introducing_D4}
Repeat the above problem, but do it for a square instead of a triangle.  The corresponding set is called $D_4$.
\end{problem}

The sets $D_3$ and $D_4$ are examples of dihedral groups. In general, for $n\geq 3$, $D_n$ consists of the symmetries (rotations and reflections) of a regular $n$-gon and is called the \textbf{dihedral group of order $2n$}. In this case, the word ``order" simply means the number of symmetries in the set. Do you see why $D_n$ consists of $2n$ actions? As expected, we will prove that every $D_n$ really is a group.

\begin{problem}\label{prob:introducing_S3}
Consider the set $S_3$ consisting of the net actions that permute the positions of three coins (without flipping them over) that are sitting side by side in a line.  Assume that you can tell the coins apart.
\begin{enumerate}[label=\rm{(\alph*)}]
\item Write down all distinct net actions in $S_3$ using verbal descriptions. Some of these will be tricky to describe. How many distinct net actions are in $S_3$?
\item Is composition of net actions a binary operation on $S_3$?
\end{enumerate}
\end{problem}

The set $S_3$ is an example of a symmetric group. In general, $S_n$ is the \textbf{symmetric group on $n$ objects} and consists of the net actions that rearrange the $n$ objects. Such rearrangements are called \textbf{permutations}. Later we will prove that each $S_n$ is a group under composition of permutations.

\begin{problem}
Explain why composition of spins is not a binary operation on the set of spins in $\Spin_{3\times 3}$.
\end{problem}

Some binary operations have additional properties.

\begin{definition}
Let $A$ be a nonempty set and let $*$ be a binary operation on $A$.
\begin{enumerate}[label=\rm{(\alph*)}]
\item We say that $*$ is \textbf{associative} if and only if $(a*b)*c=a*(b*c)$ for all $a,b,c\in A$.
\item We say that $*$ is \textbf{commutative} if and only if $a*b=b*a$ for all $a,b\in A$.
\end{enumerate}
\end{definition}

\begin{problem}
Provide an example of each of the following.
\begin{enumerate}[label=\rm{(\alph*)}]
\item A binary operation on a set that is commutative.
\item A binary operation on a set that is not commutative.
\end{enumerate}
\end{problem}

\begin{problem}
Provide an example of a set $A$ and a binary operation $*$ on $A$ such that $(a*b)^2\neq a^2*b^2$ for some $a,b\in A$.  Under what conditions will $(a*b)^2= a^2*b^2$ for all $a,b\in A$? \emph{Note:} The notation $x^2$ is shorthand for $x*x$.
\end{problem}

\begin{problem}
Define the binary operation $*$ on $\mathbb{R}$ via $a*b=1+ab$. In this case, $ab$ denotes the multiplication of the real numbers $a$ and $b$. Determine whether $*$ is associative on $\mathbb{R}$.
\end{problem}

\begin{theorem}\label{thm:function_comp_associative}
Let $A$ be a nonempty set and let $F$ be the set of functions from $A$ to $A$.  Then function composition is an associative binary operation on $F$.
\end{theorem}

When the set $A$ is finite, we can represent a binary operation on $A$ using a table in which the elements of the set are listed across the top and down the left side (in the same order).  The entry in the $i$th row and $j$th column of the table represents the output of combining the element that labels the $i$th row with the element that labels the $j$th column (order matters).

\begin{example}\label{example:table}
Consider the following table.
\begin{center}
\begin{tabu}{c|[2pt]c|c|c}
$*$ & $a$ & $b$ & $c$ \\ \tabucline[2pt]{-}
$a$ & $b$ & $c$ & $b$ \\
\hline $b$ & $a$ & $c$ & $b$  \\
\hline $c$ & $c$ & $b$ & $a$
\end{tabu}
\end{center}
This table represents a binary operation on the set $A=\{a,b,c\}$.  In this case, $a*b=c$ while $b*a=a$.  This shows that $*$ is not commutative.
\end{example}

\begin{problem}\label{prob:table}
Consider the following table that displays the binary operation $*$ on the set $\{x,y,z\}$.
\begin{center}
\begin{tabu}{c|[2pt]c|c|c}
$*$ & $x$ & $y$ & $z$ \\ \tabucline[2pt]{-}
$x$ & $x$ & $y$ & $z$ \\
\hline $y$ & $y$ & $x$ & $x$  \\
\hline $z$ & $y$ & $x$ & $x$
\end{tabu}
\end{center}
\begin{enumerate}[label=\rm{(\alph*)}]
\item Determine whether $*$ is commutative.
\item Determine whether $*$ is associative.
\end{enumerate}
\end{problem}

\begin{problem}
What property must the table for a binary operation have in order for the operation to be commutative?
\end{problem}

\end{section}

%%----------------------------%%

\begin{section}{Groups}

%%----------------------------%%

Without further ado, here is our official definition of a group.

\begin{definition}\label{def:group}
A \textbf{group} $(G,*)$ is a set $G$ together with a binary operation $*$ such that the following axioms hold.
\begin{enumerate}
\item[(0)] The set $G$ is closed under $*$.
\item[(1)] The operation $*$ is associative.
\item[(2)] There is an element $e\in G$ such that for all $g\in G$, $e*g=g*e=g$.  We call $e$ the \textbf{identity}.
\item[(3)] Corresponding to each $g\in G$, there is an element $g'\in G$ such that $g*g'=g'*g=e$.  In this case, $g'$ is called the \textbf{inverse} of $g$, which we shall denote as $g^{-1}$.
\end{enumerate}
The \textbf{order} of $G$, denoted $|G|$, is the cardinality of the set $G$. If $|G|$ is finite, then we say that $G$ has finite order. Otherwise, we say that $G$ has infinite order.
\end{definition}

In the definition of a group, the binary operation $*$ is not required to be commutative.  If $*$ is commutative, then we say that $G$ is \textbf{abelian}. Commutative groups are called abelian in honor of the Norwegian mathematician Niels Abel (1802--1829). A few additional comments are in order.
\begin{itemize}
\item Axiom 2 forces $G$ to be nonempty.
\item If $(G,*)$ is a group, then we say that \textbf{$G$ is a group under $*$}.
\item We refer to $a*b$ as the \textbf{product} of $a$ and $b$ even if $*$ is not actually multiplication. 
\item For simplicity, if $(G,*)$ is a group, we will often refer to $G$ as being the group and suppress any mention of $*$ whatsoever.  In particular, we will often abbreviate $a*b$ as $ab$. 
\end{itemize}

\begin{problem}
Explain why Axiom 0 is unnecessary.
\end{problem}

\begin{problem}
Verify that each of the following is a group under composition of actions and determine the order. Which of the groups are abelian?
\begin{enumerate}[label=\rm{(\alph*)}]
\item $\Spin_{3\times 3}$
\item $R_4$ (see Problem~\ref{prob:introducing_R4})
\item $D_3$ (see Problem~\ref{prob:introducing_D3})
\item $D_4$ (see Problem~\ref{prob:introducing_D4})
\item $S_3$ (see Problem~\ref{prob:introducing_S3})
\end{enumerate}
\end{problem}

\begin{problem}
Determine whether each of the following is a group.  If the pair is a group, determine the order, identify the identity, describe the inverses, and determine whether the group is abelian. If the pair is not a group, explain why.
\begin{enumerate}[label=\rm{(\alph*)}]
\item $(\mathbb{Z},+)$
\item $(\mathbb{N},+)$
\item $(\mathbb{Z},\cdot)$
\item $(\mathbb{R},+)$
\item $(\mathbb{R},\cdot)$
\item $(\mathbb{R}\setminus \{0\},\cdot)$
\item $(M_{2\times 2}(\mathbb{R}),+)$
\item $(M_{2\times 2}(\mathbb{R}),*)$, where $*$ is matrix multiplication.
\item $(\{a,b,c\},*)$, where $*$ is the operation determined by the table in Example~\ref{example:table}.
\item $(\{x,y,z\},*)$, where $*$ is the operation determined by the table in Problem~\ref{prob:table}.
\end{enumerate}
\end{problem}

Notice that in Axiom 2 of Definition~\ref{def:group}, we said \emph{the} identity and not \emph{an} identity.  Implicitly, this implies that the identity is unique.

\begin{theorem}\label{thm:unique_id}
If $G$ is a group, then there is a unique identity element in $G$.  That is, there is only one element $e\in G$ such that $ge=eg=g$ for all $g\in G$.
\end{theorem}

\begin{problem}
Provide an example of a group of order 1.  Can you find more than one such group?
\end{problem}

Any group of order 1 is called a \textbf{trivial group}. It follows immediately from the definition of a group that the element of a trivial group must be the identity.

The following theorem is crucial for proving many theorems about groups.

\begin{theorem}[Cancellation Law]
Let $G$ be a group and let $g,x,y\in G$.  Then $gx=gy$ if and only if $x=y$.  Similarly, $xg=yg$ if and only if $x=y$.\footnote{You only need to prove one of these statements as the proof of the other is similar.}
\end{theorem}

\begin{problem}
Show that $(\mathbb{R},\cdot)$ fails the Cancellation Law confirming the fact that it is not a group.
\end{problem}

\begin{corollary}
If $G$ is a group, then each $g\in G$ has a unique inverse.
\end{corollary}

\begin{theorem}\label{thm:unique_soln}
If $G$ is a group, then for all $g,h\in G$, the equations $gx=h$ and $yg=h$ have unique solutions for $x$ and $y$ in $G$.  
\end{theorem}

While proving the previous few theorems, hopefully one of the things you realized is that you can multiply both sides of a group equation by the same element but that you have to do it on the same side of each half.  That is, since a group may or may not be abelian, if we multiply one side of an equation on the left by a group element, then we must multiply the other side of the equation on the left by the same group element.

Despite the fact that a group may or may not be abelian, if one product is equal to the identity, then reversing the order yields the same result.

\begin{theorem}
If $G$ is a group and $g,h\in G$ such that $gh=e$, then $hg=e$.
\end{theorem}

The upshot of the previous theorem is if we have a ``left inverse" then we automatically have a ``right inverse" (and vice versa). The next theorem should not be surprising.

\begin{theorem}
If $G$ is a group, then $(g^{-1})^{-1}=g$ for all $g\in G$.
\end{theorem}

The next theorem is analogous to the ``socks and shoes theorem" for composition of functions.

\begin{theorem}
If $G$ is a group, then $(gh)^{-1}=h^{-1}g^{-1}$ for all $g,h\in G$.
\end{theorem}

\begin{definition}\label{def:exponents}
If $G$ is a group and $g\in G$, then for all $n\in \mathbb{N}$, we define:
\begin{enumerate}[label=\rm{(\alph*)}]
\item $g^n=\underbrace{gg\cdots g}_{n\text{ factors}}$
\item $g^{-n}=\underbrace{g^{-1}g^{-1}\cdots g^{-1}}_{n\text{ factors}}$
\item $g^0=e$
\end{enumerate}
\end{definition}

Note that if $G$ is a group under $+$, then we can reinterpret Definition~\ref{def:exponents} as:
\begin{enumerate}[label=\rm{(\alph*)}]
\item $ng=\underbrace{g+g+\cdots +g}_{n\text{ summands}}$
\item $-ng=\underbrace{-g+-g+\cdots +-g}_{n\text{ summands}}$
\item $0g=0$
\end{enumerate}

The good news is that the rules of exponents you are familiar with still hold for groups.

\begin{theorem}\label{thm:exponents}
If $G$ is a group and $g\in G$, then for all $n,m\in\mathbb{Z}$, we have the following:
\begin{enumerate}[label=\rm{(\alph*)}]
\item $g^n*g^m=g^{n+m}$,
\item $(g^n)^{-1}=g^{-n}$.
\end{enumerate}
\end{theorem}

\begin{problem}
Reinterpret Theorem~\ref{thm:exponents} if $G$ is a group under addition.
\end{problem}

\end{section}

%%----------------------------%%

\begin{section}{Generating Sets}

%%----------------------------%%

In this section, we explore the concept of a generating set for a group.

\begin{definition}
Let $G$ be a group and let $S$ be a subset of $G$. A finite product (under the operation of $G$) consisting of elements from $S$ or their inverses is called a \textbf{word} in $S$. That is, a word in $S$ is of the form
\[
s_{x_1}s_{x_2}\cdots s_{x_n},
\]
where each $s_{x_i}$ is either an element of $S$ or the inverse of an element of $S$. Each $s_{x_i}$ is called a \textbf{letter} and the set $S$ is called the \textbf{alphabet}. By convention, the identity of $G$ can be represented by the \textbf{empty word}, which is the word having no letters. The set of elements of $G$ that can be written as words in $S$ is denoted by $\langle S\rangle$ and is called the \textbf{group generated by $S$}.
\end{definition}

For example, if $a, b,$ and $c$ are elements of a group $G$, then $ab$, $c^{-1}acc$, and $ab^{-1}caa^{-1}bc^{-1}$ are words in the set $\{a,b,c\}$.  It is important to point out that two different words may be equal to the same element in $G$. We saw this happen when we studied Spinpossible in Section~\ref{sec:first_example}. For example, see Problems~\ref{prob:3_different_spins}--\ref{prob:braid_relation}.

\begin{theorem}
If $G$ is a group under $*$ and $S$ is a subset of $G$, then $\langle S\rangle$ is also a group under $*$.
\end{theorem}

\begin{definition}
If $G$ is a group and $S$ is a subset of $G$ such that $G=\langle S\rangle$, then $S$ is called a \textbf{generating set} of $G$. In other words, $S$ is a generating set of $G$ if every element of $G$ can be expressed as a word in $S$.  In this case, we say $S$ \textbf{generates} $G$.  A generating set $S$ for $G$ is a \textbf{minimal generating set} if removing any single element from $S$ results in a set that no longer generates the group.
\end{definition}

A generating set for a group is analogous to a spanning set for a vector space and a minimal generating set for a group is analogous to a basis for a vector space.  

If we know what the elements of $S$ actually are, then we will list them inside the angle brackets without the set braces.  For example, if $S=\{a,b,c\}$, then we will write $\langle a, b, c\rangle$ instead of $\langle \{a,b,c\}\rangle$. In the special case when the generating set $S$ consists of a single element, say $g$, we have
\[
G=\langle g\rangle =\{g^k\mid k\in\mathbb{Z}\}
\]
and say that $G$ as a \textbf{cyclic group}.  As we shall see, $\langle g\rangle$ may be finite or infinite. 

\begin{example}
In Section~\ref{sec:first_example}, we discovered that the set of spins is a non-minimal generating set for $\Spin_{3\times 3}$ while the set $T=\{s_{11}, s_{12}, s_{23}, s_{36}, s_{56}, s_{45}, s_{47}, s_{78}, s_{89}\}$ is a minimal generating set.
\end{example}

\begin{problem}
Consider the rotation group $R_4$ that we introduced in Problem~\ref{prob:introducing_R4}. Let $r$ be the element of $R_4$ that rotates the square by $90^\circ$ clockwise. 
\begin{enumerate}[label=\rm{(\alph*)}]
\item Describe the action of $r^{-1}$ on the square and express $r^{-1}$ as a word using $r$ only.
\item Prove that $R_4=\langle r\rangle$ by writing every element of $R_4$ as a word using $r$ only.
\item Is $\{r\}$ a minimal generating set for $R_4$?
\item Is $R_4$ a cyclic group?
\end{enumerate}
\end{problem}

\begin{problem}\label{prob:revisit_D3}
Consider the dihedral group $D_3$ introduced in Problem~\ref{prob:introducing_D3}. To give us a common starting point, let's assume the triangle and hole are positioned so that one of the tips of the triangle is pointed up. Let $r$ be rotation by $120^\circ$ in the clockwise direction and let $s$ be the reflection in $D_3$ that fixes the top of the triangle.
\begin{enumerate}[label=\rm{(\alph*)}]
\item Describe the action of $r^{-1}$ on the triangle and express $r^{-1}$ as a word using $r$ only.
\item Describe the action of $s^{-1}$ on the triangle and express $s^{-1}$ as a word using $s$ only.
\item Prove that $D_3=\langle r,s\rangle$ by writing every element of $D_3$ as a word in $r$ or $s$.
\item Is $\{r,s\}$ a minimal generating set for $D_3$?
\item Explain why there is no single generating set for $D_3$ consisting of a single element. This proves that $D_3$ is not cyclic.
\end{enumerate}
\end{problem}

It is important to point out that the fact that $\{r,s\}$ is a minimal generating set for $D_3$ does not imply that $D_3$ is not a cyclic group. There are examples of cyclic groups that have minimal generating sets consisting of more than one element (see Problem~\ref{prob:introducing_R6}).

\begin{problem}\label{prob:alternate_D3}
Let's consider the group $D_3$ again. Let $s$ be the same reflection as in Problem~\ref{prob:revisit_D3} and let $s'$ be the reflection in $D_3$ that fixes the bottom right corner of the triangle.
\begin{enumerate}[label=\rm{(\alph*)}]
\item Express $r$ as a word in $s$ and $s'$.
\item Use part (a) together with Problem~\ref{prob:revisit_D3} to prove that $\langle s,s'\rangle=D_3$.
\end{enumerate}
\end{problem}

\begin{problem}\label{prob:revisiting_D4}
Consider the dihedral group $D_4$ introduced in Problem~\ref{prob:introducing_D4}. Let $r$ be clockwise rotation by $90^\circ$ and let $s$ be the reflection over the vertical midline of the square.
\begin{enumerate}[label=\rm{(\alph*)}]
\item Describe the action of $r^{-1}$ on the square and express $r^{-1}$ as a word using $r$ only.
\item Describe the action of $s^{-1}$ on the square and express $s^{-1}$ as a word using $s$ only.
\item Prove that $\{r,s\}$ is generating set for $D_4$.
\item Is $\{r,s\}$ a minimal generating set for $D_4$?
\item Find a different generating set for $D_4$.
\item Is $D_4$ a cyclic group?
\end{enumerate} 
\end{problem}

\begin{problem}\label{prob:revisiting_S3}
Consider the symmetric group $S_3$ that was introduced in Problem~\ref{prob:introducing_S3}. Let $s_1$ be the action that swaps the positions of the first and second coins and let $s_2$ be the action that swaps the positions of the second and third coins. Prove that $S_3=\langle s_1, s_2\rangle$.
\end{problem}

\begin{problem}\label{prob:introducing_S2}
Consider the symmetric group $S_2$ that consists of the permutations of two coins that are sitting side by side.
\begin{enumerate}[label=\rm{(\alph*)}]
\item What is the order of $S_2$?
\item Find a generating set for $S_2$.
\item Is $S_2$ cyclic?
\item Is $S_2$ abelian?
\end{enumerate}
\end{problem}

\begin{problem}
Find a minimal generating set for $(\mathbb{Z},+)$.  Is $\mathbb{Z}$ a cyclic group under addition?
\end{problem}

\end{section}

%%----------------------------%%

\begin{section}{Group Tables}

%%----------------------------%%

Maybe I should do a quick intro to group tables and Cayley diagrams (introducing a couple new groups) and then end the chapter.  The next chapter could be subgroups and isomorphisms (introducing more groups).

Maybe half of this should come before generating sets.

Introduce at least one new group, say $V_4$.

\end{section}

%%----------------------------%%

\begin{section}{Cayley Diagrams}

%%----------------------------%%

Blah

\end{section}

%%----------------------------%%

\begin{section}{Isomorphisms}

%%----------------------------%%

Blah

\end{section}

%%----------------------------%%

\begin{section}{Subgroups}

%%----------------------------%%

Maybe this should come before group tables???

\end{section}



%%----------------------------%%

\begin{section}{Scraps}

%%----------------------------%%

%right spot for this?
\begin{problem}
Can you describe a group that has exactly $n$ elements for any natural number $n$?
\end{problem}

\begin{theorem}%similar theorem above
If $G$ is a group under $*$ and $S$ is a subset of $G$, then $\langle S\rangle$ is also a group under $*$. Smallest...
\end{theorem}

\begin{theorem}
If $G$ is a group, then $\langle G\rangle =G$.
\end{theorem}

\begin{problem}
If $G$ is a group, then what is $\langle \emptyset\rangle$?
\end{problem}

\end{section}


\chapter{Subgroups and Isomorphisms}
\label{chapter:subgroups_isomorphisms}
\thispagestyle{empty}

For the next two sections, it would be useful to have all of the Cayley diagrams we've encountered in one place for reference. So, before continuing, gather up the following Cayley diagrams:
\begin{itemize}
\item $\Spin_{1\times 2}$. There are 3 of these.  I drew one for you in Section~\ref{sec:cayley_diagrams} and you discovered two more in Problem~\ref{cayley:altSpin1x2}.
\item $S_2$.  See Problem~\ref{prob:make_Cayley_diagrams}\ref{cayley:S2}.
\item $R_4$.  See Problem~\ref{prob:make_Cayley_diagrams}\ref{cayley:R4}.
\item $V_4$.  See Problem~\ref{prob:make_Cayley_diagrams}\ref{cayley:V4}.
\item $D_3$.  There are two of these.  See Problems~\ref{prob:make_Cayley_diagrams}\ref{cayley:D3} and \ref{prob:make_Cayley_diagrams}\ref{cayley:altD3}.
\item $S_3$.  See Problem~\ref{prob:make_Cayley_diagrams}\ref{cayley:S3}.
\item $D_4$.  See Problem~\ref{prob:make_Cayley_diagrams}\ref{cayley:D4}.
\end{itemize}

%%----------------------------%%

\begin{section}{Subgroups}

%%----------------------------%%

\begin{problem}
Recall the definition of ``subset."  What do you think ``subgroup" means?  Try to come up with a potential definition.  Try not to read any further before doing this.
\end{problem}

\begin{problem}\label{prob:R4_in_D4}
Examine your Cayley diagrams for $D_4$ (with generating set $\{r,s\}$) and $R_4$ (with generating set $\{r\}$) and make some observations.  How are they similar and how are they different?  Can you reconcile the similarities and differences by thinking about the actions of each group?
\end{problem}

Hopefully, one of the things you noticed in the previous problem is that we can ``see" $R_4$ inside of $D_4$.  You may have used different colors in each case and maybe even labeled the vertices with different words, but the overall structure of $R_4$ is there nonetheless.

\begin{problem}\label{prob:R4_subgroup_D_4}
If you ignore the labels on the vertices and just pay attention to the configuration of arrows, it appears that there are two copies of the Cayley diagram for $R_4$ in the Cayley diagram for $D_4$.  Isolate these two copies by ignoring the edges that correspond to the generator $s$.  Now, paying close attention to the words that label the vertices from the original Cayley diagram for $D_4$, are either of these groups in their own right?
\end{problem}

Recall that the identity must be one of the elements included in a group.  If this didn't occur to you when doing the previous problem, you might want to go back and rethink your answer.  Just like in the previous problem, we can often ``see" smaller groups living inside larger groups.  These smaller groups are called \textbf{subgroups}.

\begin{definition}
Let $G$ be a group and let $H$ be a subset of $G$.  Then $H$ is a \textbf{subgroup} of $G$, written $H\leq G$, provided that $H$ is a group in its own right under the binary operation inherited from $G$.
\end{definition}

The phrase ``under the binary operation inherited from $G$" means that to combine two elements in $H$, we should treat the elements as if they were in $G$ and perform the binary operation of $G$.

In light of Problem~\ref{prob:R4_subgroup_D_4}, we would write $R_4\leq D_4$.  The second sub-diagram of the Cayley diagram for $D_4$ (using $\{r,s\}$ as the generating set) that resembles $R_4$ cannot be a subgroup because it does not contain the identity.  However, since it looks a lot like $R_4$, we call it a \textbf{clone} of $R_4$. For convenience, we also say that a subgroup is a clone of itself.

\begin{problem}\label{prob:informal_subgroup_criterion}
Let $G$ be a group and let $H\subseteq G$. If we wanted to determine whether $H$ is a subgroup of $G$, can we skip checking any of the axioms? Which axioms must we verify?
\end{problem}

Let's make the observations of the previous problem a bit more formal.

\begin{theorem}[Two Step Subgroup Test]\label{thm:subgroup_criterion}
Suppose $G$ is a group and $H$ is a nonempty subset of $G$.  Then $H\leq G$ if and only if (i) for all $h\in H$, $h^{-1} \in H$, as well, and (ii) $H$ is closed under the binary operation of $G$.
\end{theorem}

Notice that one of the hypotheses of Theorem~\ref{thm:subgroup_criterion} is that $H$ be nonempty.  This means that if we want to prove that a certain subset $H$ is a subgroup of a group $G$, then one of the things we must do is verify that $H$ is in fact nonempty. In light of this, the ``Two Step Subgroup Test" should probably be called the ``Three Step Subgroup Test".

As Theorems~\ref{thm:trivial_subgroup} and \ref{thm:improper_subgroup} will illustrate, there are a couple of subgroups that every group contains.

\begin{theorem}\label{thm:trivial_subgroup}
If $G$ is a group, then $\{e\}\leq G$.
\end{theorem}

The subgroup $\{e\}$ is referred to as the \textbf{trivial subgroup}.  All other subgroups are called \textbf{nontrivial}.

\begin{problem}
Let $G$ be a group. What does the Cayley diagram for the subgroup $\{e\}$ look like? What are you using as your generating set?
\end{problem}

Earlier, we referred to subgroups as being ``smaller."  However, our definition does not imply that this has to be the case.

\begin{theorem}\label{thm:improper_subgroup}
If $G$ is a group, then $G\leq G$.
\end{theorem}

We refer to subgroups that are not equal to the whole group as \textbf{proper subgroups}. If $H$ is a proper subgroup, then we may write $H<G$.

Recall Theorem~\ref{thm:subgroup_generated_by_S} that states that if $G$ is a group under $*$ and $S$ is a subset of $G$, then $\langle S\rangle$ is also a group under $*$.  Let's take this a step further.

\begin{theorem}\label{thm:smallest_subgroup_containing_S}
If $G$ is a group and $S\subseteq G$, then $\langle S\rangle \leq G$.  In particular, $\langle S\rangle$ is the smallest subgroup of $G$ containing $S$.
\end{theorem}

The subgroup $\langle S\rangle$ is called the \textbf{subgroup generated by $S$}.  In the special case when $S$ equals a single element, say $S=\{a\}$, then
\[
\langle a\rangle =\{a^n\mid n\in\mathbb{Z}\},
\]
which is called the (\textbf{cyclic}) \textbf{subgroup generated by $a$}. Every subgroup can be written in the ``generated by" form.  That is, if $H$ is a subgroup of a group $G$, then there always exists a subset $S$ of $G$ such that $\langle S\rangle=H$.  In particular, $\langle G\rangle=G$.

\begin{problem}
Consider $\Spin_{1\times 2}$ with generating set $\{s_{11}, s_{22},s_{12}\}$.  
\begin{enumerate}[label=\rm{(\alph*)}]
\item Find the Cayley diagram for the subgroup $\langle s_{11}\rangle$ inside the Cayley diagram for $\Spin_{1\times 2}$.  Identify all of the clones of $\langle s_{11}\rangle$ inside $\Spin_{1\times 2}$.
\item Find the Cayley diagram for the subgroup $\langle s_{11}, s_{22}\rangle$ inside the Cayley diagram of $\Spin_{1\times 2}$.  Identify the clones of $\langle s_{11}, s_{22}\rangle$ inside $\Spin_{1\times 2}$.
\end{enumerate}
\end{problem}

One of the benefits of Cayley diagrams is that they are useful for visualizing subgroups.  However, recall that if we change our set of generators, we might get a very different looking Cayley diagram.  The upshot of this is that we may be able to see a subgroup in one Cayley diagram for a given group, but not be able to see it in the Cayley diagram arising from a different generating set.

\begin{problem}
We currently have two different Cayley diagrams for $D_3$ (see Problems~\ref{prob:introducing_D3} and \ref{prob:alternate_D3}).  
\begin{enumerate}[label=\rm{(\alph*)}]
\item Can you find the Cayley diagram for the trivial subgroup $\langle e\rangle$ in either Cayley diagram for $D_3$?  Identify all of the clones of $\langle e\rangle$ in both Cayley diagrams for $D_3$.
\item Can you find the Cayley diagram for the subgroup $\langle r\rangle =R_3$ in either Cayley diagram for $D_3$?  If possible, identify all of the clones of $R_3$ in the Cayley diagrams for $D_3$.
\item Can you find the Cayley diagrams for $\langle s\rangle$ and $\langle s'\rangle$ in either Cayley diagram for $D_3$?  If possible, identify all of the clones of $\langle s\rangle$ and $\langle s'\rangle$ in the Cayley diagrams for $D_3$.
\end{enumerate}
\end{problem}

\begin{problem}\label{prob:subgroups_D4}
Consider $D_4$.  Let $h$ be the reflection of the square over the horizontal midline and let $v$ be the reflection over the vertical midline.  Which of the following are subgroups of $D_4$?  In each case, justify your answer.  If a subset is a subgroup, try to find a minimal generating set.  Also, determine whether you can see the subgroups in our Cayley diagram for $D_4$ with generating set $\{r,s\}$.
\begin{enumerate}[label=\rm{(\alph*)}]
\item $\{e, r^2\}$
\item $\{e,h\}$
\item $\{e, h, v\}$
\item\label{V4} $\{e, h, v, r^2\}$
\end{enumerate}
\end{problem}

Perhaps you recognized the set in part~\ref{V4} of the previous problem as being the Klein four-group $V_4$. It follows that $V_4\leq D_4$.














%Lots of groups have been given formal names (e.g., $D_4$, $R_4$, etc.).  However, not every group or subgroup has a name.  In this case, it's useful to have notation to refer to specific subgroups.

%\begin{definition}\label{def:subgroup_gen_by}
%Let $G$ be a group of actions and let $g_1,\ldots, g_n$ be distinct actions from $G$.  We define $\langle g_1,\ldots, g_n\rangle$ to be the smallest subgroup containing $g_1,\ldots, g_n$.  In this case, we call $\langle g_1,\ldots, g_n\rangle$ the \textbf{subgroup generated by} $g_1,\ldots, g_n$.
%\end{definition}
%
%For example, consider $r, s, s'\in D_3$ as defined in Exercises~\ref{prob:introducing_D3} and \ref{prob:alternate_D3}.  Then $D_3\langle r,s\rangle=\langle s, s'\rangle$.  Recall that $R_4$ is the subgroup of $D_4$ consisting of rotational symmetries of the square. In this case, $R_4=\langle r\rangle$.  Similarly, the group of rotations of an equilateral triangle is called $R_3$.  Then using the $r$ from $D_3$, we have $R_3=\langle r\rangle$, which is a subgroup of $D_3$.
%
%Note that in Definition~\ref{def:subgroup_gen_by}, we used a finite number of generators.  There's no reason we have to do this.  That is, we can consider groups/subgroups generated by infinitely many elements.

%\begin{problem}
%Suppose $\{g_1,\ldots,g_n\}$ is a generating set for a group $G$.
%\begin{enumerate}[label=\rm{(\alph*)}]
%\item Explain why $\{g^{-1}_1,\ldots,g^{-1}_n\}$ is also a generating set for $G$.
%\item How does the Cayley diagram for $G$ with generating set $\{g_1,\ldots,g_n\}$ compare to the Cayley diagram with generating set $\{g^{-1}_1,\ldots,g^{-1}_n\}$?
%\end{enumerate}
%\end{problem}

%\begin{problem}
%Consider $\Spin_{1\times 2}$.  
%\begin{enumerate}[label=\rm{(\alph*)}]
%\item Can you find the Cayley diagram for $\langle t_1\rangle$ inside the Cayley diagram for $\Spin_{1\times 2}$ that we previously constructed?
%\item Write down all the actions of the subgroup $\langle t_1, t_2\rangle$ by writing them as words in $t_1$ and $t_2$.  Can you find the Cayley diagram for $\langle t_1, t_2\rangle$ as a subgroup of $\Spin_{1\times 2}$?  Can you find a clone for $\langle t_1, t_2\rangle$ that is not the subgroup itself?
%\end{enumerate}
%\end{problem}
%
%One of the benefits of Cayley diagrams is that they are useful for visualizing subgroups.  However, recall that if we change our set of generators, we might get a very different looking Cayley diagram.  The upshot of this is that we may be able to see a subgroup in one Cayley diagram for a given group, but not be able to see it in the Cayley diagram arising from a different generating set.
%
%\begin{problem}
%We currently have two different Cayley diagrams for $D_3$ (see Exercises \ref{prob:introducing_D3} and \ref{prob:alternate_D3}).  
%\begin{enumerate}[label=\rm{(\alph*)}]
%\item Can you find the Cayley diagram for $\langle e\rangle$ as a subgroup of $D_3$?  Can you see it in both Cayley diagrams for $D_3$?  Can you find all the clones?
%\item Can you find the Cayley diagram for $\langle r\rangle =R_3$ as a subgroup of $D_3$?  Can you see it in both Cayley diagrams?  Can you find all the clones?
%\item Find the Cayley diagrams for $\langle s\rangle$ and $\langle s'\rangle$ as subgroups of $D_3$.  Can you see them in both Cayley diagrams for $D_3$?  Can you find all the clones?
%\end{enumerate}
%\end{problem}
%
%\begin{problem}\label{prob:subgroups_D4}
%Consider $D_4$.  Let $h$ be the action that reflects (i.e., flips over) the square over the horizontal midline and let $v$ be the action that reflects the square over the vertical midline.  Also, recall that $r^2$ is shorthand for the action $rr$ that does $r$ twice in a row.  Which of the following are subgroups of $D_4$?  In each case, justify your answer.  If a subset is a subgroup, try to find a minimal set of generators.  Also, determine whether you can see the subgroups in our Cayley diagram for $D_4$.
%\begin{enumerate}[label=\rm{(\alph*)}]
%\item $\{e, r^2\}$
%\item $\{e,h\}$
%\item $\{e, h, v\}$
%\item\label{V4} $\{e, h, v, r^2\}$
%\end{enumerate}
%\end{problem}
%
%The subgroup in Problem~\ref{prob:subgroups_D4}\ref{V4} is often referred to as the \textbf{Klein four-group} and is denoted by $V_4$.

%\begin{problem}\label{prob:V4}
%Draw the Cayley diagram for $V_4$ using $\{v,h\}$ as the generating set.
%\end{problem}

Let's introduce a group we haven't seen yet.  Define the \textbf{quaternion group} to be the group $Q_8=\{1,-1,i,-i,j,-j,k,-k\}$ having the Cayley diagram with generating set $\{i, j, -1\}$ given in Figure~\ref{fig:Q8}.  In this case, 1 is the identity of the group.

\tikzstyle{vert} = [circle, draw, fill=grey,inner sep=0pt, minimum size=6.5mm]
\tikzstyle{b} = [draw,very thick,blue,-stealth]
\tikzstyle{r} = [draw, very thick, red,-stealth]
\tikzstyle{g} = [draw, very thick, green, stealth-stealth]

\begin{figure}[!ht]
\centering
\begin{tikzpicture}[scale=1.5,auto]
\node (1) at (135:2) [vert] {\scriptsize $1$};
\node (i) at (45:2) [vert] {\scriptsize $i$};
\node (k) at (-45:2) [vert] {\scriptsize $k$};
\node (j) at (-135:2) [vert] {\scriptsize $j$};
\node (-1) at (135:1) [vert] {\scriptsize $-1$};
\node (-i) at (45:1) [vert] {\scriptsize $-i$};
\node (-k) at (-45:1) [vert] {\scriptsize $-k$};
\node (-j) at (-135:1) [vert] {\scriptsize $-j$};

\path[b] (1) to (i);
\path[b] (i) to (-1);
\path[b] (-1) to (-i);
\path[b] (-i) to (1);

\path[b] (-j) to (-k);
\path[b] (-k) to (j);
\path[b] (j) to (k);
\path[b] (k) to (-j);

\path[r] (-k) to (-i);
\path[r] (-i) to (k);
\path[r] (k) to (i);
\path[r] (i) to (-k);

\path[r] (1) to (j);
\path[r] (j) to (-1);
\path[r] (-1) to (-j);
\path[r] (-j) to (1);

\path[g] (1) to (-1);
\path[g] (j) to (-j);
\path[g] (i) to (-i);
\path[g] (k) to (-k);

\end{tikzpicture}
\caption{Cayley diagram for $Q_8$ with generating set $\{-1, i, j\}$.}\label{fig:Q8}
\end{figure}

Notice that I didn't mention what the actions actually do.  For now, let's not worry about that.  The relationship between the arrows and vertices tells us everything we need to know.  Also, let's take it for granted that $Q_8$ actually is a group.

\begin{problem}
Consider the Cayley diagram for $Q_8$ given in Figure~\ref{fig:Q8}.
\begin{enumerate}[label=\rm{(\alph*)}]
\item Which arrows correspond to which generators in our Cayley diagram for $Q_8$?
\item What is $i^2$ equal to?  That is, what element of $\{1,-1,i,-i,j,-j,k,-k\}$ is $i^2$ equal to?  How about $i^3$, $i^4$, and $i^5$?
\item What are $j^2$, $j^3$, $j^4$, and $j^5$ equal to?
\item What is $(-1)^2$ equal to?
\item What is $ij$ equal to?  How about $ji$?
\item Can you determine what $k^2$ and $ik$ are equal to?
\item Can you identify a generating set consisting of only two elements?  Can you find more than one?
\item What subgroups of $Q_8$ can you see in the Cayley diagram in Figure~\ref{fig:Q8}?
\item Find a subgroup of $Q_8$ that you cannot see in the Cayley diagram.
\end{enumerate}
\end{problem}

\begin{problem}
Consider $(\mathbb{R}^3,+)$, where $\mathbb{R}^3$ is the set of all 3-entry row vectors with real number entries (e.g., $(a,b,c)$ where $a,b,c\in\mathbb{R}$) and $+$ is ordinary vector addition.  It turns out that $(\mathbb{R}^3,+)$ is an abelian group with identity $(0,0,0)$.  
\begin{enumerate}[label=\rm{(\alph*)}]
\item Let $H$ be the subset of $\mathbb{R}^3$ consisting of vectors with first coordinate 0.  Is $H$ a subgroup of $\mathbb{R}^3$?  Prove your answer.
\item Let $K$ be the subset of $\mathbb{R}^3$ consisting of vectors whose entries sum to 0.  Is $K$ a subgroup of $\mathbb{R}^3$?  Prove your answer.
\item Construct a subset of $\mathbb{R}^3$ (different from $H$ and $K$) that is \emph{not} a subgroup of $\mathbb{R}^3$.
\end{enumerate}
\end{problem}

\begin{problem}\label{prob:nZ}
Consider the group $(\mathbb{Z},+)$ (under ordinary addition).
\begin{enumerate}[label=\rm{(\alph*)}]
\item Show that the even integers, written $2\mathbb{Z}:=\{2k\mid k\in\mathbb{Z}\}$, form a subgroup of $\mathbb{Z}$.
\item Show that the odd integers are not a subgroup of $\mathbb{Z}$.
\item Show that all subsets of the form $n\mathbb{Z}:=\{nk\mid k\in\mathbb{Z}\}$ for $n\in\mathbb{Z}$ are subgroups of $\mathbb{Z}$.
\item\label{prob:nZothers} Are there any other subgroups besides the ones listed in part (c)?  Explain your answer.
\item For $n\in \mathbb{Z}$, write the subgroup $n\mathbb{Z}$ in the ``generated by" notation.  That is, find a set $S$ such that $\langle S\rangle =n\mathbb{Z}$.  Can you find more than one way to do it?
\end{enumerate}
\end{problem}

\begin{problem}
Consider the group of symmetries of a regular octagon.  This group is denoted by $D_8$, where the operation is composition of actions.  The group $D_8$ consists of 16 elements (8 rotations and 8 reflections).  Let $H$ be the subset consisting of the following clockwise rotations: $0^\circ$, $90^\circ$, $180^\circ$, and $270^\circ$.  Determine whether $H$ is a subgroup of $D_8$ and justify your answer.
\end{problem}

\begin{problem}
Consider the groups $(\mathbb{R},+)$ and $(\mathbb{R}\setminus\{0\},\cdot)$.  Explain why $\mathbb{R}\setminus\{0\}$ is not a subgroup of $\mathbb{R}$ despite the fact that $\mathbb{R}\setminus\{0\}\subseteq\mathbb{R}$ and both are groups (under the respective binary operations).
\end{problem}

\begin{theorem}
If $G$ is an abelian group such that $H\leq G$, then $H$ is an abelian subgroup.
\end{theorem}

\begin{problem}
Is the converse of the previous theorem true?  If so, prove it.  Otherwise, provide a counterexample.
\end{problem}

As we've seen, some groups are abelian and some are not.  If $G$ is a group, then we define the \textbf{center} of $G$ to be
\[
Z(G):=\{z\in G\mid zg=gz\text{ for all } g\in G\}.
\]
Notice that if $G$ is abelian, then $Z(G)=G$.  However, if $G$ is not abelian, then $Z(G)$ will be a proper subset of $G$.  In some sense, the center of a group is a measure of how close $G$ is to being abelian.

\begin{theorem}
If $G$ is a group, then $Z(G)$ is an abelian subgroup of $G$.
\end{theorem}

\begin{problem}
Find the center of each of the following groups.
\begin{enumerate}[label=\rm{(\alph*)}]
\item $S_2$
\item $V_4$
\item $S_3$
\item $D_3$
\item $D_4$
\item $R_4$
\item $R_6$
\item $\Spin_{1\times 2}$
\item $Q_8$
\item $(\mathbb{Z},+)$
\item $(\mathbb{R}\setminus\{0\},\cdot)$
\end{enumerate}
\end{problem}

\end{section}

%%----------------------------%%

\begin{section}{Subgroup Lattices}

%%----------------------------%%

Coming soon.

%This needs to go in the proper spot.  Also, add something about $\langle H\cup K\rangle$ being the smallest subgroup containing $H$ and $K$.  Draw a picture of the diamond.

%\begin{theorem}
%If $G$ is a group such that $H,K\leq G$, then $H\cap K\leq G$. Moreover, $H\cap K$ is the largest subgroup contained in both $H$ and $K$.
%\end{theorem}
%
%\begin{problem}
%Can we replace intersection with union in the theorem above?  If so, prove the corresponding theorem.  If not, then provide a specific counterexample.
%\end{problem}
%
%Let's explore a couple of examples.  First, consider the group $R_4$ (where the operation is composition of actions).  What are the subgroups of $R_4$?  Theorems~\ref{thm:trivial_subgroup} and \ref{thm:improper_subgroup} tell us that $\{e\}$ and $R_4$ itself are subgroups of $R_4$.  Are there any others?  Theorem~\ref{thm:subgroup_criterion} tells us that if we want to find other subgroups of $R_4$, we need to find nonempty subsets of $R_4$ that are closed and contain all the necessary inverses.  However, the previous paragraph indicates that we can find all of the subgroups of $R_4$ by forming the subgroups generated by various combinations of elements from $R_4$.  We can certainly be more efficient, but below we list all of the possible subgroups we can generate using subsets of $R_4$.  We are assuming that $r$ is rotation by $90^{\circ}$ clockwise.  As you scan the list, you should take a moment to convince yourself that the list is accurate.
%\begin{multicols}{2}
%\begin{itemize}
%\item[] $\langle e \rangle = \{e\}$
%\item[] $\langle r \rangle  = \{e,r,r^2,r^3\}$
%\item[] $\langle r^2 \rangle  = \{e,r^2\}$
%\item[] $\langle r^3 \rangle  = \{e,r^3,r^2,r\}$
%\item[] $\langle e,r \rangle  = \{e,r,r^2,r^3\}$
%\item[] $\langle e,r^2 \rangle  = \{e,r^2\}$
%\item[] $\langle e,r^3 \rangle  = \{e,r^3,r^2,r\}$
%\item[] $\langle r,r^2 \rangle  = \{e,r,r^2,r^3\}$
%\item[] $\langle r,r^3 \rangle  = \{e,r,r^2,r^3\}$
%\item[] $\langle r^2,r^3 \rangle  = \{e,r,r^2,r^3\}$
%\item[] $\langle e,r,r^2 \rangle  = \{e,r,r^2,r^3\}$
%\item[] $\langle e,r,r^3 \rangle  = \{e,r,r^2,r^3\}$
%\item[] $\langle e,r^2,r^3 \rangle  = \{e,r,r^2,r^3\}$
%\item[] $\langle r,r^2,r^3 \rangle  = \{e,r,r^2,r^3\}$
%\item[] $\langle e,r,r^2,r^3 \rangle = \{e,r,r^2,r^3\}$
%\end{itemize}
%\end{multicols}
%Let's make a few observations.  Scanning the list, we see only three distinct subgroups: $\{e\}, \{e,r^2\},\{e,r,r^2,r^3\}$.  Our exhaustive search guarantees that these are the only subgroups of $R_4$.  It is also worth pointing out that if a subset contains either $r$ or $r^3$, then that subset generates all of $R_4$.  The reason for this is that $r$ and $r^3$ are each generators for $R_4$, respectively.  Also, observe that if we increase the size of the subset using an element that was already contained in the subgroup generated by the smaller set, then we don't get anything new.  For example, consider $\langle r^2\rangle=\{e,r^2\}$.  Since $e\in\langle r^2\rangle$, we don't get anything new by including $e$ in our generating set.  We can state this as a general fact.
%
%\begin{theorem}
%Let $(G,*)$ be a group and let $g_1,g_2,\ldots,g_n\in G$.  If $x\in\langle g_1,g_2,\ldots,g_n\rangle$, then $\langle g_1,g_2,\ldots,g_n\rangle = \langle g_1,g_2,\ldots,g_n,x\rangle$.
%\end{theorem}
%
%It is important to point out that in the theorem above, we are not saying that $\{g_1,g_2,\ldots,g_n\}$ is a generating set for $G$---although this may be the case.  Instead, are simply making a statement about the subgroup $\langle g_1,g_2,\ldots,g_n\rangle$, whatever it may be.
%
%Let's return to our example involving $R_4$.  We have three subgroups, namely the two trivial subgroups $\{e\}$ and $R_4$ itself, together with one nontrivial subgroup $\{e,r^2\}$.  Notice that $\{e\}$ is also a subgroup of $\{e,r^2\}$.  We can capture the overall relationship between the subgroups using a \textbf{subgroup lattice}, which we depict in Figure~\ref{fig:latticeR4} case of $R_4$.
%
%\tikzstyle{b3} = [draw,very thick,black]
%
%\begin{figure}[!ht]
%\centering
%\begin{tikzpicture}[scale=1.5,auto]
%\node (a) at (0,0) {$\langle e\rangle=\{e\}$};
%\node (b) at (0,2) {$\langle r^2\rangle=\{e,r^2\}$};
%\node (c) at (0,4) {$\langle r\rangle=R_4$};
%
%\path[b3] (a) to (b);
%\path[b3] (b) to (c);
%\end{tikzpicture}
%\caption{Subgroup lattice for $R_4$.}
%\label{fig:latticeR4}
%\end{figure}
%
%In general, subgroups of smaller order are towards the bottom of the lattice while larger subgroups are towards the top.  Moreover, an edge between two subgroups means that the smaller set is a subgroup of the larger set.
%
%Let's see what we can do with $V_4=\{e,v,h,vh\}$.  Using an exhaustive search, we find that there are five subgroups:
%\begin{itemize}
%\item[] $\langle e \rangle = \{e\}$
%\item[] $\langle h \rangle  = \{e,h\}$
%\item[] $\langle v \rangle  = \{e,v\}$
%\item[] $\langle vh \rangle  = \{e,vh\}$
%\item[] $\langle v,h \rangle = \langle v,vh\rangle = \langle h, vh\rangle= \{e,v,h,vh\}=V_4$
%\end{itemize}
%For each subgroup above, we've used minimal generating sets to determine the group.  (Note that minimal generating sets are generating sets where we cannot remove any elements and still obtain the same group.  Two minimal generating sets for the same group do not have to have the same number of generators.)  In this case, we get the subgroup lattice in Figure~\ref{fig:latticeV4}.
%
%\begin{figure}[!ht]
%\centering
%\begin{tikzpicture}[scale=1.5,auto]
%\node (a) at (0,0) {$\langle e\rangle=\{e\}$};
%\node (b) at (0,2) {$\langle h\rangle=\{e,h\}$};
%\node (c) at (-2,2) {$\langle v\rangle=\{e,v\}$};
%\node (d) at (2,2) {$\langle vh\rangle=\{e,vh\}$};
%\node (e) at (0,4) {$\langle v,h\rangle=V_4$};
%
%\path[b3] (a) to (b);
%\path[b3] (a) to (c);
%\path[b3] (a) to (d);
%\path[b3] (b) to (e);
%\path[b3] (c) to (e);
%\path[b3] (d) to (e);
%\end{tikzpicture}
%\caption{Subgroup lattice for $V_4$.}
%\label{fig:latticeV4}
%\end{figure}
%
%Notice that there are no edges among $\langle v\rangle, \langle h\rangle$, and $\langle vh\rangle$.  The reason for this is that none of these groups are subgroups of each other.  We already know that $R_4$ and $V_4$ are not isomorphic, but this becomes even more apparent if you compare their subgroup lattices.
%
%In the next few exercises, you are asked to create subgroup lattices.  As you do this, try to minimize the amount of work it takes to come up with all the subgroups.  In particular, I do \emph{not} recommend taking a full brute-force approach like we did for $R_4$. 
%
%\begin{problem}
%Find all the subgroups of $R_5=\{e,r,r^2,r^3,r^4\}$ (where $r$ is rotation clockwise of a regular pentagon by $72^{\circ}$) and then draw the subgroup lattice for $R_5$.
%\end{problem}
%
%\begin{problem}
%Find all the subgroups of $R_6=\{e,r,r^2,r^3,r^4,r^5\}$ (where $r$ is rotation clockwise of a regular hexagon by $60^{\circ}$) and then draw the subgroup lattice for $R_6$.
%\end{problem}
%
%\begin{problem}\label{prob:latticeD3}
%Find all the subgroups of $D_3=\{e,r,r^2,s,sr,sr^2\}$ (where $r$ and $s$ are the usual actions) and then draw the subgroup lattice for $D_3$.
%\end{problem}
%
%\begin{problem}
%Find all the subgroups of $S_3=\langle s_1, s_2\rangle$ (where $s_1$ is the action is that swaps the positions of the first and second coins and $s_2$ is the action that swaps the second and third coins; see Problem~\ref{prob:S3}) and then draw the subgroup lattice for $S_3$. How does your lattice compare to the one in Problem~\ref{prob:latticeD3}? You should look back at Problem~\ref{prob:D3_iso_S3} and ponder what just happened.
%\end{problem}
%
%\begin{problem}
%Find all the subgroups of $D_4=\{e,r,r^2,r^3,s,sr,sr^2,sr^3\}$ (where $r$ and $s$ are the usual actions) and then draw the subgroup lattice for $D_4$.
%\end{problem}
%
%\begin{problem}
%Find all the subgroups of $Q_8=\{1,-1,i,-i,j,-j,k,-k\}$ and then draw the subgroup lattice for $Q_8$.
%\end{problem}
%
%\begin{problem}
%What claims can be made about the subgroup lattices of two groups that are isomorphic? What claims can be made about the subgroup lattices of two groups that are not isomorphic?  What claims can be made about two groups if their subgroup lattices look nothing alike?  \emph{Hint:} The answers to two of these questions should be obvious, but the answer to the remaining question should be something like, ``we don't have enough information to make any claims."
%\end{problem}
%
%Here are two final problems to conclude this section.
%
%\begin{problem}
%Several times we've referred to the fact that some subgroups are visible in a Cayley diagram for the parent group and some subgroups are not.  Suppose $(G,*)$ is a group and let $H\leq G$.  Can you describe a process for creating a Cayley diagram for $G$ that ``reveals" the subgroup $H$ inside of this Cayley diagram?
%\end{problem}
%
%\begin{problem}
%Suppose $(G,*)$ is a finite group and let $H\leq G$.  Can you describe a process that ``reveals" the subgroup $H$ inside the group table for $G$?  Where will the clones for $H$ end up?
%\end{problem}

\end{section}


%%----------------------------%%

\begin{section}{Isomorphisms}

%%----------------------------%%

Coming soon.

%Consider three light switches on a wall side by side.  Consider the group of actions that consists of all possible actions that you can do to the three light switches (involving ``on" and ``off").  Let's call this group $L_3$. It should be easy to see that $L_3$ has 8 distinct actions.
%\begin{enumerate}[label=\rm{(\alph*)}]
%\item Can you find a minimal generating set for $L_3$?  If so, give these actions names and then write all of the actions of $L_3$ as words in your generator(s).
%\item Using your generating set from part (a), draw a Cayley diagram for $L_3$.
%\item Is $L_3$ cyclic? Briefly justify your answer.
%\item Is $L_3$ abelian? Briefly justify your answer.
%\end{enumerate}


\end{section}

\chapter{Families of Groups} 
\label{chapter:families}
\thispagestyle{empty}

In this chapter we will explore a few families of groups, some of which we are already familiar with.

\begin{section}{Cyclic Groups}

Recall that if $G$ is a group and $g\in G$, then the \textbf{cyclic subgroup generated by $g$} is given by
\[
\langle g\rangle =\{g^k\mid k\in\mathbb{Z}\}.
\]
It is important to point out that $\langle g\rangle$ may be finite or infinite. In the finite case, the Cayley diagram with generator $g$ gives us a good indication of where the word ``cyclic" comes from (see Problem~\ref{prob:Cayley_cyclic}).  If there exists $g\in G$ such that $G=\langle g\rangle$, then we say that $G$ is a \textbf{cyclic group}.  

\begin{problem}
List all of the elements in each of the following cyclic subgroups.
\begin{enumerate}[label=\textrm{(\alph*)}]
\item $\langle r\rangle$, where $r\in D_3$
\item $\langle r\rangle$, where $r\in R_4$
\item $\langle rs\rangle$, where $rs\in D_4$
\item $\langle r^2\rangle$, where $r^2\in R_6$
\item $\langle i\rangle$, where $i\in Q_8$
\item $\langle 6\rangle$, where $6\in \mathbb{Z}$ and the operation is ordinary addition
\end{enumerate}
\end{problem}

\begin{problem}\label{prob:subgroup_generated_by_matrix}
Consider the group of invertible $2\times 2$ matrices with real number entries under the operation of matrix multiplication.  This group is denoted by $\mathrm{GL}_2(\mathbb{R})$.  List the elements in the cyclic subgroups generated by each of the following matrices.
\begin{multicols}{3}
\begin{enumerate}[label=\textrm{(\alph*)}]
\item $\begin{bmatrix} 0 & -1\\ -1 & 0\end{bmatrix}$
\item $\begin{bmatrix} 0 & -1\\ 1 & 0\end{bmatrix}$
\item $\begin{bmatrix} 2 & 0\\ 0 & 1\end{bmatrix}$
\end{enumerate}
\end{multicols}
\end{problem}

\begin{problem}\label{prob:cyclic_or_not}
Determine whether each of the following groups is cyclic.  
%Determine whether each of the groups from Problem~\ref{prob:computing_orders} is cyclic.
If the group is cyclic, find at least one generator.
\begin{multicols}{2}
\begin{enumerate}[label=\textrm{(\alph*)}]
\item $S_2$
\item $R_3$
\item $R_4$
\item $V_4$
\item $R_5$
\item $R_6$
\item $D_3$
\item $R_7$
\item $R_8$
\item $\Spin_{1\times 2}$
\item $D_4$
\item $Q_8$
\end{enumerate}
\end{multicols}
\end{problem}

\begin{problem}
Determine whether each of the following groups is cyclic.  If the group is cyclic, find at least one generator. If you believe that a group is not cyclic, try to sketch an argument.
\begin{multicols}{2}
\begin{enumerate}[label=\textrm{(\alph*)}]
\item $(\mathbb{Z},+)$
\item $(\mathbb{R},+)$
\item $(\mathbb{R}^+,\cdot)$
\item $(\{6^n\mid n\in\mathbb{Z}\},\cdot)$
\end{enumerate}
\end{multicols}
\begin{enumerate}
\item[(e)] $\textrm{GL}_2(\mathbb{R})$ under matrix multiplication
\item[(f)] $\{(\cos(\pi/4) +i\sin(\pi/4))^n\mid n\in \mathbb{Z}\}$ under multiplication of complex numbers
\end{enumerate}
\end{problem}

\begin{theorem}\label{thm:cyclic_implies_abelian}
If $G$ is a cyclic group, then $G$ is abelian.
\end{theorem}

\begin{problem}\label{prob:abelian_does_not_imply_cyclic}
Provide an example of a finite group that is abelian but not cyclic.
\end{problem}

\begin{problem}
Provide an example of an infinite group that is abelian but not cyclic.
\end{problem}

\begin{theorem}\label{thm:subgroup_generated_by_inverse}
If $G$ is a group and $g\in G$, then $\langle g\rangle=\langle g^{-1}\rangle$.
\end{theorem}

\begin{theorem}
If $G$ is a cyclic group such that $G$ has exactly one element that generates all of $G$, then the order of $G$ is at most order 2.   
\end{theorem}

\begin{theorem}
If $G$ is a group such that $G$ has no proper nontrivial subgroups, then $G$ is cyclic.
\end{theorem}

Recall that the order of a group $G$, denoted $|G|$, is the number of elements in $G$. We define the \textbf{order} of an element $g$, written $|g|$, to be the order of $\langle g\rangle$.  That is, $|g|=|\langle g\rangle|$.  It is clear that $G$ is cyclic with generator $g$ if and only if $|G|=|g|$.

\begin{problem}
What is the order of the identity in any group?
\end{problem}

\begin{problem}\label{prob:computing_orders}
Find the orders of each of the elements in each of the groups in Problem~\ref{prob:cyclic_or_not}.
\end{problem}

\begin{problem}
Consider the group $(\mathbb{Z},+)$.  What is the order of 1?  Are there any elements in $\mathbb{Z}$ with finite order?
\end{problem}

\begin{problem}
Find the order of each of the matrices in Problem~\ref{prob:subgroup_generated_by_matrix}.
\end{problem}

The next result follows immediately from Theorem~\ref{thm:subgroup_generated_by_inverse}.

\begin{theorem}
If $G$ is a group and $g\in G$, then $|g|=|g^{-1}|$.
\end{theorem}

The next result should look familiar and will come in handy a few times in this chapter. We'll take the result for granted and not worry about proving it.

\begin{theorem}[Division Algorithm]
If $n$ is a positive integer and $m$ is any integer, then there exist unique integers $q$ (called the \textbf{quotient}) and $r$ (called the \textbf{remainder}) such that $m=nq+r$, where $0\leq r<n$.
\end{theorem}

For the forward implication in the next theorem, if $\langle g\rangle$ is finite, then there exists distinct positive integers $i$ and $j$ such that $g^i=g^j$.  Can you find a useful way to rewrite this equation? For the reverse implication, let $m\in\mathbb{Z}$ and use the Division Algorithm with $m$ and $n$.

\begin{theorem}\label{thm:finite_group_finite_exponent}
Suppose $G$ is a group and let $g\in G$. The subgroup $\langle g\rangle$ is finite if and only if there exists $n\in\mathbb{N}$ such that $g^n=e$.
\end{theorem}

\begin{corollary}\label{cor:finite_group_finite_exponent}
If $G$ is a finite group, then for all $g\in G$, there exists $n\in\mathbb{N}$ such that $g^n=e$.
\end{corollary}

Note that Theorem~\ref{thm:finite_group_finite_exponent} together with the Well-Ordering Principle guarantees the existence of a smallest positive integer $n$ such that $g^n=e$ for every $g$ in a finite group $G$. In the following theorem, the claim that the set contains $n$ distinct elements is not immediate.  You need to argue that there are no repeats in the list. Choose distinct $i,j\in\{0,1,\ldots,n-1\}$ such that $i\neq j$ and then show that $g^i\neq g^j$.  Consider a proof by contradiction and try to contradict the minimality of $n$.

\begin{theorem}
Suppose $G$ is a group and let $g\in G$ such that $\langle g\rangle$ is a finite group. If $n$ is the smallest positive integer such that $g^n=e$, then $\langle g\rangle = \{e, g, g^2, \ldots, g^{n-1}\}$ and this set contains $n$ distinct elements.
\end{theorem}

The next result provides an extremely useful interpretation of the order of an element.

\begin{corollary}\label{cor:order_smallest_exponent}
If $G$ is a group and $g\in G$ such that $\langle g\rangle$ is a finite group, then the order of $g$ is the smallest positive integer $n$ such that $g^n=e$.
\end{corollary}

\begin{problem}\label{prob:Cayley_cyclic}
Suppose $G$ is a finite cyclic group such that $G=\langle g\rangle$. Using the generating set $\{g\}$, what does the Cayley diagram for $G$ look like?
\end{problem}

\begin{problem}
Suppose $G$ is a finite cyclic group of order $n$ with generator $g$.  If we write down the group table for $G$ using $e, g, g^2, \ldots, g^{n-1}$ as the labels for the rows and columns, are there any interesting patterns in the table?
\end{problem}

\begin{problem}\label{prob:finite_pos_exps}
Notice that in the definition for $\langle g\rangle$, we allow the exponents on $g$ to be negative.  Explain why we only need to use positive exponents when $\langle g\rangle$ is a finite group.
\end{problem}

The Division Algorithm should come in handy when proving the next theorem.

\begin{theorem}\label{thm:criterion_on_powers}
Suppose $G$ is a group and let $g\in G$ such that $|g|=n$.  Then $g^i=g^j$ if and only if $n$ divides $i-j$.
\end{theorem}

\begin{corollary}
Suppose $G$ is a group and let $g\in G$ such that $|g|=n$.  If $g^k=e$, then $n$ divides $k$.
\end{corollary}

Recall that for $n\geq3$, $R_n$ is the group of rotational symmetries of a regular $n$-gon, where the operation is composition of actions.

\begin{theorem}
For all $n\geq 3$, $R_n$ is cyclic.
\end{theorem}

\begin{theorem}\label{thm:finite_cyclic_groups}
Suppose $G$ is a finite cyclic group of order $n$.  Then $G$ is isomorphic to $R_n$ if $n\geq 3$, $S_2$ if $n=2$, and the trivial group if $n=1$.
\end{theorem}

Most of the previous results have involved finite cyclic groups.  What about infinite cyclic groups?

For the forward implication in the following theorem, try a proof by contradiction and suppose there exists integers $i$ and $j$ such that $g^i=g^j$.

\begin{theorem}
Suppose $G$ is a group and let $g\in G$. The subgroup $\langle g\rangle$ is infinite if and only if each $g^k$ is distinct for all $k\in\mathbb{Z}$.
\end{theorem}

\begin{theorem}\label{thm:infinite_cyclic_groups}
If $G$ is an infinite cyclic group, then $G$ is isomorphic to $\mathbb{Z}$ (under the operation of addition).
\end{theorem}

The upshot of Theorems~\ref{thm:infinite_cyclic_groups} and \ref{thm:finite_cyclic_groups} is that up to isomorphism, we know exactly what all of the cyclic groups are.

We now turn our attention to two new groups. Recall that two integers are \textbf{relatively prime} if the only positive integer that divides both of them is 1.  That is, integers $n$ and $k$ are relatively prime if and only if $\gcd(n,k)=1$.

\begin{definition}\label{def:integers_modn}
Let $n\in\mathbb{N}$ and define the following sets.
\begin{enumerate}[label=\textrm{(\alph*)}]
\item $\mathbb{Z}_n:=\{0,1,\ldots,n-1\}$
\item $U_n:=\{k\in\mathbb{Z}_n\mid \gcd(n,k)=1\}$
\end{enumerate}
\end{definition}

\begin{example}
For example, $\mathbb{Z}_{12}=\{0,1,2,3,4,5,6,7,8,9,10,11\}$ while $U_{12}=\{1,5,7,11\}$ since 1, 5, 7, and 11 are the only elements in $\mathbb{Z}_{12}$ that are relatively prime to 12.
\end{example}

For each set in Definition~\ref{def:integers_modn}, the immediate goal is to determine a binary operation that will yield a group.  The key is to use modular arithmetic.  Let $n$ be a positive integer. To calculate the sum (respectively, product) of two integers modulo $n$ (we say ``mod $n$" for short), add (respectively, multiply) the two numbers and then find the remainder after dividing the sum (respectively, product) by $n$. For example, $4+9$ is $3$ mod $5$ since $13$ has remainder 3 when divided by 5.  Similarly, $4\cdot 9$ is 1 mod $5$ since 36 has remainder 1 when divided by 5. The hope is that these two operations turn $\mathbb{Z}_n$ and $U_n$ into groups.

We write $i\equiv j\pmod n$, and say ``$i$ is equivalent to $j$ modulo $n$" or ``$i$ is equal to $j$ modulo $n$", if $i$ and $j$ both have the same remainder when divided by $n$.  It is common to abbreviate ``modulo" as ``mod".  It is also common to write $i\equiv_n j$, or even $i=j$ if the context is perfectly clear.  

It is well-known, and not too hard to prove, that $\equiv_n$ is an equivalence relation on $\mathbb{Z}$.  The corresponding equivalence classes are called congruence classes.  The elements of a single congruence class are the integers that all have the same remainder when divided by $n$. According to the Division Algorithm, there are $n$ congruence classes modulo $n$, one for each of the remainders $0,1,\ldots, n-1$. We can think of $\mathbb{Z}_n$ as the set of canonical representatives of these equivalence classes.

\begin{theorem}\label{thm:mod_divisiblity}
Let $n$ be a positive integer and let $i,j\in\mathbb{Z}$. Then $i\equiv j\pmod n$ if and only if $n$ divides $i-j$.
\end{theorem}

The next result follows immediately from Theorems~\ref{thm:mod_divisiblity} and \ref{thm:criterion_on_powers}.

\begin{corollary}
Suppose $G$ is a group and let $g\in G$ such that $|g|=n$.  Then $g^i=g^j$ if and only if $i\equiv j\pmod n$.
\end{corollary}

There are two things to prove in the next theorem.  First, you need to prove that $\mathbb{Z}_n$ is a group under addition mod $n$, and then you need to argue that the group is cyclic.

\begin{theorem}
The set $\mathbb{Z}_n$ is a cyclic group under addition mod $n$.
\end{theorem}

Like the previous theorem, there are two things to prove for the next theorem. First, prove that $U_n$ is a group under multiplication mod $n$, and then argue that the group is abelian.

\begin{theorem}
The set $U_n$ is an abelian group under multiplication mod $n$.
\end{theorem}

\begin{problem}
Consider $\mathbb{Z}_4$.
\begin{enumerate}[label=\textrm{(\alph*)}]
\item Find the group table for $\mathbb{Z}_4$.
\item Is $\mathbb{Z}_4$ cyclic? If so, list elements of $\mathbb{Z}_4$ that individually generate $\mathbb{Z}_4$.  If $\mathbb{Z}_4$ is not cyclic, explain why.
\item Is $\mathbb{Z}_4$ isomorphic to either of $R_4$ or $V_4$? Justify your answer.
\item Draw the subgroup lattice for $\mathbb{Z}_4$.
\end{enumerate}
\end{problem}

The next two problems illustrate that $U_n$ may or may not be cyclic.

\begin{problem}\label{prob:U10}
Consider $U_{10}=\{1,3,7,9\}$.
\begin{enumerate}[label=\textrm{(\alph*)}]
\item Find the group table for $U_{10}$.
\item Is $U_{10}$ cyclic? If so, list elements of $U_{10}$ that individually generate $U_{10}$.  If $U_{10}$ is not cyclic, explain why.
\item Is $U_{10}$ isomorphic to either of $R_4$ or $V_4$? Justify your answer.
\item Is $U_{10}$ isomorphic to $\mathbb{Z}_4$? Justify your answer.
\item Draw the subgroup lattice for $U_{10}$.
\end{enumerate}
\end{problem}

\begin{problem}\label{prob:U12}
Consider $U_{12}=\{1,5,7,11\}$.
\begin{enumerate}[label=\textrm{(\alph*)}]
\item Find the group table for $U_{12}$.
\item Is $U_{12}$ cyclic? If so, list elements of $U_{12}$ that individually generate $U_{12}$.  If $U_{12}$ is not cyclic, explain why.
\item Is $U_{12}$ isomorphic to either of $R_4$ or $V_4$? Justify your answer.
\item Draw the subgroup lattice for $U_{12}$.
\end{enumerate}
\end{problem}

The upshot of the next theorem is that for $n\geq 3$, $\mathbb{Z}_n$ is just the set of exponents in the set $R_n=\{e,r,r^2,\ldots,r^{n-1}\}$ (where $e=r^0$).

\begin{theorem}\label{thm:Zn_iso_to_Rn}
For $n\geq 3$, $\mathbb{Z}_n\cong R_n$. Moreover, $\mathbb{Z}_2\cong S_2$ and $\mathbb{Z}_1$ is isomorphic to the trivial group.
\end{theorem}

The next result can be thought of as a repackaging of Theorems~\ref{thm:finite_cyclic_groups} and \ref{thm:infinite_cyclic_groups}.

\begin{theorem}
Let $G$ be a cyclic group. If the order of $G$ is infinite, then $G$ is isomorphic to $\mathbb{Z}$. If $G$ has finite order $n$, then $G$ is isomorphic to $\mathbb{Z}_n$.
\end{theorem}

Now that we have a complete description of the cyclic groups, let's focus our attention on subgroups of cyclic groups.  

\begin{theorem}\label{thm:subgroups_of_cyclic_groups}
Suppose $G$ is a cyclic group. If $H\leq G$, then $H$ is also cyclic.
\end{theorem}

It turns out that for proper subgroups, the converse of Theorem~\ref{thm:subgroups_of_cyclic_groups} is not true.

\begin{problem}
Provide an example of a group $G$ such that $G$ is not cyclic, but all proper subgroups of $G$ are cyclic.
\end{problem}

The next result officially settles Problem~\ref{prob:nZ}\ref{prob:nZothers} and also provides a complete description of the subgroups of infinite cyclic groups up to isomorphism.

\begin{corollary}\label{cor:subgroups_of_Z}
The subgroups of $\mathbb{Z}$ are precisely the groups $n\mathbb{Z}$ for $n\in \mathbb{Z}$.
\end{corollary}

Let's further explore finite cyclic groups.  By Corollary~\ref{cor:order_smallest_exponent}, the order of $g^m$ is the smallest positive exponent $k$ such that $(g^m)^k=e$. To prove the next theorem, first verify that $k=\frac{n}{\gcd(n,m)}$ has the desired property and then verify that it is the smallest such exponent.

\begin{theorem}\label{thm:order_of_power}
If $G$ is a finite cyclic group with generator $g$ such that $|G|=n$, then for all $m\in\mathbb{Z}$, $\displaystyle |g^m|=\frac{n}{\gcd(n,m)}$.
\end{theorem}

Here is an extensive hint for proving the next theorem. Use Theorem~\ref{thm:order_of_power} for the forward implication. For the reverse implication, first prove that for all $m\in\mathbb{Z}$, $\langle g^m\rangle=\langle g^{\gcd(m,n)}\rangle$ by proving two set containments. To show $\langle g^m\rangle\subseteq \langle g^{\gcd(m,n)}\rangle$, use the fact that there exists an integer $q$ such that $m=q\cdot \gcd(m,n)$. For the reverse containment, you may freely use a fact known as Bezout's Lemma, which states that $\gcd(m,n)=nx+my$ for some integers $x$ and $y$.

\begin{theorem}\label{thm:subgroups_gcd}
If $G$ is a finite cyclic group with generator $g$ such that $|G|=n$, then $\langle g^m\rangle=\langle g^k\rangle$ if and only if $\gcd(m,n)=\gcd(k,n)$.
\end{theorem}

\begin{problem}
Suppose $G$ is a cyclic group of order 12 with generator $g$. 
\begin{enumerate}[label=\textrm{(\alph*)}]
\item Find the orders of each of the following elements: $g^2$, $g^7$, $g^8$.
\item Which elements of $G$ individually generate $G$?
\end{enumerate}
\end{problem}

\begin{corollary}\label{cor:generator_relatively_prime}
Suppose $G$ is a finite cyclic group with generator $g$ such that $|G|=n$. Then $\langle g\rangle=\langle g^k\rangle$ if and only if $n$ and $k$ are relatively prime. That is, $g^k$ generates $G$ if and only if $n$ and $k$ are relatively prime.
\end{corollary}

\begin{problem}
Theorem~\ref{thm:order_of_power}, Theorem~\ref{thm:subgroups_gcd}, and Corollary~\ref{cor:generator_relatively_prime} are written using multiplicative notation.  Rewrite both of these results using additive notation.
\end{problem}

\begin{problem}
Consider $\mathbb{Z}_{18}$.
\begin{enumerate}[label=\textrm{(\alph*)}]
\item Find all of the elements of $\mathbb{Z}_{18}$ that individually generate all of $\mathbb{Z}_{18}$.
\item Draw the subgroup lattice for $\mathbb{Z}_{18}$. For each subgroup, list the elements of the corresponding set.  Moreover, circle the elements in each subgroup that individually generate that subgroup.  For example, $\langle 2\rangle=\{0,2,4,6,8,10,12,14,16\}$. In this case, we should circle 2, 4, 8, 10, 14, and 16 since each of these elements individually generate $\langle 2\rangle$ and none of the remaining elements do.  I'll leave it to you to figure out why this is true.
\end{enumerate}
\end{problem}

\begin{problem}
Repeat the above exercise, but this time use $\mathbb{Z}_{12}$ instead of $\mathbb{Z}_{18}$.
\end{problem}

\begin{corollary}
If $G$ is a finite cyclic group such that $|G|=p$, where $p$ is prime, then $G$ has no proper nontrivial subgroups.
\end{corollary}

\begin{problem}
If there is exactly one group up to isomorphism of order $n$, then to what group are all the groups of order $n$ isomorphic?
\end{problem}

\begin{problem}
Suppose $G$ is a group and $x,y\in G$ such that $|x|=m$ and $|y|=n$. Is it true that $|xy|=mn$?  If this is true, provide a proof.  If this is not true, then provide a counterexample.
\end{problem}

The punchline of the next two theorems is Theorem~\ref{thm:special_abelian_implies_cyclic}. To prove the next theorem,  first verify that $(xy)^{mn}=e$ and then suppose $|xy|=k$. What do you immediately know about the relationship between $k$ and $mn$? Next, consider $(xy)^{kn}$. Argue that $m$ divides $kn$ and then argue that $m$ divides $k$. Similarly, $n$ divides $k$. Ultimately, conclude that $mn=k$.

\begin{theorem}
Suppose $G$ is a finite abelian group and let $x,y\in G$ such that $|x|=m$ and $|y|=n$. If $\gcd(m,n)=1$, then $|xy|=mn$.
\end{theorem}

Here is a hint for proving the next theorem. Suppose $g\in G$ such that $|g|=n$. Let $h$ be an arbitrary element in $G$ such that $|h|=m$. You need to show that $m$ divides $n$. For sake of a contradiction, assume otherwise. Then there exists a prime $p$ whose multiplicity as a factor of $m$ exceeds that of $n$. Let $p^a$ be the highest power of $p$ in $m$ and $p^b$ be the highest power of $p$ in $n$, so $a>b$. Consider the elements $g^{p^a}$ and $h^{m/p^a}$.

\begin{theorem}
Suppose $G$ is a finite abelian group. If $n$ is the maximal order among all elements in $G$, then the order of every element in $G$ divides $n$.
\end{theorem}

Recall that every cyclic group is abelian (see Theorem~\ref{thm:cyclic_implies_abelian}).  However, we know that not every abelian group is cyclic (see Problem~\ref{prob:abelian_does_not_imply_cyclic}).  The next theorem tells us that abelian groups with some additional properties are cyclic.


Here is one method of attack for proving the next theorem. Let $n$ be the maximal order among the elements of $G$ and let $g\in G$ be an element with order $n$. Prove that $G=\langle g\rangle$.

\begin{theorem}\label{thm:special_abelian_implies_cyclic}
If $G$ is a finite abelian group with at most one subgroup of any order, then $G$ is cyclic.
\end{theorem}

\begin{problem}
Is the converse of Theorem~\ref{thm:special_abelian_implies_cyclic} true for finite groups?  That is, if $G$ is a finite cyclic group, does that imply that $G$ contains at most one subgroup of each order? If the answer is yes, then prove it.  Otherwise, provide a counterexample.
\end{problem}

We conclude this section with a couple interesting counting problems involving the number of generators of certain cyclic groups.

\begin{problem}
Let $p$ and $q$ be distinct primes. Find the number of generators of $\mathbb{Z}_{pq}$.
\end{problem}

\begin{problem}
Let $p$ be a prime. Find the number of generators of $\mathbb{Z}_{p^r}$, where $r$ is an integer greater than or equal to 1.
\end{problem}

\end{section}

\begin{section}{Dihedral Groups}

We can think of finite cyclic groups as groups that describe rotational symmetry.  In particular, $R_n$ is the group of rotational symmetries of a regular $n$-gon.  Dihedral groups are those groups that describe both rotational and reflectional symmetry of regular $n$-gons.

\begin{definition}\label{def:dihedral}
For $n\geq 3$, the \textbf{dihedral group} $D_n$ is defined to be the group consisting of the symmetry actions of a regular $n$-gon, where the operation is composition of actions.
\end{definition}

For example, as we've seen, $D_3$ and $D_4$ are the symmetry groups of equilateral triangles and squares, respectively.  The symmetry group of a regular pentagon is denoted by $D_5$.  It is a well-known fact from geometry that the composition of two reflections in the plane is a rotation by twice the angle between the reflecting lines.

\begin{theorem}
The group $D_n$ is a non-abelian group of order $2n$.
\end{theorem}

\begin{theorem}\label{thm:generators_Dn}
Fix $n\geq 3$ and consider $D_n$. Let $r$ be rotation clockwise by $360^{\circ}/n$  and let $s$ and $s'$ be any two adjacent reflections of a regular $n$-gon.  Then
\begin{enumerate}[label=\textrm{(\alph*)}]
\item $D_n=\langle r,s\rangle =\{\underbrace{e,r,r^2,\ldots, r^{n-1}}_{\text{rotations}},\underbrace{s,sr,sr^2,\ldots,sr^{n-1}}_{\text{reflections}}\}$ and
\item $D_n=\langle s,s'\rangle = \text{all possible products of }s\text{ and }s'$.
\end{enumerate}
\end{theorem}

The next result is an obvious corollary of Theorem~\ref{thm:generators_Dn}.

\begin{corollary}
For $n\geq 3$, $R_n\leq D_n$.
\end{corollary}

The following theorem generalizes many of the relations we have witnessed in the Cayley diagrams for the dihedral groups $D_3$ and $D_4$.

\begin{theorem}
Fix $n\geq 3$ and consider $D_n$. Let $r$ be rotation clockwise by $360^{\circ}/n$  and let $s$ and $s'$ be any two adjacent reflections of a regular $n$-gon.  Then the following relations hold.
\begin{enumerate}[label=\textrm{(\alph*)}]
\item $r^n = s^2 = (s')^2 =e$,
\item $r^{-k} = r^{n-k}$ (special case: $r^{-1}=r^{n-1}$),
\item $sr^k=r^{n-k}s$ (special case: $sr=r^{n-1}s$),
\item $\underbrace{ss's\cdots}_{n\text{ factors}}=\underbrace{s'ss'\cdots}_{n\text{ factors}}$.
\end{enumerate}
\end{theorem}

\begin{problem}
From Theorem~\ref{thm:generators_Dn}, we know
\[
D_n=\langle r,s\rangle =\{\underbrace{e,r,r^2,\ldots, r^{n-1}}_{\text{rotations}},\underbrace{s,sr,sr^2,\ldots,sr^{n-1}}_{\text{reflections}}\}.
\]
If you were to create the group table for $D_n$ so that the rows and columns of the table were labeled by $e,r,r^2,\ldots, r^{n-1},s,sr,sr^2,\ldots,sr^{n-1}$ (in exactly that order), do any patterns arise?  Where are the rotations? Where are the reflections?
\end{problem}

\begin{problem}
What does the Cayley diagram for $D_n$ look like if we use $\{r,s\}$ as the generating set?  What if we use $\{s,s'\}$ as the generating set?
\end{problem}

\end{section}

\begin{section}{Symmetric Groups}

Recall the groups $S_2$ and $S_3$ from Problems~\ref{prob:introducing_S2} and \ref{prob:introducing_S3}.  These groups act on two and three coins, respectively, that are in a row by rearranging their positions (but not flipping them over). These groups are examples of symmetric groups.  In general, the \textbf{symmetric group} on $n$ objects is the set of permutations that rearranges the $n$ objects.  The group operation is composition of permutations.  Let's be a little more formal.

\begin{definition}
A \textbf{permutation of a set $A$} is a function $\sigma:A\to A$ that is both one-to-one and onto.
\end{definition}

You should take a moment to convince yourself that the formal definition of a permutation agrees with the notion of rearranging the set of objects.  The do-nothing action is the identity permutation, i.e., $\sigma(a)=a$ for all $a\in A$.  There are many ways to represent a permutation.  One visual way is using \textbf{permutation diagrams}, which we will introduce via examples.

Consider the following diagrams:
\begin{multicols}{2}
\[\alpha=\begin{pdiag}{5}{1}
\put(0.2,2.3){{1}}\put(1.2,2.3){{2}}\put(2.2,2.3){{3}}\put(3.2,2.3){{4}}\put(4.2,2.3){{5}} 
\pdmap{1}{2}\pdmap{2}{3}\pdmap{3}{4}\pdmap{4}{5}\pdmap{5}{1}\pdendmapfill 
\end{pdiag}\]

\bigskip

\[\beta=\begin{pdiag}{5}{1}
\put(0.2,2.3){{1}}\put(1.2,2.3){{2}}\put(2.2,2.3){{3}}\put(3.2,2.3){{4}}\put(4.2,2.3){{5}} 
\pdmap{2}{4}\pdmap{4}{3}\pdmap{3}{2}\pdendmapfill 
\end{pdiag}\]

\[\sigma=\begin{pdiag}{5}{1}
\put(0.2,2.3){{1}}\put(1.2,2.3){{2}}\put(2.2,2.3){{3}}\put(3.2,2.3){{4}}\put(4.2,2.3){{5}} 
\pdtrans{1}{3}\pdmap{2}{5}\pdmap{5}{4}\pdmap{4}{2}\pdendmapfill 
\end{pdiag}\]

\bigskip

\[\gamma=\begin{pdiag}{5}{1}
\put(0.2,2.3){{1}}\put(1.2,2.3){{2}}\put(2.2,2.3){{3}}\put(3.2,2.3){{4}}\put(4.2,2.3){{5}}
\pdtrans{1}{5}\pdendmapfill 
\end{pdiag}\]
\end{multicols}
\noindent Each of these diagrams represents a permutation on five objects.  I've given the permutations the names $\alpha$, $\beta$, $\sigma$, and $\gamma$.  The intention is to read the diagrams from the top down.  The numbers labeling the nodes along the top are identifying position.  Following an edge from the top row of nodes to the bottom row of nodes tells us what position an object moves to.  It is important to remember that the numbers are referring to the position of an object, not the object itself.  For example, $\beta$ is the permutation that sends the object in the second position to the fourth position, the object in the third position to the second position, and the object in the fourth position to the third position.  Moreover, the permutation $\beta$ doesn't do anything to the objects in positions 1 and 5.

\begin{problem}
Describe in words what the permutations $\sigma$ and $\gamma$ do.
\end{problem}

\begin{problem}
Draw the permutation diagram for the do-nothing permutation on 5 objects.  This is called the \textbf{identity permutation}. What does the identity permutation diagram look like in general for arbitrary $n$?
\end{problem}

\begin{definition}
The set of all permutations on $n$ objects is denoted by $S_n$.
\end{definition}

\begin{problem}
Draw all the permutation diagrams for the permutations in $S_3$.
\end{problem}

\begin{problem}
How many distinct permutations are there in $S_4$?  How about $S_n$ for any $n\in \mathbb{N}$?
\end{problem}

If $S_n$ is going to be a group, we need to know how to compose permutations.  This is easy to do using the permutation diagrams.  Consider the permutations $\alpha$ and $\beta$ from earlier.  We can represent the composition $\alpha \circ \beta$ via

\bigskip

\[\alpha \circ \beta=\begin{pdiag}{5}{2}
\put(0.2,4.3){{1}}\put(1.2,4.3){{2}}\put(2.2,4.3){{3}}\put(3.2,4.3){{4}}\put(4.2,4.3){{5}} 
\pdname{\tiny \beta}\pdmap{2}{4}\pdmap{4}{3}\pdmap{3}{2}\pdendmapfill 
\pdname{\tiny \alpha}\pdmap{1}{2}\pdmap{2}{3}\pdmap{3}{4}\pdmap{4}{5}\pdmap{5}{1}\pdendmapfill 
\end{pdiag}=
\begin{pdiag}{5}{1}
\put(0.2,2.3){{1}}\put(1.2,2.3){{2}}\put(2.2,2.3){{3}}\put(3.2,2.3){{4}}\put(4.2,2.3){{5}} 
\pdmap{1}{2}\pdmap{2}{5}\pdmap{5}{1}\pdendmapfill 
\end{pdiag}.\]
\noindent As you can see by looking at the figure, to compose two permutations, you stack the one that goes first in the composition (e.g., $\beta$ in the example above) on top of the other and just follow the edges from the top through the middle to the bottom.  If you think about how function composition works, this is very natural.  The resulting permutation is determined by where we begin and where we end in the composition.

We already know that the order of composition matters for functions, and so it should matter for the composition of permutations. To make this crystal clear, let's compose $\alpha$ and $\beta$ in the opposite order.  We see that

\bigskip

\[\beta \circ \alpha=\begin{pdiag}{5}{2}
\put(0.2,4.3){{1}}\put(1.2,4.3){{2}}\put(2.2,4.3){{3}}\put(3.2,4.3){{4}}\put(4.2,4.3){{5}} 
\pdname{\tiny \alpha}\pdmap{1}{2}\pdmap{2}{3}\pdmap{3}{4}\pdmap{4}{5}\pdmap{5}{1}\pdendmapfill 
\pdname{\tiny \beta}\pdmap{2}{4}\pdmap{4}{3}\pdmap{3}{2}\pdendmapfill 
\end{pdiag}=
\begin{pdiag}{5}{1}
\put(0.2,2.3){{1}}\put(1.2,2.3){{2}}\put(2.2,2.3){{3}}\put(3.2,2.3){{4}}\put(4.2,2.3){{5}} 
\pdmap{1}{4}\pdmap{4}{5}\pdmap{5}{1}\pdendmapfill 
\end{pdiag}.\]

\noindent The moral of the story is that composition of permutations does not necessarily commute.

\begin{problem}
Consider $\alpha$, $\beta$, $\sigma$, and $\gamma$ from earlier.  Can you find a pair of permutations that do commute?  Can you identify any features about your diagrams that indicate why they commuted?
\end{problem}

\begin{problem}
Fix $n\in\mathbb{N}$.  Convince yourself that any $\rho\in S_n$ composed with the identity permutation (in either order) equals $\rho$.
\end{problem}

If $S_n$ is going to be a group, we need to know what the inverse of a permutation is.

\begin{problem}
Given a permutation $\rho\in S_n$, describe a method for constructing $\rho^{-1}$.  Briefly justify that $\rho \circ \rho^{-1}$ will yield the identity permutation.
\end{problem}

At this point, we have all the ingredients we need to prove that $S_n$ forms a group under composition of permutations.

\begin{theorem}
The set of permutations on $n$ objects forms a group under the operation of composition.  That is, $(S_n,\circ)$ is a group.  Moreover, $|S_n|=n!$.
\end{theorem}

Note that it is standard convention to omit the composition symbol when writing down compositions in $S_n$.  For example, we will simply write $\alpha\beta$ to denote $\alpha \circ \beta$.

Permutation diagrams are fun to play with, but we need a more efficient way of encoding information.  There are several compact and efficient notations for describing permutations in $S_n$. For our purposes, it will be handy for us to describe permutations using \textbf{cycle notation}. The \textbf{cycle} $(a_1,a_2,\ldots, a_m)$ is the permutation that sends $a_i$ to $a_{i+1}$ for $1\leq i\leq m-1$ and sends $a_m$ to $a_1$. In general, for each $\sigma\in S_n$, the numbers 1 through $n$ will be rearranged and grouped into $k$ cycles of the form
\[
(a_1,a_2,\ldots, a_{m_1})(a_{m_1+1},a_{m_1+2},\ldots,a_{m_2})\cdots (a_{m_{k-1}+1},a_{m_{k-1}+2},\ldots,a_{m_k})
\]
from which the action of $\sigma$ on any number from 1 to $n$ can easily be determined.  In particular, for any $i\in\{1,2,\ldots,n\}$, locate $i$ in the expression above.  Then $\sigma(i)$ is the next number in the corresponding cycle that is cyclicly to the right (i.e., if $i$ is not at the right end of a cycle, $\sigma(i)$ is the next number to the right, while if $i$ is at the right end of a cycle, $\sigma(i)$ is the number at the left end of the same cycle). The product of all the cycles is called the \textbf{cycle decomposition} of $\sigma$.  

Notice that we can start writing a cycle with any of the numbers appearing in the cycle.  What matters is that each number in the cycle is followed by the appropriate number.  For example, $(1,3,2)=(3,2,1)=(2,1,3)$.  The \textbf{length} of a cycle is the number of entries appearing in it. If a cycle has length $m$, then it is called an \textbf{$m$-cycle}. Two cycles are said to be \textbf{disjoint} if they have no entries in common.

\begin{example}
Consider $\sigma=(1,12,8,10,4)(2,13)(3)(5,11,7)(6,9)$ in $S_{13}$.  This cycle decomposition for $\sigma$ consists of five pairwise disjoint cycles: a 5-cycle, a 2-cycle, a 1-cycle, a 3-cycle, and another 2-cycle.  For convenience, it is common to omit any 1-cycles in the decomposition.  So, we may also write $\sigma=(1,12,8,10,4)(2,13)(5,11,7)(6,9)$, keeping in mind that the absence of a number means that the permutation maps that number to itself.
\end{example}

\begin{example}
Consider $\alpha, \beta, \sigma$, and $\gamma$ in $S_{5}$ that we had previously drawn permutation diagrams for.  Below I have indicated what each permutation is equal to using cycle notation.

\begin{align*}\alpha=&\begin{pdiag}{5}{1} 
\pdmap{1}{2}\pdmap{2}{3}\pdmap{3}{4}\pdmap{4}{5}\pdmap{5}{1}\pdendmapfill 
\end{pdiag}=\ (1,2,3,4,5)\\
\\
\beta=&\begin{pdiag}{5}{1} 
\pdmap{2}{4}\pdmap{4}{3}\pdmap{3}{2}\pdendmapfill 
\end{pdiag}=\ (2,4,3)\\
\\
\sigma=&\begin{pdiag}{5}{1} 
\pdtrans{1}{3}\pdmap{2}{5}\pdmap{5}{4}\pdmap{4}{2}\pdendmapfill 
\end{pdiag}=\ (1,3)(2,5,4)\\
\\
\gamma=&\begin{pdiag}{5}{1} 
\pdtrans{1}{5}\pdendmapfill 
\end{pdiag}=\ (1,5)
\end{align*} 
\end{example}

\begin{example}
The cycle decomposition of the identity permutation in $S_n$ is $(1)(2)\cdots (n)$.  It is common to simply write this as $(1)$, again keeping in mind that the the absence of a number means that the permutation maps that number to itself. One disadvantage to this approach is that we lose information about what $n$ is.
\end{example}

\begin{problem}
Suppose $\sigma\in S_{9}$ is defined by
\[
\sigma(1)=3, \sigma(2)=4, \sigma(3)=1, \sigma(4)=9, \sigma(5)=8, \sigma(6)=2, \sigma(7)=5, \sigma(8)=7, \sigma(9)=6.
\]
Find the cycle decomposition of $\sigma$. What are the lengths of the corresponding cycles?
\end{problem}

\begin{problem}\label{prob:S3-2}
Write down all 6 elements in $S_3$ using cycle notation.
\end{problem}

\begin{problem}\label{prob:S4}
Write down all 24 elements in $S_4$ using cycle notation.
\end{problem}

Suppose $\sigma\in S_n$.  Since $\sigma$ is a bijection, it is clear that it is possible to write $\sigma$ as a product of disjoint cycles such that each $i\in\{1,2,\ldots, n\}$ appears exactly once.

Let's see if we can figure out how to multiply elements of $S_n$ using cycle notation.  Consider the permutations $\alpha=(1,3,2)$ and $\beta=(3,4)$ in $S_4$.  To compute the composition $\alpha\beta=(1,3,2)(3,4)$, let's explore what happens in each position.  Since we are doing function composition, we should work our way from right to left.  Since 1 does not appear in the cycle notation for $\beta$, we know that $\beta(1)=1$ (i.e., $\beta$ maps 1 to 1).  Now, we see what $\alpha(1)=3$.  Thus, the composition $\alpha\beta$ maps 1 to 3 (since $\alpha\beta(1)=\alpha(\beta(1))=\alpha(1)=3$).  Next, we should return to $\beta$ and see what happens to 3---which is where we ended a moment ago.  We see that $\beta$ maps 3 to 4 and then $\alpha$ maps 4 to 4 (since 4 does not appear in the cycle notation for $\alpha$).  So, $\alpha\beta(3)=4$.  Continuing this way, we see that $\beta$ maps 4 to 3 and $\alpha$ maps 3 to 2, and so $\alpha\beta$ maps 4 to 2.  Lastly, since $\beta(2)=2$ and $\alpha(2)=1$, we have $\alpha\beta(2)=1$.  Putting this altogether, we see that $\alpha\beta=(1,3,4,2)$.  Now, you should try a few.  Things get a little trickier if the composition of two permutations results in a permutation consisting of more than a single cycle.

\begin{problem}
Consider $\alpha$, $\beta$, $\sigma$, and $\gamma$ for which we drew the permutation diagrams.  Using cycle notation, compute each of the following.
\begin{multicols}{2}
\begin{enumerate}[label=\textrm{(\alph*)}]
\item $\alpha\gamma$
\item $\gamma\alpha$
%\item $\alpha^2$
%\item $\alpha^3$
%\item $\alpha^4$
%\item $\alpha^5$
\item $\sigma\alpha$
\item $\alpha\sigma$
%\item $\alpha^{-1}\sigma^{-1}$
\item $\beta^2$
\item $\beta^3$
\item $\beta^4$
%\item $\beta\gamma\alpha$
\item $\sigma^3$
\item $\sigma^6$
\end{enumerate}
\end{multicols}
\end{problem}

\begin{problem}
Write down the group table for $S_3$ using cycle notation.
\end{problem}

In Problem~\ref{prob:S4}, one of the permutations you should have written down is $(1,2)(3,4)$.  This is a product of two disjoint 2-cycles.  It is worth pointing out that each cycle is a permutation in its own right.  That is, $(1,2)$ and $(3,4)$ are each permutations.  It just so happens that their composition does not ``simplify" any further.  Moreover, these two disjoint 2-cycles commute since $(1,2)(3,4)=(3,4)(1,2)$.  In fact, this phenomenon is always true.

\begin{theorem}
Suppose $\alpha$ and $\beta$ are two disjoint cycles.  Then $\alpha\beta=\beta\alpha$.  That is, products of disjoint cycles commute.
\end{theorem}

\begin{problem}
Compute the orders of all the elements in $S_3$.  See Problem~\ref{prob:S3-2}.
\end{problem}

\begin{problem}
Compute the orders of any twelve of the elements in $S_4$.  See Problem~\ref{prob:S4}.
\end{problem}

Computing the order of a permutation is fairly easy using cycle notation once we figure out how to do it for a single cycle.  In fact, you've probably already guessed at the following theorem.

\begin{theorem}
If $\alpha\in S_n$ such that $\alpha$ consists of a single $k$-cycle, then $|\alpha|=k$.
\end{theorem}

Recall that $\lcm(k_1,\ldots, k_m)$ is the \textbf{least common multiple} of $\{k_1,\ldots, k_m\}$.

\begin{theorem}
Suppose $\alpha\in S_n$ such that $\alpha$ consists of $m$ disjoint cycles of lengths $k_1,\ldots, k_m$.  Then $|\alpha|=\lcm(k_1,\ldots, k_m)$. 
\end{theorem}

\begin{problem}
Is the previous theorem true if we do not require the cycles to be disjoint?  Justify your answer.
\end{problem}

\begin{problem}
What is the order of $(1,4,7)(2,5)(3,6,8,9)$?
\end{problem}

\begin{problem}
Draw the subgroup lattice for $S_3$.
\end{problem}

\begin{problem}
Now, using $(1,2)$ and $(1,2,3)$ as generators, draw the Cayley diagram for $S_3$.  Look familiar?
\end{problem}

\begin{problem}
Consider $S_3$. It turns out that $S_3=\langle (1,2),(1,3),(2,3)\rangle$.
\begin{enumerate}[label=\textrm{(\alph*)}]
\item Using $(1,2)$, $(1,3)$, and $(2,3)$ as generators, draw the Cayley diagram for $S_3$.
\item Is $\{(1,2),(1,3),(2,3)\}$ a minimal generating set for $S_3$?  If so, explain why.  If not, find a subset of $\{(1,2),(1,3),(2,3)\}$ that is a minimal generating set.
\end{enumerate}
\end{problem}

\begin{problem}
Recall that there are $4!=24$ permutations in $S_4$.
\begin{enumerate}[label=\textrm{(\alph*)}]
\item Pick any 12 permutations from $S_4$ and verify that you can write them as words in the 2-cycles $(1,2), (1,3), (1,4), (2,3), (2,4),(3,4)$.  In most circumstances, your words will not consist of products of disjoint 2-cycles.  For example, the permutation $(1,2,3)$ can be decomposed into $(1,2)(2,3)$, which is a word consisting of two 2-cycles that happen to not be disjoint.
\item Using your same 12 permutations, verify that you can write them as words only in the 2-cycles $(1,2),(2,3),(3,4)$.
\end{enumerate}
By the way, it might take some trial and error to come up with a way to do this.  Moreover, there is more than one way to do it.
\end{problem}

As the previous exercises hinted at, the 2-cycles play a special role in the symmetric groups.  In fact, they have a special name.  A \textbf{transposition} is a single cycle of length 2.  In the special case that the transposition is of the form $(i,i+1)$, we call it an \textbf{adjacent transposition}.  For example, $(3,7)$ is a (non-adjacent) transposition while $(6,7)$ is an adjacent transposition.

It turns out that the set of transpositions in $S_n$ is a generating set for $S_n$.  In fact, the adjacent transpositions form an even smaller generating set for $S_n$.  To get some intuition, let's play with a few examples.

\begin{problem}
Try to write each of the following permutations as a product of transpositions.  You do not necessarily need to use adjacent transpositions.
\begin{enumerate}[label=\textrm{(\alph*)}]
\item $(3,1,5)$
\item $(2,4,6,8)$
\item $(3,1,5)(2,4,6,8)$
\item $(1,6)(2,5,3)$
\end{enumerate}
\end{problem}

The products you found in the previous exercise are called \textbf{transposition representations} of the given permutation.

\begin{problem}
Consider the arbitrary $k$-cycle $(a_1,a_2,\ldots, a_k)$ from $S_n$ (with $k\leq n$).  Find a way to write this permutation as a product of 2-cycles. 
\end{problem}

\begin{problem}
Consider the arbitrary 2-cycle $(a,b)$ from $S_n$.  Find a way to write this permutation as a product of adjacent 2-cycles.
\end{problem}

The previous two problems imply the following theorem.

\begin{theorem}
Consider $S_n$.
\begin{enumerate}[label=\textrm{(\alph*)}]
\item Every permutation in $S_n$ can be written as a product of transpositions.
\item Every permutation in $S_n$ can be written as a product of adjacent transpositions.
\end{enumerate}
\end{theorem}

\begin{corollary}
The set of transpositions (respectively, the set of adjacent transpositions) from $S_n$ forms a generating set for $S_n$.
\end{corollary}

\begin{problem}
The following diagram is an unlabeled version of the Cayley diagram for $S_4$ using the adjacent transpositions $(1,2)$, $(2,3)$, and $(3,4)$ as generators.  Pick a vertex to correspond to the identity, make a suitable choice for which arrows correspond to which generators, and then label the remaining vertices with permutations in $S_4$.  The graph given below is drawn on the a three-dimensional solid called the \textbf{permutahedron} (which is a \textbf{truncated octahedron}).
\begin{center}
\begin{tikzpicture}
\begin{scope}[scale=.2]
\node (3124) at (12.15,35.1) [v-tiny] {};
\node (4123) at (23,35.8) [v-tiny] {};
\node (4213) at (28.2,35.1) [v-tiny] {};
\node (3214) at (17,34.35) [v-tiny] {};
\node (4132) at (25.5,29.2) [v-faded] {};
\node (2134) at (4.15,27.15) [v-tiny] {};
\node (4312) at (36.2,27.15) [v-tiny] {};
\node (2314) at (13.5,24.6) [v-tiny] {};
\node (4231) at (32.8,21.9) [v-faded] {};
\node (3142) at (18.3,20.8) [v-faded] {};
\node (2143) at (7.7,20.6) [v-faded] {};
\node (4321) at (38.2,20.6) [v-tiny] {};
%%
\node (1234) at (.2,17.5) [v-tiny] {};
\node (3412) at (33.9,17.5) [v-tiny] {};
\node (1324) at (4.7,15.7) [v-tiny] {};
\node (2413) at (22.4,15.7) [v-tiny] {};
\node (3241) at (25.5,14.6) [v-faded] {};
\node (1243) at (4.15,11.1) [v-tiny] {};
\node (3421) at (36.2,11.1) [v-tiny] {};
\node (1423) at (13.5,6.8) [v-tiny] {};
\node (2341) at (23,5.3) [v-faded] {};
\node (1342) at (12.2,3.1) [v-tiny] {};
\node (2431) at (28.1,3.1) [v-tiny] {};
\node (1432) at (17,.6) [v-tiny] {};
%%
\draw [o2-faded] (4132) to (3142);
\draw [o2-faded] (2134) to (2143);
\draw [o2-faded] (4231) to (3241);
\draw [o2-faded] (2341) to (2431);
\draw [p2-faded] (4132) to (4231);
\draw [p2-faded] (3142) to (3241);
\draw [p2-faded] (2143) to (1243);
\draw [p2-faded] (2341) to (1342);
\draw [g2-faded] (4123) to (4132);
\draw [g2-faded] (4231) to (4321);
\draw [g2-faded] (3142) to (2143);
\draw [g2-faded] (3241) to (2341);
%%
\draw [o2] (3124) to (4123);
\draw [o2] (3214) to (4213);
\draw [o2] (4312) to (3412);
\draw [o2] (2314) to (2413);
\draw [o2] (4321) to (3421);
\draw [o2] (1234) to (1243);
\draw [o2] (1324) to (1423);
\draw [o2] (1342) to (1432);
%%
\draw [p2] (3124) to (3214);
\draw [p2] (2134) to (1234);
\draw [p2] (4312) to (4321);
\draw [p2] (2314) to (1324);
\draw [p2] (3412) to (3421); 
\draw [p2] (2413) to (1423);
\draw [p2] (2431) to (1432);
\draw [p2] (4123) to (4213);
%%
\draw [g2] (3124) to (2134);
\draw [g2] (4213) to (4312);
\draw [g2] (3214) to (2314);
\draw [g2] (1234) to (1324);
\draw [g2] (3412) to (2413);
\draw [g2] (1243) to (1342);
\draw [g2] (3421) to (2431);
\draw [g2] (1423) to (1432);
\end{scope}
\end{tikzpicture}
\end{center}
\end{problem}

It is important to point out that the transposition representation of a permutation is not unique.  That is, there are many words in the transpositions that will equal the same permutation.  This is exhibited in the previous problem where there are multiple paths from the vertex corresponding to the identity to another vertex in the Cayley diagram for $S_4$ using the adjacent transpositions as the generators.  However, as we shall see in the next section, given two transposition representations for the same permutation, the number of transpositions will have the same parity (i.e., even versus odd).

\begin{problem}\label{prob:Cayley diagrams for S4}
It turns out that
\[
S_4 =\langle (1,2), (1,3), (1,4)\rangle = \langle (1,2,3,4), (1,2)\rangle.
\]
Determine which of the generating sets listed above yield each of the Cayley diagrams given below. Label the vertices in each diagram with permutations (written in cycle notation as a product of disjoint cycles) of $S_4$. The graph on the right is sometimes called the \textbf{Nauru graph}.

\[
\begin{tikzpicture}[scale=.9]
\begin{scope}[shift={(0,0)}]
\node (a1) at (0:1) [v-tiny] {};  %% center hexagon
\node (a2) at (60:1) [v-tiny] {};
\node (a3) at (120:1) [v-tiny] {};
\node (a4) at (180:1) [v-tiny] {};
\node (a5) at (240:1) [v-tiny] {};
\node (a6) at (300:1) [v-tiny] {}; 
\node (b1) at (2,-.58) [v-tiny] {};  %% outer sides of 3 squares
\node (b2) at (.5,1.87) [v-tiny] {};
\node (b3) at (-.5,1.87) [v-tiny] {};  %% original yellow vertex
\node (b4) at (-2,-.58) [v-tiny] {};
\node (b5) at (-1.5,-1.44) [v-tiny] {};
\node (b6) at (1.5,-1.44) [v-tiny] {};
\node (c1) at (0:3) [v-tiny] {};   %% middle hexagon
\node (c2) at (60:3) [v-tiny] {};
\node (c3) at (120:3) [v-tiny] {};
\node (c4) at (180:3) [v-tiny] {};
\node (c5) at (240:3) [v-tiny] {};
\node (c6) at (300:3) [v-tiny] {};
\node (d1) at (0:4) [v-tiny] {};   % outer hexagon
\node (d2) at (60:4) [v-tiny] {};
\node (d3) at (120:4) [v-tiny] {};
\node (d4) at (180:4) [v-tiny] {};
\node (d5) at (240:4) [v-tiny] {};
\node (d6) at (300:4) [v-tiny] {};
\draw [o2] (a1) to (a2); \draw [p] (a3) to (a2); \draw [o2] (a3) to (a4);
\draw [p] (a5) to (a4); \draw [o2] (a5) to (a6); \draw [p] (a1) to (a6);
\draw [p] (b1) to (a1); \draw [p] (a2) to (b2); \draw [p] (b3) to (a3);
\draw [p] (a4) to (b4); \draw [p] (b5) to (a5); \draw [p] (a6) to (b6);
\draw [o2] (b1) to (c1); \draw [o2] (b2) to (c2); \draw [o2] (b3) to (c3);
\draw [o2] (b4) to (c4); \draw [o2] (b5) to (c5); \draw [o2] (b6) to (c6);
\draw [p] (c1) to (d1); \draw [p] (d2) to (c2); \draw [p] (c3) to (d3);
\draw [p] (d4) to (c4); \draw [p] (c5) to (d5); \draw [p] (d6) to (c6);
\draw [p] (b2) to (b3); \draw [p] (b4) to (b5); \draw [p] (b6) to (b1);
\draw [p] (c2) to (c1); \draw [p] (c4) to (c3); \draw [p] (c6) to (c5);
\draw [p] (d1) to (d2); \draw [o2] (d2) to (d3); \draw [p] (d3) to (d4);
\draw [o2] (d4) to (d5); \draw [p] (d5) to (d6); \draw [o2] (d6) to (d1);
\end{scope}
%%
\begin{scope}[shift={(9,0)},scale=1.75]
\node (4321) at (0:1) [v-tiny] {};
\node (3142) at (30:1) [v-tiny] {};
\node (2314) at (60:1) [v-tiny] {};
\node (1432) at (90:1) [v-tiny] {};
\node (4213) at (120:1) [v-tiny] {};
\node (3421) at (150:1) [v-tiny] {};
\node (2143) at (180:1) [v-tiny] {};
\node (1324) at (210:1) [v-tiny] {};
\node (4132) at (240:1) [v-tiny] {};
\node (3214) at (270:1) [v-tiny] {};
\node (2431) at (300:1) [v-tiny] {};
\node (1243) at (330:1) [v-tiny] {};
%%
\node (2341) at (0:2) [v-tiny] {};
\node (1342) at (30:2) [v-tiny] {};
\node (4312) at (60:2) [v-tiny] {};
\node (3412) at (90:2) [v-tiny] {};
\node (2413) at (120:2) [v-tiny] {};
\node (1423) at (150:2) [v-tiny] {};
\node (4123) at (180:2) [v-tiny] {};
\node (3124) at (210:2) [v-tiny] {};
\node (2134) at (240:2) [v-tiny] {};
\node (1234) at (270:2) [v-tiny] {};
\node (4231) at (300:2) [v-tiny] {};
\node (3241) at (330:2) [v-tiny] {};
%%
\draw [o2] (1432) to (2431); \draw [o2] (4213) to (3214);
\draw [o2] (2143) to (3142); \draw [o2] (1324) to (4321);
\draw [o2] (2413) to (3412); \draw [o2] (1342) to (2341);
\draw [o2] (4231) to (1234); \draw [o2] (3124) to (4123);
\draw [o2] (2314) to (4312); \draw [o2] (1243) to (3241);
\draw [o2] (4132) to (2134); \draw [o2] (3421) to (1423);
%%
\draw [g2] (1432) to (4132); \draw [g2] (2314) to (3214);
\draw [g2] (3421) to (4321); \draw [g2] (2143) to (1243);
\draw [g2] (3412) to (4312); \draw [g2] (2134) to (1234);
\draw [g2] (2341) to (3241); \draw [g2] (4123) to (1423);
\draw [g2] (4213) to (2413); \draw [g2] (2431) to (4231);
\draw [g2] (1324) to (3124); \draw [g2] (3142) to (1342);
%%
\draw [p2] (4213) to (1243); \draw [p2] (3421) to (2431);
\draw [p2] (1324) to (2314); \draw [p2] (4132) to (3142);
\draw [p2] (4312) to (1342); \draw [p2] (2134) to (3124);
\draw [p2] (1423) to (2413); \draw [p2] (4231) to (3241);
\draw [p2] (1432) to (3412); \draw [p2] (3214) to (1234);
\draw [p2] (2143) to (4123); \draw [p2] (4321) to (2341);
 \end{scope}
\end{tikzpicture}
\]
\end{problem}

\begin{problem}\label{prob:more Cayley diagrams for S4}
Two Cayley diagrams for the symmetric group $S_4$ are given below.
\newcommand\aaa{1}\newcommand\bbb{2}\newcommand\ccc{3}\newcommand\ddd{4.6}
%\tikzstyle{v-tiny} = [circle, draw, fill=lightgrey,inner sep=0pt, minimum size=2.5mm]
%\tikzstyle{p2} = [draw, very thick, cb-purple,stealth-stealth]
\[
\begin{tikzpicture}
\begin{scope}[shift={(0,0)}]
\node (nw1) at (-1.5,1.5) [v-tiny] {};
\node (nw2) at (-.5,1.5) [v-tiny] {}; 
\node (nw3) at (-1.5,.5) [v-tiny] {};
\node (ne1) at (1.5,1.5) [v-tiny] {};
\node (ne2) at (1.5,.5) [v-tiny] {}; 
\node (ne3) at (.5,1.5) [v-tiny] {};
\node (se1) at (1.5,-1.5) [v-tiny] {};
\node (se2) at (.5,-1.5) [v-tiny] {};
\node (se3) at (1.5,-.5) [v-tiny] {}; 
\node (sw1) at (-1.5,-1.5) [v-tiny] {};
\node (sw2) at (-1.5,-.5) [v-tiny] {}; 
\node (sw3) at (-.5,-1.5) [v-tiny] {};
%
\draw [p2] (nw2) to (ne3); \draw [p2] (ne2) to (se3);
\draw [p2] (se2) to (sw3); \draw [p2] (sw2) to (nw3);
\draw[o](nw2)to(nw1); \draw [o] (nw3) to (nw2); \draw [o] (nw1) to (nw3);
\draw[o](ne2)to(ne1); \draw [o] (ne3) to (ne2); \draw [o] (ne1) to (ne3);
\draw[o](se2)to(se1); \draw [o] (se3) to (se2); \draw [o] (se1) to (se3);
\draw[o](sw2)to(sw1); \draw [o] (sw3) to (sw2); \draw [o] (sw1) to (sw3);
%%
\node (NW1) at (-2.25,2.25) [v-tiny] {};
\node (NW2) at (-3.25,2.25) [v-tiny] {}; 
\node (NW3) at (-2.25,3.25) [v-tiny] {};
\node (NE1) at (2.25,2.25) [v-tiny] {};
\node (NE2) at (2.25,3.25) [v-tiny] {}; 
\node (NE3) at (3.25,2.25) [v-tiny] {};
\node (SE1) at (2.25,-2.25) [v-tiny] {};
\node (SE2) at (3.25,-2.25) [v-tiny] {};
\node (SE3) at (2.25,-3.25) [v-tiny] {}; 
\node (SW1) at (-2.25,-2.25) [v-tiny] {};
\node (SW2) at (-2.25,-3.25) [v-tiny] {}; 
\node (SW3) at (-3.25,-2.25) [v-tiny] {};
%%
\draw [p2] (NW3) to (NE2); \draw [p2] (NE3) to (SE2);
\draw [p2] (SE3) to (SW2); \draw [p2] (SW3) to (NW2);
%
\draw[o](NW2)to(NW1); \draw [o] (NW3) to (NW2); \draw [o] (NW1) to (NW3);
\draw[o](NE2)to(NE1); \draw [o] (NE3) to (NE2); \draw [o] (NE1) to (NE3);
\draw[o](SE2)to(SE1); \draw [o] (SE3) to (SE2); \draw [o] (SE1) to (SE3);
\draw[o](SW2)to(SW1); \draw [o] (SW3) to (SW2); \draw [o] (SW1) to (SW3);
%%
\draw [p2] (nw1) to (NW1); \draw [p2] (ne1) to (NE1);
\draw [p2] (se1) to (SE1); \draw [p2] (sw1) to (SW1);
\end{scope}
%%
\begin{scope}[shift={(9,0)}]
\node (ne-1) at (45:\aaa) [v-tiny] {};
\node (ne-2) at (67.5:\bbb) [v-tiny] {};
\node (ne-3) at (22.5:\bbb) [v-tiny] {};
\node (ne-4) at (67.5:\ccc) [v-tiny] {};
\node (ne-5) at (22.5:\ccc) [v-tiny] {};
\node (ne-6) at (45:\ddd) [v-tiny] {};
%%
\node (nw-1) at (45+90:\aaa) [v-tiny] {};
\node (nw-2) at (67.5+90:\bbb) [v-tiny] {};
\node (nw-3) at (22.5+90:\bbb) [v-tiny] {};
\node (nw-4) at (67.5+90:\ccc) [v-tiny] {};
\node (nw-5) at (22.5+90:\ccc) [v-tiny] {};
\node (nw-6) at (45+90:\ddd) [v-tiny] {};
%%
\node (se-1) at (45-90:\aaa) [v-tiny] {};
\node (se-2) at (67.5-90:\bbb) [v-tiny] {};
\node (se-3) at (22.5-90:\bbb) [v-tiny] {};
\node (se-4) at (67.5-90:\ccc) [v-tiny] {};
\node (se-5) at (22.5-90:\ccc) [v-tiny] {};
\node (se-6) at (45-90:\ddd) [v-tiny] {};
%%
\node (sw-1) at (45+180:\aaa) [v-tiny] {};
\node (sw-2) at (67.5+180:\bbb) [v-tiny] {};
\node (sw-3) at (22.5+180:\bbb) [v-tiny] {};
\node (sw-4) at (67.5+180:\ccc) [v-tiny] {};
\node (sw-5) at (22.5+180:\ccc) [v-tiny] {};
\node (sw-6) at (45+180:\ddd) [v-tiny] {};
%%
\node (se-1) at (-45:1) [v-tiny] {};
%%
\draw [o] (ne-2) to (ne-1); \draw [o] (ne-3) to (ne-2);
\draw [o] (ne-1) to (ne-3);
\draw [p] (ne-2) to (ne-4); \draw [p] (ne-5) to (ne-3);
\draw [o] (ne-4) to (ne-5); \draw [o] (ne-6) to (ne-4);
\draw [o] (ne-5) to (ne-6);
%%
\draw [o] (nw-2) to (nw-1); \draw [o] (nw-3) to (nw-2);
\draw [o] (nw-1) to (nw-3);
\draw [p] (nw-2) to (nw-4); \draw [p] (nw-5) to (nw-3);
\draw [o] (nw-4) to (nw-5); \draw [o] (nw-6) to (nw-4);
\draw [o] (nw-5) to (nw-6);
%%
\draw [o] (se-2) to (se-1); \draw [o] (se-3) to (se-2);
\draw [o] (se-1) to (se-3);
\draw [p] (se-2) to (se-4); \draw [p] (se-5) to (se-3);
\draw [o] (se-4) to (se-5); \draw [o] (se-6) to (se-4);
\draw [o] (se-5) to (se-6);
%%
\draw [o] (sw-2) to (sw-1); \draw [o] (sw-3) to (sw-2);
\draw [o] (sw-1) to (sw-3);
\draw [p] (sw-2) to (sw-4); \draw [p] (sw-5) to (sw-3);
\draw [o] (sw-4) to (sw-5); \draw [o] (sw-6) to (sw-4);
\draw [o] (sw-5) to (sw-6);
%%
\draw [p] (ne-1) to (nw-1); \draw [p] (se-1) to (ne-1);
\draw [p] (sw-1) to (se-1); \draw [p] (nw-1) to (sw-1);
%%
\draw [p] (nw-6) to (ne-6); \draw [p] (sw-6) to (nw-6);
\draw [p] (se-6) to (sw-6); \draw [p] (ne-6) to (se-6);
%%
\draw [p] (ne-4) to (nw-5); \draw [p] (se-4) to (ne-5);
\draw [p] (sw-4) to (se-5); \draw [p] (nw-4) to (sw-5);
%%
\draw [p] (nw-3) to (ne-2); \draw [p] (sw-3) to (nw-2);
\draw [p] (se-3) to (sw-2); \draw [p] (ne-3) to (se-2);
\end{scope}
\end{tikzpicture}
\]
Determine what generating sets will yield these Cayley diagrams. Then label the nodes with permutations in cycle notation, written as a product of disjoint cycles.
\end{problem}

Here is an interesting fact that I will let you ponder. The group of rigid motion symmetries for a cube is isomorphic to $S_4$.  %To convince yourself of this fact, first prove that this group has 24 actions and then ponder the action of $S_4$ on the four long diagonals of a cube. 
Is there a Cayley diagram in one of the last two problems that helps you visualize this fact?

\end{section}

\begin{section}{Permutation Groups and Cayley's Theorem}

It turns out that the subgroups of symmetric groups play an important role in group theory.

\begin{definition}
Every subgroup of a symmetric group is called a \textbf{permutation group}.
\end{definition}

The proof of the following theorem isn't too bad, but we'll take it for granted. After tinkering with a few examples, you should have enough intuition to see why the theorem is true and how a possible proof might go.

\begin{theorem}[Cayley's Theorem]
Every finite group is isomorphic to some permutation group.  In particular, if $G$ is a group of order $n$, then $G$ is isomorphic to a subgroup of $S_n$.
\end{theorem}

Cayley's Theorem guarantees that every finite group is isomorphic to a permutation group and it turns out that there is a rather simple algorithm for constructing the corresponding permutation group.  I'll briefly explain an example and then let you try a couple.

Consider the Klein four-group $V_4=\{e,v,h,vh\}$.  Recall that $V_4$ has the following group table.

\begin{center}
\begin{tabular}{c!{\vrule width 2pt}c|c|c|c}
$*$ & $e$ & $v$ & $h$ & $vh$ \\ \noalign{\hrule height 2pt}
$e$ & $e$ & $v$ & $h$ & $vh$ \\
\hline $v$ & $v$ & $e$ & $vh$ & $h$  \\
\hline $h$ & $h$ & $vh$ & $e$ & $v$\\
\hline $vh$ & $vh$ & $h$ & $v$ & $e$
\end{tabular}
\end{center}

If we number the elements $e,v,h,$ and $vh$ as $1,2,3,$ and $4$, respectively, then we obtain the following table.

\begin{center}
\begin{tabular}{c!{\vrule width 2pt}c|c|c|c}
 & $1$ & $2$ & $3$ & $4$ \\ \noalign{\hrule height 2pt}
$1$ & $1$ & $2$ & $3$ & $4$ \\
\hline $2$ & $2$ & $1$ & $4$ & $3$  \\
\hline $3$ & $3$ & $4$ & $1$ & $2$\\
\hline $4$ & $4$ & $3$ & $2$ & $1$
\end{tabular}
\end{center}

\noindent Comparing each of the four columns to the leftmost column, we can obtain the corresponding permutations.  In particular, we obtain
\begin{align*}
e&\leftrightarrow (1)\\
v&\leftrightarrow (1,2)(3,4)\\
h&\leftrightarrow (1,3)(2,4)\\
vh&\leftrightarrow(1,4)(2,3). 
\end{align*}
Do you see where these permutations came from?  The claim is that the set of permutations $\{(1),(1,2)(3,4),(1,3)(2,4),(1,4)(2,3)\}$ is isomorphic to $V_4$.  In this particular case, it's fairly clear that this is true.  However, it takes some work to prove that this process will always result in an isomorphic permutation group.  In fact, verifying the algorithm is essentially the proof of Cayley's Theorem. 

Since there are potentially many ways to rearrange the rows and columns of a given table, it should be clear that there are potentially many isomorphisms that could result from the algorithm described above.

Here's another way to obtain a permutation group that is isomorphic to a given group.  Let's consider $V_4$ again.  Recall that $V_4$ is a subset of $D_4$, which is the symmetry group for a square.  Alternatively, $V_4$ is the symmetry group for a non-square rectangle.  Label the corners of the rectangle 1, 2, 3, and 4 by starting in the upper left corner and continuing clockwise.  Recall that $v$ is the action that reflects the rectangle over the vertical midline.  The result of this action is that the corners labeled by 1 and 2 switch places and the corners labeled by 3 and 4 switch places.  Thus, $v$ corresponds to the permutation $(1,2)(3,4)$.  Similarly, $h$ swaps the corners labeled by 1 and 4 and the corners labeled by 2 and 3, and so $h$ corresponds to the permutation $(1,4)(2,3)$.  Notice that this is not the same answer we got earlier and that's okay as there may be many permutation representations for a given group.  Lastly, $vh$ rotates the rectangle $180^{\circ}$ which sends ends up swapping corners labeled 1 and 3 and swapping corners labeled by 2 and 4.  Therefore, $vh$ corresponds to the permutation $(1,3)(2,4)$.

\begin{problem}
Consider $D_4$.
\begin{enumerate}[label=\textrm{(\alph*)}]
\item Using the method outlined above, find a subgroup of $S_8$ that is isomorphic to $D_4$.
\item Label the corners of a square 1--4. Find a subgroup of $S_4$ that is isomorphic to $D_4$ by considering the natural action of $D_4$ on the labels on the corners of the square.
\end{enumerate}
\end{problem}

\begin{problem}
Consider $\mathbb{Z}_6$.
\begin{enumerate}[label=\textrm{(\alph*)}]
\item Using the method outlined earlier, find a subgroup of $S_6$ that is isomorphic to $\mathbb{Z}_6$.
\item Label the corners of a regular hexagon 1--6. Find a subgroup of $S_6$ that is isomorphic to $\mathbb{Z}_6$ by considering the natural action of $\mathbb{Z}_6$ on the labels on the corners of the hexagon.
\end{enumerate}
\end{problem}

\end{section}

\begin{section}{Alternating Groups}

In this section, we describe a special class of permutation groups.  To get started, let's play with a few exercises.

\begin{problem}
Write down every permutation in $S_3$ as a product of 2-cycles in the most efficient way you can find (i.e., use the fewest possible transpositions).  Now, write every permutation in $S_3$ as a product of adjacent 2-cycles, but don't worry about whether your decompositions are efficient.  Any observations about the number of transpositions you used in each case?  Think about even versus odd.
\end{problem}

\begin{theorem}
If $\alpha_1,\alpha_2,\ldots,\alpha_k$ is a collection of 2-cycles in $S_n$ such that $\alpha_1\alpha_2\cdots\alpha_k=(1)$, then $k$ must be even.
\end{theorem}

\begin{proof}
Suppose $\alpha_1,\alpha_2,\ldots,\alpha_k$ is a collection of 2-cycles in $S_n$ such that $\alpha_1\alpha_2\cdots\alpha_k=(1)$.  We need to show that $k$ is even. We proceed by strong induction. First, it is clear that the statement is not true when $k=1$, but is true when $k=2$. 

Now, assume that $k>2$ and if $j\leq k-1$ and we have a product of $j$ 2-cycles that equals the identity, then $j$ is even. Consider $\alpha_1\alpha_2$. The only possibilities are:
\begin{enumerate}
\item[(i)] $\alpha_1\alpha_2=(a,b)(a,b)$,
\item[(ii)] $\alpha_1\alpha_2=(a,b)(a,c)$,
\item[(iii)] $\alpha_1\alpha_2=(a,b)(c,d)$,
\item[(iv)] $\alpha_1\alpha_2=(a,b)(b,c)$.
\end{enumerate}
If case (i) happens, then
\[
(1)=\alpha_1\alpha_2\cdots\alpha_k=\alpha_3\alpha_4\cdots\alpha_k.
\]
Since the expression on the right consists of $k-2$ factors, $k-2$ must be even by induction, which implies that $k$ is even. Now, suppose we are in one of cases (ii), (iii), or (iv). Observe that:
\begin{enumerate}
\item[(ii)] $(a,b)(a,c)=(b,c)(a,b)$,
\item[(iii)] $(a,b)(c,d)=(c,d)(a,b)$,
\item[(iv)] $(a,b)(b,c)=(b,c)(a,c)$.
\end{enumerate}
In each case, we were able to move $a$ from the original left 2-cycle to a new right 2-cycle. That is, we were able to rewrite $\alpha_1\alpha_2$ so that $a$ does not appear in the left 2-cycle. Systematically repeat this process for the pairs $\alpha_2\alpha_3$, $\alpha_3\alpha_4$,\ldots, $\alpha_{k-1}\alpha_k$. If we ever encounter case (i) along the way, then we are done by induction. Otherwise, we are able to rewrite $\alpha_1\alpha_2\cdots\alpha_k$ so that $a$ only appears in the rightmost 2-cycle. But this implies that $\alpha_1\alpha_2\cdots\alpha_k$ does not fix $a$, which contradicts $\alpha_1\alpha_2\cdots\alpha_k =(1)$. This implies that at some point we must encounter case (i), and hence $k$ is even by induction.
\end{proof}

\begin{theorem}
If $\sigma\in S_n$, then every transposition representation of $\sigma$ has the same parity.
\end{theorem}

The previous theorem tells us that the following definition is well-defined.

\begin{definition}
A permutation is \textbf{even} (respectively, \textbf{odd}) if one of its transposition representations consists of an even (respectively, odd) number of transpositions.
\end{definition}

\begin{problem}
Classify all of the permutations in $S_3$ as even or odd.
\end{problem}

\begin{problem}
Classify all of the permutations in $S_4$ as even or odd.
\end{problem}

\begin{problem}
Identify the even permutations in the Cayley diagrams for $S_4$ given in Problems~\ref{prob:Cayley diagrams for S4} and \ref{prob:more Cayley diagrams for S4}. Notice any nice patterns?
\end{problem}

\begin{problem}
Determine whether $(1,4,2,3,5)$ is even or odd.  How about $(1,4,2,3,5)(7,9)$?
\end{problem}

\begin{problem}
Consider the arbitrary $k$-cycle $(a_1,a_2,\ldots, a_k)$ from $S_n$ (with $k\leq n$).  When will this cycle be odd versus even?  Briefly justify your answer. 
\end{problem}

\begin{problem}
Conjecture a statement about when a permutation will be even versus odd.  Briefly justify your answer.
\end{problem}

And finally, we are ready to introduce the alternating groups.

\begin{definition}
The set of all even permutations in $S_n$ is denoted by $A_n$ and is called the \textbf{alternating group}.
\end{definition}

Since we referred to $A_n$ as a group, it darn well better be a group! To show that $A_n$ is a group, argue that $A_n$ is a subgroup of $S_n$ using the Two-Step Subgroup Test (see Theorem~\ref{thm:subgroup_criterion}). As expected, for $n>1$, the order of $A_n$ is exactly half the order of $S_n$. To show that $|A_n|=n!/2$ for $n>1$, prove that the number of even permutations in $S_n$ is the same as the number of odd permutations in $S_n$.  Here is one way to accomplish this. Define $f:A_n\to S_n\setminus A_n$ via $f(\sigma)=(1,2)\sigma$.  Note that $S_n\setminus A_n$ is the set of odd permutations in $S_n$. Show that $f$ is a bijection.

\begin{theorem}
The set $A_n$ forms a group under composition of permutations and has order $n!/2$ when $n>1$.
\end{theorem}

\begin{problem}
Find $A_3$.  What group is $A_3$ isomorphic to?
\end{problem}

\begin{problem}
Find $A_4$ and then draw its subgroup lattice. Is $A_4$ abelian?
\end{problem}

\begin{problem}\label{prob:Cayley diagrams for A4}
Two Cayley diagrams for $A_4$ are shown below. 
\[
\begin{tikzpicture}[scale=.8,auto]
\begin{scope}[shift={(0,0)}]
\node (l1) at (0,2.25) [v-tiny] {};
\node (l2) at (.866,3.75) [v-tiny] {};
\node (l3) at (-.866,3.75) [v-tiny] {};
\node (t1) at (0,1) [v-tiny] {};
\node (t2) at (.866,-.5) [v-tiny] {};
\node (t3) at (-.866,-.5) [v-tiny] {};
\node (r1) at (-3.68,-1.125) [v-tiny] {};
\node (r2) at (-2.814,-2.5) [v-tiny] {};
\node (r3) at (-1.948,-1.125) [v-tiny] {};
\node (m2) at (1.948,-1.125) [v-tiny] {};
\node (m1) at (3.68,-1.125) [v-tiny] {};
\node (m3) at (2.814,-2.5) [v-tiny] {};
\draw [p2] (l2) to (m1);
\draw [p2] (m2) to (t2);
\draw [p2] (r2) to (m3);
\draw [p2] (l1) to (t1);
\draw [p2] (l3) to (r1);
\draw [p2] (r3) to (t3);
\draw [o] (l1) to (l2);
\draw [o] (l2) to (l3);
\draw [o] (l3) to (l1);
\draw [o] (t1) to (t3);
\draw [o] (t2) to (t1);
\draw [o] (t3) to (t2);
\draw [o] (r1) to (r2);
\draw [o] (r2) to (r3);
\draw [o] (r3) to (r1);
\draw [o] (m1) to (m2);
\draw [o] (m2) to (m3);
\draw [o] (m3) to (m1);
\node at (0,-.2) {};
\end{scope}
%%
\begin{scope}[shift={(10,.6)},scale=1.1]
\node (a2) at (-45:1) [v-tiny] {};
\node (a4) at (-135:1) [v-tiny] {};
\node (a6) at (-225:1) [v-tiny] {};
\node (a8) at (-315:1) [v-tiny] {};
\node (b1) at (0:2) [v-tiny] {};
\node (b3) at (-90:2) [v-tiny] {};
\node (b5) at (-180:2) [v-tiny] {};
\node (b7) at (-270:2) [v-tiny] {};
\node (c2) at (-45:4) [v-tiny] {};
\node (c4) at (-135:4) [v-tiny] {};
\node (c6) at (-225:4) [v-tiny] {};
\node (c8) at (-315:4) [v-tiny] {};
\draw [o] (b1) to (a8); \draw [o] (a8) to (a2); \draw [o] (a2) to (b1);
\draw [o] (a4) to (a6); \draw [o] (a6) to (b5); \draw [o] (b5) to (a4);
\draw [o] (c2) to (b3); \draw [o] (b3) to (c4); \draw [o] (c4) to (c2);
\draw [o] (c6) to (b7); \draw [o] (b7) to (c8); \draw [o] (c8) to (c6);
\draw [p] (b1) to (c2); \draw [p] (c2) to (c8); \draw [p] (c8) to (b1);
\draw [p] (b5) to (c6); \draw [p] (c6) to (c4); \draw [p] (c4) to (b5);
\draw [p] (b3) to (a2); \draw [p] (a2) to (a4); \draw [p] (a4) to (b3);
\draw [p] (a6) to (a8); \draw [p] (a8) to (b7); \draw [p] (b7) to (a6);
\end{scope}
\end{tikzpicture}
\]
Determine what generating sets will yield these Cayley diagrams. Then label the nodes with permutations in cycle notation, written as a product of disjoint cycles.
\end{problem}

\begin{problem}
What is the order of $A_5$?  Is $A_5$ abelian?
\end{problem}

\begin{problem}
What orders of elements occur in $S_6$ and $A_6$?  What about $S_7$ and $A_7$?
\end{problem}

\begin{problem}
Does $A_8$ contain an element of order 15?  If so, find one.  If not, explain why no such element exists.
\end{problem}

\begin{remark}
Below are a few interesting facts about $A_4$ and $A_5$, which we will state without proof.
\begin{enumerate}[label=\textrm{(\alph*)}]
\item The group of rigid motion symmetries for a regular tetrahedron is isomorphic to $A_4$.
\item You can arrange the Cayley diagram for $A_4$ with generators $(1,2)(3,4)$ and $(2,3,4)$ (see the left diagram in Problem~\ref{prob:Cayley diagrams for A4}) on a truncated tetrahedron, which is depicted in Figure~\ref{fig:TruncatedTetrahedron}.
\item You can arrange the Cayley diagram for $A_5$ with generators $(1,2)(3,4)$ and $(1,2,3,4,5)$ on a truncated icosahedron, which is given in Figure~\ref{fig:TruncatedIcosahedron}.  You can also arrange the Cayley diagram for $A_5$ with generators $(1,2,3)$ and $(1,5)(2,4)$ on a truncated dodecahedron seen in Figure~\ref{fig:TruncatedDodecahedron}. 
\end{enumerate}
\end{remark}

\begin{figure}[!ht]
\begin{center}
\subcaptionbox{\label{fig:TruncatedTetrahedron}}[.3\textwidth]{
\includegraphics[width=1.5in]{TruncatedTetrahedron}
}
\subcaptionbox{\label{fig:TruncatedIcosahedron}}[.3\textwidth]{
\includegraphics[width=1.5in]{TruncatedIcosahedron}}
\subcaptionbox{\label{fig:TruncatedDodecahedron}}[.3\textwidth]{
\includegraphics[width=1.5in]{TruncatedDodecahedron}}
\caption{Truncated tetrahedron, truncated icosahedron, and truncated dodecahedron. [Image source: \href{https://en.wikipedia.org/wiki/Truncated_tetrahedron}{Wikipedia}]}
\end{center}
\end{figure}

\end{section}
\chapter{Cosets, Lagrange's Theorem, and Normal Subgroups}
\label{chapter:cosets_lagrange_normal}
\thispagestyle{empty}

\begin{section}{Cosets}

Undoubtably, you've noticed numerous times that if $G$ is a group with $H\leq G$ and $g\in G$, then both $|H|$ and $|g|$ divide $|G|$.  The theorem that says this is always the case is called Lagrange's theorem and we'll prove it towards the end of this chapter.  We begin with a definition.

\begin{definition}
Let $G$ be a group and let $H\leq G$ and $g\in G$.  The subsets
\[
aH:=\{ah\mid h\in H\}
\]
and
\[
Ha:=\{ha\mid h\in H\}
\]
are called the \textbf{left} and \textbf{right cosets of $H$ containing $a$}, respectively.
\end{definition}

To gain some insight, let's tinker with an example.  Consider the dihedral group $D_3=\langle r,s\rangle$ and let $H=\langle s\rangle\leq D_3$.  To compute the right cosets of $H$, we need to multiply all of the elements of $H$ on the right by the elements of $G$.  We see that
\begin{align*}
& He =\{ee,se\}=\{e,s\}=H\\
& Hr=\{er,sr\}=\{r,sr\}\\
& Hr^2=\{er^2,sr^2\}=\{r^2,rs\}\\
& Hs=\{es,ss\}=\{s,e\}=H\\
& Hsr=\{esr,ssr\}=\{sr,r\}\\
& Hrs=\{ers,srs\}=\{rs,ssr^2\}=\{rs,r^2\}.
\end{align*}
Despite the fact that we made six calculations (one for each element in $D_3$), if we scan the list, we see that there are only 3 distinct cosets, namely
\begin{align*}
& H=He=Hs=\{e,s\}\\
& Hr=Hsr=\{r,sr\}\\
& Hr^2=Hrs=\{r^2,rs\}.
\end{align*}
We can make a few more observations.  First, the resulting cosets formed a partition of $D_3$.  That is, every element of $D_3$ appears in exactly one coset.  Moreover, all the cosets are the same size---two elements in each coset in this case.  Lastly, each coset can be named in multiple ways.  In particular, the elements of the coset are exactly the elements of $D_3$ we multiplied $H$ by.  For example, $Hr=Hsr$ and the elements of this coset are $r$ and $sr$.  Shortly, we will see that these observations hold, in general.

Here is another significant observation we can make.  Consider the Cayley diagram for $D_3$ with generators $r$ and $s$.

\tikzstyle{vert} = [circle, draw, fill=grey,inner sep=0pt, minimum size=8mm]
\tikzstyle{h} = [draw,very  thick, blue,stealth-stealth]
\tikzstyle{r} = [draw, very thick, red,-stealth]

\begin{center}
\begin{tikzpicture}[scale=.75,auto]
\node (e) at (90:2) [vert] {$e$};
\node (r) at (-30:2) [vert] {$r$};
\node (rr) at (210:2) [vert] {$r^2$};
\node (h) at (90:4) [vert] {$s$};
\node (hr) at (210:4) [vert] {$rs$};
\node (hrr) at (-30:4) [vert] {$sr$};
\draw [r] (e) to [bend left] (r);
\draw [r] (r) to [bend left] (rr);
\draw [r] (rr) to [bend left] (e);
\draw [r] (h) to [bend right] (hr);
\draw [r] (hr) to [bend right] (hrr);
\draw [r] (hrr) to [bend right] (h);
\draw [h] (e) to (h);
\draw [h] (r) to (hrr);
\draw [h] (rr) to (hr);
\end{tikzpicture}
\end{center}
Given this Cayley diagram, we can visualize the subgroup $H$ and it's clones.  Moreover, $H$ and it's clones are exactly the 3 right cosets of $H$.  We'll see that, in general, the \emph{right} cosets of a given subgroup are always the subgroup and its clones.

\begin{exercise}\label{exer:leftcosetsD3}
Consider the group $D_3$.  Find all the left cosets for $H=\langle s\rangle$.  Are they the same as the right cosets?  Are they the same as the subgroup $H$ and its clones that we can see in the Cayley graph for $D_3$ with generators $r$ and $s$?
\end{exercise}

As the previous exercise indicates, the collections of left and right cosets may not be the same and when they are not the same, the subgroup and its clones do not coincide with the left cosets.

You might be thinking that somehow right cosets are better than left cosets since we were able to visualize them in the Cayley graph.  However, this is not the case.  Our convention of composing actions from right to left is what is dictating the visualization.  If we had adopted a left to right convention, then we would be able to visualize the left cosets.  

Computing left and right cosets using a group table is fairly easy.  Hopefully, you figured out in Exercise~\ref{exer:leftcosetsD3} that the left cosets of $H=\langle s\rangle$ in $D_3$ are $H=\{e,s\}$, $srH=\{r^2,sr\}$, and $rsH=\{r,rs\}$.  Now, consider the following group table for $D_3$ that has the rows and columns arranged according to the left cosets of $H$.

\begin{center}
\begin{tabu}{c|[2pt]c|c|c|c|c|c}
$*$ & $e$ & $s$ & $sr$ & $r^2$ & $rs$ & $r$ \\ \tabucline[2pt]{-}
$e$ & $e$ & $s$ & $sr$ & $r^2$ & $rs$ & $r$\\
\hline $s$ & $s$ & $e$ & $r$ & $rs$ & $r^2$ & $sr$ \\
\hline $sr$ & $sr$ & $r^2$ & $e$ & $s$ & $r$ & $rs$\\
\hline $r^2$ & $r^2$ & $sr$ & $rs$ & $r$ & $s$ & $e$\\
\hline $rs$ & $rs$ & $r$ & $r^2$ & $sr$ & $e$ & $s$\\
\hline $r$ & $r$ & $rs$ & $s$ & $e$ & $sr$ & $r^2$\\
\end{tabu}
\end{center}
The left coset $srH$ must appear in the row labeled by $sr$ and in the columns labeled by the elements of $H=\{e,s\}$.  We've depicted this below.

\begin{center}
\begin{tabu}{c|[2pt]c|c|c|c|c|c}
$*$ & \cellcolor{lightgray}$e$ & \cellcolor{lightgray}$s$ & $sr$ & $r^2$ & $rs$ & $r$ \\ \tabucline[2pt]{-}
$e$ & $e$ & $s$ & $sr$ & $r^2$ & $rs$ & $r$\\
\hline $s$ & $s$ & $e$ & $r$ & $rs$ & $r^2$ & $sr$ \\
\hline \cellcolor{lightgray}$sr$ & \cellcolor{blue}$sr$ & \cellcolor{blue}$r^2$ & $e$ & $s$ & $r$ & $rs$\\
\hline $r^2$ & $r^2$ & $sr$ & $rs$ & $r$ & $s$ & $e$\\
\hline $rs$ & $rs$ & $r$ & $r^2$ & $sr$ & $e$ & $s$\\
\hline $r$ & $r$ & $rs$ & $s$ & $e$ & $sr$ & $r^2$\\
\end{tabu}
\end{center}
On the other hand, the right coset $Hsr$ must appear in the column labeled by $sr$ and the rows labeled by the elements of $H=\{e,s\}$:
\begin{center}
\begin{tabu}{c|[2pt]c|c|c|c|c|c}
$*$ & $e$ & $s$ & \cellcolor{lightgray}$sr$ & $r^2$ & $rs$ & $r$ \\ \tabucline[2pt]{-}
\cellcolor{lightgray}$e$ & $e$ & $s$ & \cellcolor{blue}$sr$ & $r^2$ & $rs$ & $r$\\
\hline \cellcolor{lightgray}$s$ & $s$ & $e$ & \cellcolor{blue}$r$ & $rs$ & $r^2$ & $sr$ \\
\hline $sr$ & $sr$ & $r^2$ & $e$ & $s$ & $r$ & $rs$\\
\hline $r^2$ & $r^2$ & $sr$ & $rs$ & $r$ & $s$ & $e$\\
\hline $rs$ & $rs$ & $r$ & $r^2$ & $sr$ & $e$ & $s$\\
\hline $r$ & $r$ & $rs$ & $s$ & $e$ & $sr$ & $r^2$\\
\end{tabu}
\end{center}
As we can see from the tables, $srH\neq Hsr$ since $\{sr,r^2\}\neq \{sr,r\}$.  If we color the entire group table for $D_3$ according to which \emph{left} coset an element belongs to, we get the following.
 
\begin{center}
\begin{tabu}{c|[2pt]c|c|c|c|c|c}
$*$ & $e$ & $s$ & $sr$ & $r^2$ & $rs$ & $r$ \\ \tabucline[2pt]{-}
$e$ & \cellcolor{red}$e$ & \cellcolor{red}$s$ & \cellcolor{blue}$sr$ & \cellcolor{blue}$r^2$ & \cellcolor{green}$rs$ & \cellcolor{green}$r$\\
\hline $s$ & \cellcolor{red}$s$ & \cellcolor{red}$e$ & \cellcolor{green}$r$ & \cellcolor{green}$rs$ & \cellcolor{blue}$r^2$ & \cellcolor{blue}$sr$ \\
\hline $sr$ & \cellcolor{blue}$sr$ & \cellcolor{blue}$r^2$ & \cellcolor{red}$e$ & \cellcolor{red}$s$ & \cellcolor{green}$r$ & \cellcolor{green}$rs$\\
\hline $r^2$ & \cellcolor{blue}$r^2$ & \cellcolor{blue}$sr$ & \cellcolor{green}$rs$ & \cellcolor{green}$r$ & \cellcolor{red}$s$ & \cellcolor{red}$e$\\
\hline $rs$ & \cellcolor{green}$rs$ & \cellcolor{green}$r$ & \cellcolor{blue}$r^2$ & \cellcolor{blue}$sr$ & \cellcolor{red}$e$ & \cellcolor{red}$s$\\
\hline $r$ & \cellcolor{green}$r$ & \cellcolor{green}$rs$ & \cellcolor{red}$s$ & \cellcolor{red}$e$ & \cellcolor{blue}$sr$ & \cellcolor{blue}$r^2$\\
\end{tabu}
\end{center}
We would get a similar table if we colored the elements according to the right cosets.

Let's tackle a few more examples.

\begin{exercise}
Consider $D_3$ and let $K=\langle r\rangle$.  
\begin{enumerate}
\item[(a)] Find all of the left cosets of $K$ and then find all of the right cosets of $K$ in $D_3$.  Any observations?
\item[(b)] Write down the group table for $D_3$, but this time arrange the rows and columns according to the left cosets for $K$.  Color the entire table according to which \emph{left} coset an element belongs to.  Can you visualize the observations you make in part (a)?
\end{enumerate}
\end{exercise}

\begin{exercise}
Consider $Q_8$.  Let $H=\langle i\rangle$ and $K=\langle -1\rangle$.
\begin{enumerate}
\item[(a)] Find all of the left cosets of $H$ and all of the right cosets of $H$ in $Q_8$.  
\item[(b)] Write down the group table for $Q_8$ so that rows and columns are arranged according to the left cosets for $H$.  Color the entire table according to which \emph{left} coset an element belongs to.
\item[(c)] Find all of the left cosets of $K$ and all of the right cosets of $K$ in $Q_8$.
\item[(d)] Write down the group table for $Q_8$ so that rows and columns are arranged according to the left cosets for $H$.  Color the entire table according to which \emph{left} coset an element belongs to.
\end{enumerate}
\end{exercise}

\begin{exercise}
Consider $S_4$.  Find all of the left cosets and all of the right cosets of $A_4$ in $S_4$.
\end{exercise}

\begin{exercise}
Consider $\mathbb{Z}_8$.  Find all of the left cosets and all of the right cosets of $\langle 4\rangle$ in $\mathbb{Z}_8$.
\end{exercise}

\begin{exercise}
Consider $(\mathbb{Z},+)$.  Find all of the left cosets and all of the right cosets of $3\mathbb{Z}$ in $\mathbb{Z}$.
\end{exercise}

Now that we've played with a few examples, let's make a few general observations.

\begin{theorem}
Let $G$ be a group and let $H\leq G$.
\begin{enumerate}
\item[(a)] If $a\in G$, then $a\in aH$ (respectively, $Ha$).
\item[(b)] \label{thm:coset_representative} If $b\in aH$ (respectively, $Ha$), then $aH=bH$ (respectively, $Ha=Hb$).
\item[(c)] If $a\in H$, then $aH=H=Ha$.
\item[(d)] If $a\notin H$, then for all $h\in H$, $ah\notin aH$ (respectively, $ha \notin Ha$).
\end{enumerate}
\end{theorem}

The upshot of Theorem~\ref{thm:coset_representative}(b) is that cosets can have different names.  In particular, if $b$ is an element of the left coset $aH$, then we could have just as easily called the coset by the name $bH$.  In this case, both $a$ and $b$ are called \textbf{coset representatives}.

In all of the examples we've seen so far, the left and right cosets partitioned $G$ into equal-sized chunks.  We need to prove that this is true in general.  To prove that the cosets form a partition, we'll define an appropriate equivalence relation.

\begin{theorem}
Let $G$ be a group and let $H\leq G$.  Define $\sim_L$ and $\sim_R$ via 
\begin{center}
$a\sim_L b$ iff $a^{-1}b\in H$
\end{center}
and
\begin{center}
$a\sim_R b$ iff $ab^{-1}\in H$.
\end{center}
Then both $\sim_L$ and $\sim_R$ are equivalence relations.\footnote{You only need to prove that either $\sim_L$ or $\sim_R$ is an equivalence relation as the proof for the other is similar.}
\end{theorem}

\begin{problem}
If $[a]_{\sim_L}$ (respectively, $[a]_{\sim_R}$) denotes the equivalence class of $a$ under $\sim_L$ (respectively, $\sim_R$), what is $[a]_{\sim_L}$ (respectively, $[a]_{\sim_R}$)?  \emph{Hint:} It's got something to do with cosets.
\end{problem}

\begin{corollary}
If $G$ is a group and $H\leq G$, then the left (respectively, right) cosets of $H$ form a partition of $G$.
\end{corollary}

Next, we argue that all of the cosets have the same size.

\begin{theorem}
Let $G$ be a group, $H\leq G$, and $a\in G$.  Define $\phi:H\to aH$ via $\phi(h)=ah$.  Then $\phi$ is one-to-one and onto.
\end{theorem}

\begin{corollary}\label{cor:cosets_same_size}
Let $G$ be a group and let $H\leq G$.  Then all of the left and right cosets of $H$ are the same size as $H$.  In other words $|aH|=|H|=|Ha|$ for all $a\in G$.\footnote{As you probably expect, $|aH|$ denotes the size of $aH$. Note that everything works out just fine even if $H$ has infinite order.}
\end{corollary}

\end{section}

\begin{section}{Lagrange's Theorem}

We're finally ready to state Lagrange's Theorem, which is named after the Italian born mathematician Joseph Louis Lagrange.  It turns out that Lagrange did not actually prove the theorem that is named after him.  The theorem was actually proved by Carl Friedrich Gauss in 1801.

\begin{theorem}[Lagrange's Theorem]
Let $G$ be a finite group and let $H\leq G$.  Then $|H|$ divides $|G|$.
\end{theorem}

This simple sounding theorem is extremely powerful.  One consequence is that groups and subgroups have a fairly rigid structure.  Suppose $G$ is a finite group and let $H\leq G$.  Since $G$ is finite, there must be a finite number of distinct left cosets, say $H, a_2H, \ldots, a_{n}H$.  Corollary~\ref{cor:cosets_same_size} tells us that each of these cosets is the same size.  In particular, Lagrange's Theorem implies that for each $\i\in\{1,\ldots, n\}$, $|a_iH|=|G|/n$, or equivalently $n=|G|/|a_iH|$.  We can depict this using the following figure, where each rectangle represents a coset and we've labeled a single coset representative in each case.

\begin{center}
\begin{tikzpicture}[scale=1.6]
\draw (0,0) -- (5,0) -- (5,2) -- (0,2) -- (0,0);
\draw (1,0) -- (1,2);
\draw (2,0) -- (2,2);
\draw (4,0) -- (4,2);
\draw[fill=black] (.2,1.75) circle (1pt) node[below right] {$e$};
\draw[fill=black] (1.2,1.75) circle (1pt) node[below right] {$a_2$};
\draw[fill=black] (4.2,1.75) circle (1pt) node[below right] {$a_{n}$};
\node at (.5,.5) {$H$};
\node at (1.5,.5) {$a_2H$};
\node at (4.5,.5) {$a_{n}H$};
\node at (3,1) {$\cdots$};
\end{tikzpicture}
\end{center}

One important consequence of Lagrange's Theorem is that it narrows down the possible sizes for subgroups.

\begin{exercise}
Suppose $G$ is a group of order 48.  What are the possible orders for subgroups of $G$?  
\end{exercise}

Lagrange's Theorem tells us what the possible orders of a subgroup are, but if $k$ is a divisor of the order of a group, it does not guarantee that there is a subgroup of order $k$.  As we shall see later, the converse of Lagrange's Theorem is true for abelian groups.  However, it's not true, in general.  It turns out that $A_4$ is the smallest example of a group where the converse of Lagrange's Theorem fails.

\begin{problem}
Consider the alternating group $A_4$.  Lagrange's Theorem tells us that the possible orders of subgroups for $A_4$ are 1, 2, 3, 4, 6, and 12.
\begin{enumerate}
\item[(a)] Find examples of subgroups of $A_4$ of orders 1, 2, 3, 4, and 12.
\item[(b)] Write down all of the elements of order 2 in $A_4$.
\item[(c)] Argue that any subgroup of $A_4$ that contains any two elements of order 2 must contain a subgroup isomorphic to $V_4$.
\item[(d)] Argue that if $A_4$ has a subgroup of order 6, that it cannot be isomorphic to $R_6$.
\item[(e)] It turns out that up to isomorphism, there are only two groups of order 6, namely $S_3$ and $R_6$.  Suppose that $H$ is a subgroup of $A_4$ of order 6.  Part (d) guarantees that $H\cong S_3$.   Argue that $H$ must contain all of the elements of order 2 from $A_4$.
\item[(f)] Explain why $A_4$ cannot have a subgroup of order 6. 
\end{enumerate}
\end{problem}

Using Lagrange's Theorem, we can quickly prove both of the following theorems.

\begin{theorem}\label{thm:order_element_divides_group_order}
Let $G$ be a finite group and let $a\in G$.  Then $|a|$ divides $|G|$.
\end{theorem}

\begin{theorem}
Every group of prime order is cyclic.
\end{theorem}

Since the converse of Lagrange's Theorem is not true, the converse of Theorem~\ref{thm:order_element_divides_group_order} is not true either.  However, it is much easier to find a counterexample.

\begin{problem}
Argue that $A_4$ does not have any elements of order 8.
\end{problem}

Lagrange's Theorem motivates the following definition.

\begin{definition}
Let $G$ be a group and let $H\leq G$.  The \textbf{index} of $H$ in $G$ is the number of cosets (left or right) of $H$ in $G$. Equivalently, if $G$ is finite, then the index of $H$ in $G$ is equal to $|G|/|H|$.  We denote the index via $[G:H]$.
\end{definition}

\begin{exercise}
Let $H=\langle (1,2)(3,4),(1,3)(2,4)\rangle$.
\begin{enumerate}
\item[(a)] Find $[A_4:H]$.
\item[(b)] Find $[S_4:H]$.
\end{enumerate}
\end{exercise}

\begin{exercise}
Find $[\mathbb{Z}:4\mathbb{Z}]$.
\end{exercise}

\end{section}

\begin{section}{Normal Subgroups}

We've seen an example where the left and right cosets of a subgroup were different and a few examples where they coincided.  In the latter case, the subgroup has a special name.

\begin{definition}
Let $G$ be a group and let $H\leq G$.  If $aH=Ha$ for all $a\in G$, then we say that $H$ is a \textbf{normal subgroup}.  If $H$ is a normal subgroup of $G$, then we write $H\trianglelefteq G$.
\end{definition}

\begin{exercise}
Provide an example of group that has a subgroup that is not normal.
\end{exercise}

\begin{problem}
Suppose $G$ is a finite group and let $H\leq G$.  If $H\trianglelefteq G$ and we arrange the rows and columns of the group table for $G$ according to the left cosets of $H$ and then color the corresponding cosets, what property will the table have?  Is the converse true?  That is, if the table has the property you discovered, will $H$ be normal in $G$?
\end{problem}

There are a few instances where we can guarantee that a subgroup will be normal.

\begin{theorem}
Suppose $G$ is a group.  Then $\{e\}\trianglelefteq G$ and $G\trianglelefteq G$.
\end{theorem}

\begin{theorem}
If $G$ is an abelian group, then all subgroups of $G$ are normal.
\end{theorem}

A group does not have to be abelian in order for all the proper subgroups to be normal.

\begin{problem}
Argue that all of the proper subgroups of $Q_8$ are normal in $Q_8$. 
\end{problem}

\begin{theorem}
Suppose $G$ is a group and let $H\leq G$ such that $[G:H]=2$.  Then $H\trianglelefteq G$.
\end{theorem}

It turns out that normality is not transitive.

\begin{problem}
Let $H=\langle (1,2)(3,4),(2,3)(1,4)\rangle$.  Show that $H\trianglelefteq A_4$ and $A_4\trianglelefteq S_4$, but $H\not\trianglelefteq S_4$.
\end{problem}

The previous problem illustrates that $H\trianglelefteq K \trianglelefteq G$ does not imply $H\trianglelefteq G$.

\begin{theorem}
Suppose $G$ is a group and let $H\leq G$.  Then $H\trianglelefteq G$ if and only if $aHa^{-1}=H$ for all $a\in G$, where
\[
aHa^{-1}=\{aha^{-1}\mid h\in H\}.
\]
\end{theorem}

The previous theorem is often used as the definition of normal.  It also motivates the following definition.

\begin{definition}
Let $G$ be a group and let $H\leq G$.  The \textbf{normalizer of $H$ in $G$} is defined via
\[
N_G(H):=\{g\in G\mid gHg^{-1}=H\}.
\]
\end{definition}

\begin{theorem}
If $G$ is a group and $H\leq G$, then $N_G(H)$ is a subgroup of $G$.
\end{theorem}

\begin{theorem}
If $G$ is a group and $H\leq G$, then $H\trianglelefteq N_G(H)$.  Moreover, $N_G(H)$ is the largest subgroup of $G$ in which $H$ is normal. 
\end{theorem}

It is worth pointing out that the ``smallest" $N_G(H)$ can be is $H$ itself---certainly a subgroup is a normal subgroup of itself.  Also, the ``largest" that $N_G(H)$ can be is $G$, which happens precisely when $H$ is normal in $G$.

\begin{exercise}
Find $N_{V_4}(D_4)$.
\end{exercise}

\begin{exercise}
Find $N_{\langle s\rangle}(S_3)$.
\end{exercise}

We conclude this chapter with a few remarks.  We've seen examples of groups that have subgroups that are normal and subgroups that are not normal.  In an abelian group, all the subgroups are normal.  It turns out that there are examples of groups that have no normal subgroups.  This groups are called \textbf{simple groups}.  The smallest simple group is $A_5$, which has 120 elements and lots of subgroups, none of which are normal.

%borrowing from wikipedia:

The classification of the finite simple groups is a theorem stating that every finite simple group belongs to one of four categories:
\begin{enumerate}
\item A cyclic group with prime order;
\item An alternating group of degree at least 5;
\item A simple group of Lie type, including both 
\begin{enumerate}
\item the classical Lie groups, namely the simple groups related to the projective special linear, unitary, symplectic, or orthogonal transformations over a finite field;
\item the exceptional and twisted groups of Lie type (including the Tits group);
\end{enumerate}
\item The 26 sporadic simple groups.
\end{enumerate}
These groups can be seen as the basic building blocks of all finite groups, in a way reminiscent of the way the prime numbers are the basic building blocks of the natural numbers.

The classification theorem has applications in many branches of mathematics, as questions about the structure of finite groups (and their action on other mathematical objects) can sometimes be reduced to questions about finite simple groups. Thanks to the classification theorem, such questions can sometimes be answered by checking each family of simple groups and each sporadic group.  The proof of the theorem consists of tens of thousands of pages in several hundred journal articles written by about 100 authors, published mostly between 1955 and 2004.

The classification of the finite simple groups is a modern achievement in abstract algebra and I highly encourage you to go learn more about it.

\end{section}
\chapter{Products and Quotients of Groups}
\label{chapter:products_quotients}

\begin{section}{Products of Groups}

In this section, we will discuss a method for using existing groups as building blocks to form new groups.

Suppose $(G,*)$ and $(H,\odot)$ are two groups.  Recall that the Cartesian product of $G$ and $H$ is defined to be
\[
G\times H=\{(g,h)\mid g\in G,h\in H\}
\]
Using the binary operations for the groups $G$ and $H$, we can define a binary operation on the set $G\times H$.  Define $\star$ on $G\times H$ via
\[
(g_1,h_1)\star(g_2,h_2)=(g_1*g_2,h_1\odot h_2).
\]
This looks fancier than it is.  We're just doing the operation of each group in the appropriate component.  It turns out that $(G\times H,\star)$ is a group.

\begin{theorem}
Suppose $(G,*)$ and $(H,\odot)$ are two groups, where $e$ and $e'$ are the identity elements of $G$ and $H$, respectively.   Then $(G\times H,\star)$ is a group, where $\star$ is defined as above.  Moreover, $(e,e')$ is the identity of $G\times H$ and the inverse of $(g,h)\in G\times H$ is given by $(g,h)^{-1}=(g^{-1},h^{-1})$.
\end{theorem}

We refer to $G\times H$ as the \textbf{direct product} of the groups $G$ and $H$.  In this case, each of $G$ and $H$ is called a \textbf{factor} of the direct product.  We often abbreviate $(g_1,h_1)\star(g_2,h_2)=(g_1*g_2,h_1\odot h_2)$ by $(g_1,h_1)(g_2,h_2)=(g_1 g_2,h_1 h_2)$. One exception to this is if we are using the operation of addition in each component.  For example, consider $\mathbb{Z}_4\times \mathbb{Z}_2$ under the operation of addition mod 4 in the first component and addition mod 2 in the second component.  Then
\[
\mathbb{Z}_4 \times \mathbb{Z}_2=\{(0,0),(1,0),(2,0),(3,0),(0,1),(1,1),(2,1),(3,1)\}.
\]
In this case, we will use additive notation in $\mathbb{Z}_4\times \mathbb{Z}_2$.  For example, in $\mathbb{Z}_4 \times \mathbb{Z}_2$ we have
\[
(2,1)+(3,1)=(1,0)
\]
and
\[
(1,0)+(2,1)=(3,1).
\]
Moreover, the identity of the group is $(0,0)$.  As an example, the inverse of $(1,1)$ is $(3,1)$ since $(1,1)+(3,1)=(0,0)$.  There is a very natural generating set for $\mathbb{Z}_4\times \mathbb{Z}_2$, namely, $\{(1,0),(0,1)\}$ since $1\in \mathbb{Z}_4$ and $1\in \mathbb{Z}_2$ generate $\mathbb{Z}_4$ and $\mathbb{Z}_2$, respectively.  The corresponding Cayley diagram is given in Figure~\ref{fig:Cayley_Z4xZ2}.

\tikzstyle{vert} = [circle, draw, fill=grey,inner sep=0pt, minimum size=6.5mm]
\tikzstyle{b} = [draw,very thick,blue,stealth-stealth]
\tikzstyle{r} = [draw, very thick, red,-stealth]

\begin{figure}[!ht]
\centering
\begin{tikzpicture}[scale=1.5,auto]
\node (01) at (135:2.2) [vert] {\tiny $(0,1)$};
\node (11) at (45:2.2) [vert] {\tiny $(1,1)$};
\node (21) at (-45:2.2) [vert] {\tiny $(2,1)$};
\node (31) at (-135:2.2) [vert] {\tiny $(3,1)$};
\node (00) at (135:1) [vert] {\tiny $(0,0)$};
\node (10) at (45:1) [vert] {\tiny $(1,0)$};
\node (20) at (-45:1) [vert] {\tiny $(2,0$};
\node (30) at (-135:1) [vert] {\tiny $(3,0)$};

\path[r] (00) to (10);
\path[r] (10) to (20);
\path[r] (20) to (30);
\path[r] (30) to (00);

\path[r] (01) to (11);
\path[r] (11) to (21);
\path[r] (21) to (31);
\path[r] (31) to (01);

\path[b] (00) to (01);
\path[b] (10) to (11);
\path[b] (20) to (21);
\path[b] (30) to (31);
\end{tikzpicture}
\caption{Cayley diagram for $\mathbb{Z}_4 \times \mathbb{Z}_2$ with generating set $\{(1,0),(0,1)\}$.}\label{fig:Cayley_Z4xZ2}
\end{figure}

\begin{problem}
Consider the group $\mathbb{Z}_4 \times \mathbb{Z}_2$.  Is this group abelian?  Is the group cyclic?  Determine whether $\mathbb{Z}_4 \times \mathbb{Z}_2$ is isomorphic to any of $D_4$, $Q_8$, $\mathbb{Z}_8$, or $L_3$.
\end{problem}

The upshot of the previous problem is that there are at least five groups of order 8 up to isomorphism.  It turns out that there are exactly five groups of order 8 up to isomorphism.  In particular, every group of order 8 is isomorphic to one of the following groups: $\mathbb{Z}_8$, $\mathbb{Z}_4 \times \mathbb{Z}_2$, $L_3$, $D_4$, and $Q_8$. Note that $R_8\cong \mathbb{Z}_8$ and $\Spin_{1\times 2}\cong D_4$. Three of the isomorphism classes correspond to abelian groups while the other two correspond to non-abelian groups.  Unfortunately, we will not develop the tools necessary to prove that this classification is complete.

The next two theorems should not be terribly surprising.

\begin{theorem}
If $G_1$ and $G_2$ are groups, then $G_1\times G_2\cong G_2\times G_1$.
\end{theorem}

\begin{theorem}
Suppose $G_1$ and $G_2$ are groups with identities $e_1$ and $e_2$, respectively.  Then $\{e_1\}\times G_2\cong G_2$ and $G_1\times \{e_2\}\cong G_1$.
\end{theorem}

There's no reason we can't take the direct product of more than two groups.  If $A_1, A_2, \ldots, A_n$ is a collection of sets, we define
\[
\prod_{i=1}^nA_i:=A_1\times A_2\times \cdots \times A_n.
\]
Each element of $\prod_{i=1}^nA_i$ is of the form $(a_1,a_2,\ldots, a_n)$, where $a_i\in A_i$.

\begin{theorem}
Let $G_1, G_2,\ldots, G_n$ be groups.  For $(a_1,a_2, \ldots, a_n), (b_1,b_2,\ldots, b_n)\in \prod_{i=1}^nG_i$, define
\[
(a_1,a_2, \ldots, a_n)(b_1,b_2,\ldots, b_n)=(a_1b_1,a_2b_2,\ldots, a_nb_n).
\]
Then $\prod_{i=1}^nG_i$, the \textbf{direct product} of $G_1,\ldots, G_n$, is a group under this binary operation.
\end{theorem}

One way to think about direct products is that we can navigate the product by navigating each factor simultaneously but independently. Computing the order of a group that is a direct product is straightforward.

\begin{theorem}
Let $G_1, G_2,\ldots, G_n$ be finite groups.  Then
\[
|G_1\times G_2\times \cdots \times G_n|=|G_1|\cdot|G_2|\cdots |G_n|.
\]
\end{theorem}

\begin{theorem}
Let $G_1, G_2,\ldots, G_n$ be groups.  Then $|G_1\times G_2\times \cdots \times G_n|$ is infinite if and only if at least one $|G_i|$ is infinite.
\end{theorem}

The following theorem should be clear.

\begin{theorem}\label{thm:product_abelian_groups}
Let $G_1, G_2,\ldots, G_n$ be groups.  Then $\prod_{i=1}^nG_i$ is abelian if and only if each $G_i$ is abelian.
\end{theorem}

Let's play with a few more examples.

\begin{problem}
Draw the Cayley diagram for $\mathbb{Z}_2\times \mathbb{Z}_3$ using $\{(1,0),(0,1)\}$ as the generating set. Is $\mathbb{Z}_2\times \mathbb{Z}_3$ an abelian group?  Is it cyclic? What familiar group is $\mathbb{Z}_2\times \mathbb{Z}_3$ isomorphic to?
\end{problem}

\begin{problem}
Consider $\mathbb{Z}_2\times \mathbb{Z}_2$ under the operation of addition mod 2 in each component.  Find a generating set for $\mathbb{Z}_2\times \mathbb{Z}_2$ and then create a Cayley diagram for this group.  What well-known group is $\mathbb{Z}_2\times \mathbb{Z}_2$ isomorphic to?
\end{problem}

Consider the similarities and differences between $\mathbb{Z}_2\times \mathbb{Z}_3$ and $\mathbb{Z}_2\times \mathbb{Z}_2$.  Both groups are abelian by Theorem~\ref{thm:product_abelian_groups}, but only the former is cyclic.  Here's another exercise.

%\begin{problem}
%Consider $\mathbb{Z}_2\times \mathbb{Z}_4$ under the operation of addition mod 2 in the first component and addition mod 4 in the second component. 
%\begin{enumerate}[label=\rm{(\alph*)}]
%\item Using $\{(1,0),(0,1)\}$ as the generating set, draw the Cayley diagram for $\mathbb{Z}_2\times \mathbb{Z}_4$.
%\item Draw the subgroup lattice for $\mathbb{Z}_2\times \mathbb{Z}_4$.
%\item Show that $\mathbb{Z}_2\times \mathbb{Z}_4$ is abelian but not cyclic.
%\item Argue that $\mathbb{Z}_2\times \mathbb{Z}_4$ cannot be isomorphic to any of $D_4$, $R_8$, and $Q_8$.
%\end{enumerate}
%\end{problem}

\begin{problem}
Consider the group $\mathbb{Z}_2 \times \mathbb{Z}_2 \times \mathbb{Z}_2$.  Find a generating set for $\mathbb{Z}_2 \times \mathbb{Z}_2 \times \mathbb{Z}_2$ and then create a Cayley diagram for this group.  Is there a group that we have seen before that $\mathbb{Z}_2 \times \mathbb{Z}_2 \times \mathbb{Z}_2$ isomorphic to?
\end{problem}

The next theorem tells us how to compute the order of an element in a direct product of groups.

\begin{theorem}
Suppose $G_1, G_2,\ldots, G_n$ are groups and let $(g_1,g_2,\ldots, g_n)\in \prod_{i=1}^nG_i$.  If $|g_i|=r_i<\infty$, then $|(g_1,g_2,\ldots, g_n)|=\lcm(r_1,r_2,\ldots,r_n)$.
\end{theorem}

\begin{problem}
Find the order of each of the following elements.
\begin{enumerate}[label=\rm{(\alph*)}]
\item $(6,5)\in\mathbb{Z}_{12}\times \mathbb{Z}_7$.
\item $(r,i)\in D_3\times Q_8$.
\item $((1,2)(3,4),3)\in S_4\times \mathbb{Z}_{15}$.
\end{enumerate}
\end{problem}

\begin{problem}
Find the largest possible order of elements in each of the following groups.
\begin{enumerate}[label=\rm{(\alph*)}]
\item $\mathbb{Z}_6\times \mathbb{Z}_8$
\item $\mathbb{Z}_9\times \mathbb{Z}_{12}$
\item $\mathbb{Z}_4\times \mathbb{Z}_{18}\times \mathbb{Z}_{15}$
\end{enumerate}
\end{problem}

\begin{theorem}
The group $\mathbb{Z}_m\times \mathbb{Z}_n$ is cyclic if and only if $m$ and $n$ are relatively prime.
\end{theorem}

\begin{corollary}
The group $\mathbb{Z}_m\times \mathbb{Z}_n$ is isomorphic to $\mathbb{Z}_{mn}$ if and only if $m$ and $n$ are relatively prime.
\end{corollary}

The previous results can be extended to more than two factors.

\begin{theorem}
The group $\prod_{i=1}^n \mathbb{Z}_{m_i}$ is cyclic and isomorphic to $\mathbb{Z}_{m_1m_2\cdots m_n}$ if and only if every pair from the collection $\{m_1,m_2,\ldots, m_n\}$ is relatively prime.
\end{theorem}

\begin{problem}
Determine whether each of the following groups is cyclic.
\begin{enumerate}[label=\rm{(\alph*)}]
\item $\mathbb{Z}_7\times \mathbb{Z}_8$
\item $\mathbb{Z}_7\times \mathbb{Z}_7$
\item $\mathbb{Z}_2\times \mathbb{Z}_7\times \mathbb{Z}_8$
\item $\mathbb{Z}_5\times \mathbb{Z}_7\times \mathbb{Z}_8$
\end{enumerate}
\end{problem}

\begin{theorem}
Suppose $n=p_1^{n_1}p_2^{n_2}\cdots p_r^{n_r}$, where each $p_i$ is a distinct prime number.  Then
\[
\mathbb{Z}_n\cong \mathbb{Z}_{p_1^{n_1}}\times \mathbb{Z}_{p_2^{n_2}}\times \cdots \times \mathbb{Z}_{p_r^{n_r}}.
\]
\end{theorem}

The next theorem tells us that the direct product of subgroups is always a subgroup.

\begin{theorem}\label{thm:ProductOfSubgroups}
Suppose $G_1$ and $G_2$ are groups such that $H_1\leq G_1$ and $H_2\leq G_2$.  Then $H_1\times H_2\leq G_1\times G_2$.
\end{theorem}

However, not every subgroup of a direct product has the form above. 

\begin{problem}
Find an example that illustrates that not every subgroup of a direct product is the direct product of subgroups of the factors.
\end{problem}

\begin{problem}
Can we extend Theorem~\ref{thm:ProductOfSubgroups} to normal subgroups?  That is, if $H_1\trianglelefteq G_1$ and $H_2\trianglelefteq G_2$, is it the case that $H_1\times H_2\trianglelefteq G_1\times G_2$? If so, prove it.  Otherwise, provide a counterexample.
\end{problem}

%\begin{theorem}
%Suppose $G_1$ and $G_2$ are groups with identities $e_1$ and $e_2$, respectively.  Then $\{e_1\}\times G_2\trianglelefteq G_1\times G_2$ and $G_1\times \{e_2\}\trianglelefteq G_1\times G_2$.
%\end{theorem}

The next theorem describes precisely the structure of finite abelian groups.  We will omit its proof, but allow ourselves to utilize it as needed.

\begin{theorem}[Fundamental Theorem of Finitely Generated Abelian Groups]
Every finitely generated abelian group $G$ is isomorphic to a direct product of cyclic groups of the form
\[
\mathbb{Z}_{p_1^{n_1}}\times \mathbb{Z}_{p_2^{n_2}}\times \cdots \times \mathbb{Z}_{p_r^{n_r}}\times \mathbb{Z}^k,
\]
where each $p_i$ is a prime number (not necessarily distinct).  The product is unique up to rearrangement of the factors.
\end{theorem}

Note that the number $k$ is called the \textbf{Betti number}.  A  finitely generated abelian group is finite if and only if the Betti number is 0.

\begin{problem}\label{prob:groupsOrder8}
Find all abelian groups up to isomorphism of order 8.  How many different groups up to isomorphism (both abelian and non-abelian) have we seen and what are they?
\end{problem}

\begin{problem}
Find all abelian groups up to isomorphism for each of the following orders.
\begin{enumerate}[label=\rm{(\alph*)}]
\item 16
\item 12
\item 25
\item 30
\item 60
\end{enumerate}
\end{problem}

\end{section}

\begin{section}{Quotients of Groups}

In the previous section, we discussed a method for constructing ``larger" groups from ``smaller" groups using a direct product construction.  In this section, we will in some sense do the opposite.

Problem~\ref{prob:checkerboard} hinted that if $H\leq G$ and we arrange the group table according to the left cosets of $H$, then the group table will have checkerboard pattern if and only if $H$ is normal in G (i.e., the left and right cosets of $H$ are the same).  For example, see the colored table prior to Problem~\ref{prob:normal_in_D3} versus the ones you created in Exercises~\ref{prob:normal_in_D3}, \ref{prob:normal_in_Q8}.  If we have the checkerboard pattern in the group table that arises from a normal subgroup, then by ``gluing together" the colored blocks, we obtain a group table for a smaller group that has the cosets as the elements. 

For example, let's consider $K=\langle -1\rangle \leq Q_8$.  Problem~\ref{prob:normal_in_Q8} showed us that $K$ is normal $Q_8$.  The left (and right) cosets of $K$ in $Q_8$ are
\[
K=\{1,-1\}, iK=\{i,-i\}, jK=\{j,-j\}, \text{ and } kK=\{k,-k\}.
\]
As you found in Problem~\ref{prob:normal_in_Q8}, if we arrange the rows and columns of $Q_8$ according to these cosets, we obtain the following group table.

\begin{center}
\begin{tabular}{c!{\vrule width 2pt}c|c|c|c|c|c|c|c}
$*$ & $1$ & $-1$ & $i$ & $-i$ & $j$ & $-j$ & $k$ & $-k$ \\ \noalign{\hrule height 2pt}
$1$ & \cellcolor{green}$1$ & \cellcolor{green}$-1$ & \cellcolor{blue}$i$ & \cellcolor{blue}$-i$ & \cellcolor{red}$j$ & \cellcolor{red}$-j$ & \cellcolor{purple}$k$ & \cellcolor{purple}$-k$\\
\hline $-1$ & \cellcolor{green}$-1$ & \cellcolor{green}$1$ & \cellcolor{blue}$-i$ & \cellcolor{blue}$i$ & \cellcolor{red}$-j$ & \cellcolor{red}$j$ & \cellcolor{purple}$-k$ & \cellcolor{purple}$k$ \\
\hline $i$ & \cellcolor{blue}$i$ & \cellcolor{blue}$-i$ & \cellcolor{green}$-1$ & \cellcolor{green}$1$ & \cellcolor{purple}$k$ & \cellcolor{purple}$-k$ & \cellcolor{red}$-j$ & \cellcolor{red} $j$\\
\hline $-i$ & \cellcolor{blue}$-i$ & \cellcolor{blue}$i$ & \cellcolor{green}$1$ & \cellcolor{green}$-1$ & \cellcolor{purple}$-k$ & \cellcolor{purple}$k$ & \cellcolor{red}$j$ & \cellcolor{red}$-j$\\
\hline $j$ & \cellcolor{red}$j$ & \cellcolor{red}$-j$ & \cellcolor{purple}$-k$ & \cellcolor{purple}$k$ & \cellcolor{green}$-1$ & \cellcolor{green}$1$ & \cellcolor{blue}$i$ & \cellcolor{blue}$-i$\\
\hline $-j$ & \cellcolor{red}$-j$ & \cellcolor{red}$j$ & \cellcolor{purple}$k$ & \cellcolor{purple}$-k$ & \cellcolor{green}$1$ & \cellcolor{green}$-1$ & \cellcolor{blue}$-i$ & \cellcolor{blue}$i$\\
\hline $k$ & \cellcolor{purple}$k$ & \cellcolor{purple}$-k$ & \cellcolor{red}$j$ & \cellcolor{red}$-j$ & \cellcolor{blue}$-i$ & \cellcolor{blue}$i$ & \cellcolor{green}$-1$ & \cellcolor{green}$1$\\
\hline $-k$ & \cellcolor{purple}$-k$ & \cellcolor{purple}$k$ & \cellcolor{red}$-j$ & \cellcolor{red}$j$ & \cellcolor{blue}$i$ & \cellcolor{blue}$-i$ & \cellcolor{green}$1$ & \cellcolor{green}$-1$
\end{tabular}
\end{center}

If we consider the $2\times 2$ blocks as elements, it appears that we have a group table for a group with 4 elements.  Closer inspection reveals that this looks like the table for $V_4$.  If the table of $2\times 2$ blocks is going to represent a group, we need to understand the binary operation.  How do we ``multiply" cosets?  For example, the table suggest that the coset $jK$ (colored in \textcolor{red}{red}) times the coset $iK$ (colored in \textcolor{blue}{blue}) is equal to $kK$ (colored in \textcolor{purple}{purple}) despite the fact that $ji=-k\neq k$.  Yet, it is true that the product $ji=-k$ is an element in the coset $kK$.  In fact, if we look closely at the table, we see that if we pick any two cosets, the product of any element of the first coset times any element of the second coset will always result in an element in the same coset regardless of which representatives we chose.

In other words, it looks like we can multiply cosets by choosing any representative from each coset and then seeing what coset the product of the representatives lies in.  However, it is important to point out that this will only work if we have a checkerboard pattern of cosets, which we have seen evidence of only happening when the corresponding subgroup is normal.

Before continuing, let's continue tinkering with the same example.  Consider the Cayley diagram for $Q_8$ with generators $\{i,j,-1\}$ that is given in Figure~\ref{fig:Q8QuotientA}.

\tikzstyle{2b} = [draw,very thick,blue,stealth-stealth]
\tikzstyle{2r} = [draw, very thick, red,stealth-stealth]
\tikzstyle{greenvert} = [circle, draw, fill=green,inner sep=0pt, minimum size=6.5mm]
\tikzstyle{vert} = [circle, draw, fill=grey,inner sep=0pt, minimum size=6.5mm]
\tikzstyle{b} = [draw,very thick,blue,-stealth]
\tikzstyle{r} = [draw, very thick, red,-stealth]
\tikzstyle{g} = [draw, very thick, green, stealth-stealth]

\begin{figure}[!ht]
\centering
\subcaptionbox{\label{fig:Q8QuotientA}}[.45\textwidth]{
\begin{tikzpicture}[scale=1.5,auto]
\node (1) at (135:2) [vert] {{\scriptsize $1$}};
\node (i) at (45:2) [vert] {\scriptsize {$i$}};
\node (k) at (-45:2) [vert] {{\scriptsize $k$}};
\node (j) at (-135:2) [vert] {{\scriptsize $j$}};
\node (-1) at (135:1) [vert] {{\scriptsize $-1$}};
\node (-i) at (45:1) [vert] {{\scriptsize $-i$}};
\node (-k) at (-45:1) [vert] {{\scriptsize $-k$}};
\node (-j) at (-135:1) [vert] {{\scriptsize $-j$}};
\path[b] (1) to (i);
\path[b] (i) to (-1);
\path[b] (-1) to (-i);
\path[b] (-i) to (1);
\path[b] (-j) to (-k);
\path[b] (-k) to (j);
\path[b] (j) to (k);
\path[b] (k) to (-j);
\path[r] (-k) to (-i);
\path[r] (-i) to (k);
\path[r] (k) to (i);
\path[r] (i) to (-k);
\path[r] (1) to (j);
\path[r] (j) to (-1);
\path[r] (-1) to (-j);
\path[r] (-j) to (1);
\path[g] (1) to (-1);
\path[g] (j) to (-j);
\path[g] (i) to (-i);
\path[g] (k) to (-k);
\end{tikzpicture}}
\subcaptionbox{\label{fig:Q8QuotientB}}[.45\textwidth]{
\begin{tikzpicture}[scale=1.5,auto]
\node (K) at (135:1.5) [greenvert] {{\scriptsize $K$}};
\node (iK) at (45:1.5) [greenvert] {\scriptsize {$iK$}};
\node (kK) at (-45:1.5) [greenvert] {{\scriptsize $kK$}};
\node (jK) at (-135:1.5) [greenvert] {{\scriptsize $jK$}};
\path[2b] (K) to (iK);
\path[2r] (iK) to (kK);
\path[2b] (kK) to (jK);
\path[2r] (jK) to (K);
\end{tikzpicture}}
\caption{The left subfigure shows the Cayley diagram for $Q_8$ with generating set $\{i,j,-1\}$. The right subfigure shows the collapsed Cayley diagram for $Q_8$ according to the left cosets of $K=\langle -1\rangle$.}
\label{fig:Q8Repeat}
\end{figure}

We can visualize the right cosets of $K$ as the clumps of vertices connected together with the two-way green arrows.  In this case, we are also seeing the left cosets since $K$ is normal in $Q_8$.  If we collapse the cosets onto each other and collapse the corresponding arrows, we obtain the diagram given in Figure~\ref{fig:Q8QuotientB}. It is clear that this diagram is the Cayley diagram for a group that is isomorphic to $V_4$.  For reasons we will understand shortly, this processing of collapsing a Cayley diagram according to the cosets of a normal subgroup is called the ``quotient process."

\begin{problem}\label{prob:bad_quotient}
Let's see what happens if we attempt the quotient process for a subgroup that is not normal.  Consider $H=\langle s\rangle \leq D_3$.  In Problem~\ref{prob:left_cosets_D3}, we discovered that the left cosets of $H$ are not the same as the right cosets of $H$.  This implies that $H$ is not normal in $D_3$.  Consider the standard Cayley diagram for $D_3$ that uses the generators $r$ and $s$.  Draw the diagram that results from attempting the quotient process on $D_3$ using the subgroup $H$.  Explain why this diagram cannot be the diagram for a group.
\end{problem}

The problem that arises in Problem~\ref{prob:bad_quotient} is that if the same arrow types (i.e., those representing the same generator) leaving a coset do not point at elements in the same coset, attempting the quotient process will result in a diagram that cannot be a Cayley diagram for a group since we have more than one arrow of the same type leaving a vertex.  In Figure~\ref{fig:QuotientProcessBad}, we illustrate what goes wrong if all the arrows for a generator pointing out of a coset do not unanimously point to elements in the same coset.  In Figure~\ref{fig:QuotientProcessGood}, all the arrows point to elements in the same coset, and in this case, it appears that everything works out just fine.

%figure adapted from Matt Macauley.
\begin{figure}[!ht]
\centering
\subcaptionbox{\label{fig:QuotientProcessBad}}[.45\textwidth]{
\begin{tabular}{ccc}
\makecell{\begin{tikzpicture}[scale=1,shorten >= -3pt, shorten <= -3pt]
\draw (0,0) circle (.65); 
\draw (2,0) circle (.65);
\draw (1,1.8) circle (.65);
\node at (0,-.15) {\scriptsize $g_2H$};
\node at (2,-.15) {\scriptsize $g_3H$};
\node at (1,1.9) {\scriptsize $g_1H$};
\node (a1) at (.75,1.6) {\tiny $\bullet$};
\node (b1) at (.9,1.5) {\tiny $\bullet$};
\node (c1) at (1.1,1.5) {\tiny $\bullet$};
\node (d1) at (1.25,1.6) {\tiny $\bullet$};
\node (a2) at (-.1,.2) {\tiny $\bullet$};
\node (b2) at (.2,.1) {\tiny $\bullet$};
\node (c3) at (1.8,.1) {\tiny $\bullet$};
\node (d3) at (2.1,.2) {\tiny $\bullet$};
\draw[b] (a1) to [bend right=10] (a2);
\draw[b] (b1) to [bend right=5] (b2);
\draw[b] (c1) to [bend left=5] (c3);
\draw[b] (d1) to [bend left=10] (d3);
\end{tikzpicture}} & \makecell{$\longrightarrow$} &
\makecell{
\begin{tikzpicture}[scale=1]
\node (g1H) at (.8,1.5) [vert] {\scriptsize $g_1H$};
\node (g2H) at (0,0) [vert] {\scriptsize $g_2H$};
\node (g3H) at (1.6,0) [vert] {\scriptsize $g_3H$};
\draw[b] (g1H) to [bend right=5] (g2H);
\draw[b] (g1H) to [bend left=5] (g3H);
\end{tikzpicture}}
\end{tabular}
}
\subcaptionbox{\label{fig:QuotientProcessGood}}[.45\textwidth]{
\begin{tabular}{ccc}
\makecell{\begin{tikzpicture}[scale=1,,shorten >= -2pt, shorten <= -2pt]
\draw (1,0) circle (.65);
\draw (1,1.8) circle (.65);
\node at (1,-.2) {\scriptsize $g_2H$};
\node at (1,1.9) {\scriptsize $g_1H$};
\node (a1) at (.65,1.6) {\tiny $\bullet$};
\node (b1) at (.9,1.5) {\tiny $\bullet$};
\node (c1) at (1.1,1.5) {\tiny $\bullet$};
\node (d1) at (1.35,1.6) {\tiny $\bullet$};
\node (a2) at (.65,0) {\tiny $\bullet$};
\node (b2) at (.9,.2) {\tiny $\bullet$};
\node (c2) at (1.1,.2) {\tiny $\bullet$};
\node (d2) at (1.35,0) {\tiny $\bullet$};
\draw[b] (a1) to [bend right=5] (a2);
\draw[b] (b1) to [bend right=5] (b2);
\draw[b] (c1) to [bend left=5] (c2);
\draw[b] (d1) to [bend left=5] (d2);
\end{tikzpicture}} & \makecell{$\longrightarrow$} &
\makecell{
\begin{tikzpicture}[scale=1]
\node (g1H) at (0,1.5) [vert] {\scriptsize $g_1H$};
\node (g2H) at (0,0) [vert] {\scriptsize $g_2H$};
\draw[b] (g1H) to (g2H);
\end{tikzpicture}}
\end{tabular}
}
\caption{In the left subfigure, blue arrows go from elements of the left coset $g_1H$ to elements of \emph{multiple} left cosets, which results in ambiguous blue arrows in the collapsed diagram. This implies that left coset multiplication is not well-defined in this case. In the right subfigure, blue arrows go from elements of the left coset $g_1H$ to elements inside a \emph{unique} left coset, which does not result in any ambiguity.}
\label{fig:QuotientProcess}
\end{figure}

\begin{problem}
In Problem~\ref{prob:normal_in_D3}, we learned that the subgroup $K=\langle r\rangle$ is normal in $D_3$ since the left cosets are equal to the right cosets.  Note that this follows immediately from Theorem~\ref{thm:index2} since $[D_3:K]=2$.  Draw the diagram that results from performing the quotient process to $D_3$ using the subgroup $K$.  Does the resulting diagram represent a group?  If so, what group is it isomorphic to?
\end{problem}

Now, suppose $G$ is an arbitrary group and let $H\leq G$. Consider the set of left cosets of $H$.  We define
\[
(aH)(bH):=(ab)H.
\]
The natural question to ask is whether this operation is well-defined.  That is, does the result of multiplying two left cosets depend on our choice of representatives?  More specifically, suppose $c\in aH$ and $d\in bH$.  Then $cH=aH$ and $dH=bH$.  According to the operation defined above, $(cH)(dH)=cdH$.  It better be the case that $cdH=abH$, otherwise the operation is not well-defined.

\begin{problem}
Let $H=\langle s\rangle \leq D_3$.  Find specific examples of $a,b,c,d\in D_3$ such that
\[
(aH)(bH)\neq (cH)(dH)
\]
even though $aH=cH$ and $bH=dH$.
\end{problem}

\begin{theorem}
Let $G$ be a group and let $H\leq G$.  Then left coset multiplication (as defined above) is well-defined if and only if $H\trianglelefteq G$.
\end{theorem}

\begin{theorem}\label{thm:quotient_grp}
Let $G$ be a group and let $H\trianglelefteq G$.  Then the set of left cosets of $H$ in $G$ forms a group under left coset multiplication.
\end{theorem}

The group from Theorem~\ref{thm:quotient_grp} is denoted by $G/H$, read ``$G$ mod $H$", and is referred to as the \textbf{quotient group} (or \textbf{factor group}) of $G$ by $H$.  If $G$ is a finite group, then $G/H$ is exactly the group that arises from ``gluing together" the colored blocks in a checkerboard-patterned group table.  It's also the group that we get after applying the quotient process to the Cayley diagram.  It's important to point out once more that this only works properly if $H$ is a normal subgroup.

Recall that Theorem~\ref{thm:AbelianImpliesNormal} tells us that if $G$ is abelian, then every subgroup is normal.  This implies that when $G$ is abelian, $G/H$ is a well-defined group for every subgroup $H$ of $G$.  However, it is not necessary for $G$ to be abelian in order for $G/H$ to be a well-defined group.  The quotient group $Q_8/\langle -1\rangle$ is an example where this happens. 

The next theorem tells us how to compute the order of a quotient group.

\begin{theorem}
Let $G$ be a group and let $H\trianglelefteq G$.  Then $|G/H|=[G:H]$.  In particular, if $G$ is finite, then $|G/H|=|G|/|H|$.
\end{theorem}

It's important to point out that the order of a quotient group might be finite even if $G$ has infinite order.

\begin{problem}
Consider the group $(\mathbb{Z},+)$. Since $\mathbb{Z}$ is abelian, every subgroup is normal. For example, $4\mathbb{Z} \trianglelefteq \mathbb{Z}$, which implies that $\mathbb{Z}/4\mathbb{Z}$ is a well-defined quotient group.  Moreover, both $\mathbb{Z}$ and $4\mathbb{Z}$ have infinite order. What is $|\mathbb{Z}/4\mathbb{Z}|$ equal to?  Can you determine what well-known group $\mathbb{Z}/4\mathbb{Z}$ is isomorphic to?
\end{problem}

Suppose $G$ is a group and $H\trianglelefteq G$, so that $G/H$ is a group. Recall that the elements of the group $G/H$ are the left cosets of $H$, which are of the form $aH$ where $a\in G$. The operation of the group is defined via
\[
(aH)(bH) = abH.
\]
Moreover, the identity in $G/H$ is $eH = H$ since $(aH)(eH) = aH$. By Corollary~\ref{cor:order_smallest_exponent} $|aH| = k$ if and only $(aH)^k = H$ and $k$ is the smallest such positive exponent with this property. But notice that $(aH)^k = a^kH$. So, in order to compute the order of $aH$, we need to find the smallest positive exponent $k$ such that $a^kH = H$, but $a^kH = H$ exactly when $a^k$ is in $H$. The upshot is that to find the order of $aH$ in $G/H$, we need the smallest positive $k$ such that $a^k$ is in $H$.

\begin{problem}%Borrowing from Fraleigh Exercises 9--15 in Section 14.
Find the order of the given element in the quotient group. You may assume that we are taking the quotient by a normal subgroup. 
\begin{enumerate}[label=\rm{(\alph*)}]
\item $s\langle r\rangle \in D_4/\langle r\rangle$
\item $j\langle -1\rangle \in Q_8/\langle -1\rangle$
\item $5+\langle 4\rangle \in \mathbb{Z}_{12}/\langle 4\rangle$
\item $(2,1)+\langle (1,1)\rangle \in (\mathbb{Z}_3\times \mathbb{Z}_6)/\langle (1,1)\rangle$
\item $(1,3)+\langle (0,2)\rangle\in (\mathbb{Z}_4\times \mathbb{Z}_8)/\langle (0,2)\rangle$
\end{enumerate}
\end{problem}

\begin{problem}
For each quotient group below, describe the group.  If possible, state what group each is isomorphic to.  You may assume that we are taking the quotient by a normal subgroup. 
\begin{enumerate}[label=\rm{(\alph*)}]
\item $Q_8/\langle -1\rangle$
\item $Q_8/\langle i\rangle$
\item $\mathbb{Z}_4/\langle 2\rangle$
\item $V_4/\langle h\rangle$
\item $A_4/\langle (1,2)(3,4),(1,3)(2,4)\rangle$
\item $(\mathbb{Z}_2\times \mathbb{Z}_2)/\langle (1,1)\rangle$
\item $\mathbb{Z}/4\mathbb{Z}$
\item $S_4/A_4$
\item $(\mathbb{Z}_4\times \mathbb{Z}_2)/(\{0\}\times \mathbb{Z}_2)$
\end{enumerate}
\end{problem}

\begin{problem}
Compute the order of every element in the quotient group $(\mathbb{Z}_2\times \mathbb{Z}_4)/\langle (0,2)\rangle$. What well-known group is $(\mathbb{Z}_2\times \mathbb{Z}_4)/\langle (0,2)\rangle$ isomorphic to?
\end{problem}

\begin{theorem}
Let $G$ be a group.  Then
\begin{enumerate}[label=\rm{(\alph*)}]
\item $G/\{e\}\cong G$
\item $G/G\cong \{e\}$
\end{enumerate}
\end{theorem}

\begin{theorem}
For all $n\in \mathbb{N}$, we have the following.
\begin{enumerate}[label=\rm{(\alph*)}]
\item $S_n/A_n\cong \mathbb{Z}_2$ (for $n\geq 3$)
\item $\mathbb{Z}/n\mathbb{Z}\cong \mathbb{Z}_n$
\item $\mathbb{R}/n\mathbb{R}\cong \{e\}$
\end{enumerate}
\end{theorem}

\begin{theorem}
Let $G$ be a group and let $H\trianglelefteq G$.  If $G$ is abelian, then so is $G/H$.
\end{theorem}

\begin{problem}
Show that the converse of the previous theorem is not true by providing a specific counterexample.
\end{problem}

\begin{problem}
Consider the quotient group $(\mathbb{Z}_4\times \mathbb{Z}_6)/\langle (0,1)\rangle$.
\begin{enumerate}[label=\rm{(\alph*)}]
\item What is the order of $(\mathbb{Z}_4\times \mathbb{Z}_6)/\langle (0,1)\rangle$?
\item Is the group abelian?  Why?
\item Write down all the elements of $(\mathbb{Z}_4\times \mathbb{Z}_6)/\langle (0,1)\rangle$.
\item Does one of the elements generate the group?
\item What well-known group is $(\mathbb{Z}_4\times \mathbb{Z}_6)/\langle (0,1)\rangle$ isomorphic to?
\end{enumerate}
\end{problem}

\begin{theorem}
Let $G$ be a group and let $H\trianglelefteq G$.  If $G$ is cyclic, then so is $G/H$.
\end{theorem}

\begin{problem}
Show that the converse of the previous theorem is not true by providing a specific counterexample.
\end{problem}

Here are few additional exercises.  These ones are a bit tougher.

\begin{problem}
For each quotient group below, describe the group.  If possible, state what group each is isomorphic to.  You may assume that we are taking the quotient by a normal subgroup. 
\begin{enumerate}[label=\rm{(\alph*)}]
\item $(\mathbb{Z}_4\times \mathbb{Z}_6)/\langle (0,2)\rangle$
\item $(\mathbb{Z}\times \mathbb{Z})/\langle (1,1)\rangle$
\item $\mathbb{Q}/\langle 1\rangle$ (the operation on $\mathbb{Q}$ is addition)
\end{enumerate}
\end{problem}

\end{section}
\chapter{Homomorphisms and the Isomorphism Theorems}
\label{chapter:homomorphisms}
\thispagestyle{empty}

\begin{section}{Homomorphisms}
Let $G_1$ and $G_2$ be groups. Recall that $\phi:G_1\to G_2$ is an isomorphism if and only if $\phi$
\begin{enumerate}[label=\rm{(\alph*)}]
\item is one-to-one, 
\item is onto, and
\item satisfies the homomorphic property.
\end{enumerate}
We say that $G_1$ is isomorphic to $G_2$ and write $G_1\cong G_2$ if such a $\phi$ exists. Loosely speaking, two groups are isomorphic if they have the ``same structure."  What if we drop the one-to-one and onto requirement?

\begin{definition}
Let $(G_1,*)$ and $(G_2,\odot)$ be groups. A function $\phi:G_1\to G_2$ is a \textbf{homomorphism} if and only if $\phi$ satisfies the homomorphic property:
\[
\phi(x*y)=\phi(x)\odot\phi(y)
\]
for all $x,y\in G_1$. At the risk of introducing ambiguity, we will usually omit making explicit reference to the binary operations and write the homomorphic property as
\[
\phi(xy)=\phi(x)\phi(y).
\]
\end{definition}

Group homomorphisms are analogous to linear transformations on vector spaces that one encounters in linear algebra.

Figure~\ref{fig:isoGroupTables2} captures a visual representation of the homomorphic property.  We encountered this same representation in Figure~\ref{fig:isoGroupTables}. If $\phi(x)=x'$, $\phi(y)=y'$, and $\phi(z)=z'$ while $z'=x'\odot y'$, then the only way $G_2$ may respect the structure of $G_1$ is for
\[
\phi(x*y)=\phi(z)=z'=x'\odot y'=\phi(x)\odot \phi(y).
\]

\begin{figure}
\begin{center}
\begin{tabu}{c|[2pt]ccc|c|c}
$*$                & & & & \cellcolor{green}$y$  & \\ \tabucline[2pt]{-}
                   & & & &                       & \\ \hline
\cellcolor{red}$x$ & & & & \cellcolor{orange}$z$ & \\ \hline
                   & & & &                      & \\
                   & & & &                      &
\end{tabu}
\hspace{1cm}
$\longrightarrow$
\hspace{1cm}
\begin{tabu}{c|[2pt]ccc|c|c}
$\odot$                & & & & \cellcolor{green}$y'$  & \\ \tabucline[2pt]{-}
                   & & & &                       & \\ \hline
\cellcolor{red}$x'$ & & & & \cellcolor{orange}$z'$ & \\ \hline
                   & & & &                      & \\
                   & & & &                      &
\end{tabu}
\end{center}
\caption{}\label{fig:isoGroupTables2}
\end{figure}

\begin{problem}\label{prob:homomorphism}
Define $\phi:\mathbb{Z}_3\to D_3$ via $\phi(k)=r^k$. Prove that $\phi$ is a homomorphism and then determine whether $\phi$ is one-to-one or onto. Also, try to draw a picture of the homomorphism in terms of Cayley diagrams.
\end{problem}

\begin{problem}
Let $G$ and $H$ be groups. Prove that the function $\phi:G\times H\to G$ given by $\phi(g,h)=g$ is a homomorphism. This function is an example of a \textbf{projection map}.
\end{problem}

There is always at least one homomorphism between two groups.

\begin{theorem}\label{thm:trivial_homomorphism}
Let $G_1$ and $G_2$ be groups. Define $\phi:G_1\to G_2$ via $\phi(g)=e_2$ (where $e_2$ is the identity of $G_2$).  Then $\phi$ is a homomorphism. This function is often referred to as the \textbf{trivial homomorphism} or the \textbf{$0$-map}.
\end{theorem}

Back in Section~\ref{sec:revisiting_isomorphisms}, we encountered several theorems about isomorphisms.  However, at the end of that section we remarked that some of those theorems did not require that the function be one-to-one and onto.  We collect those results here for convenience.

\begin{theorem}
Let $G_1$ and $G_2$ be groups and suppose $\phi:G_1\to G_2$ is a homomorphism.
\begin{enumerate}[label=\rm{(\alph*)}]
\item If $e_1$ and $e_2$ are the identity elements of $G_1$ and $G_2$, respectively, then $\phi(e_1)=e_2$.
\item For all $g\in G_1$, we have $\phi(g^{-1})=[\phi(g)]^{-1}$.
\item If $H\leq G_1$, then $\phi(H)\leq G_2$, where
\[
\phi(H):=\{y\in G_2\mid \text{there exists } h\in H\text{ such that }\phi(h)=y\}. 
\]
Note that $\phi(H)$ is called the \textbf{image} of $H$. A special case is when $H=G_1$. Notice that $\phi$ is onto exactly when $\phi(G_1)=G_2$.
\end{enumerate}
\end{theorem}

The following theorem is a consequence of Lagrange's Theorem.

\begin{theorem}
Let $G_1$ and $G_2$ be groups such that $G_2$ is finite and let $H\leq G_1$. If $\phi:G_1\to G_2$ is a homomorphism, then $|\phi(H)|$ divides $|G_2|$.
\end{theorem}

The next theorem tells us that under a homomorphism, the order of the image of an element must divide the order of the pre-image of that element.

\begin{theorem}\label{thm:orderImage}
Let $G_1$ and $G_2$ be groups and suppose $\phi:G_1\to G_2$ is a homomorphism. If $g\in G_1$ such that $|g|$ is finite, then $|\phi(g)|$ divides $|g|$.
\end{theorem}

Every homomorphism has an important subset of the domain associated with it.

\begin{definition}
Let $G_1$ and $G_2$ be groups and suppose $\phi:G_1\to G_2$ is a homomorphism.  The \textbf{kernel} of $\phi$ is defined via
\[
\ker(\phi):=\{g\in G_1\mid \phi(g)=e_2\}.
\]
\end{definition}

The kernel of a homomorphism is analogous to the null space of a linear transformation of vector spaces.  

\begin{problem}
Identify the kernel and image for the homomorphism given in Problem~\ref{prob:homomorphism}.
\end{problem}

\begin{problem}
What is the kernel of a trivial homomorphism (see Theorem~\ref{thm:trivial_homomorphism}).
\end{problem}

\begin{theorem}\label{thm:kernel_normal}
Let $G_1$ and $G_2$ be groups and suppose $\phi:G_1\to G_2$ is a homomorphism. Then $\ker(\phi)\trianglelefteq G_1$.
\end{theorem}

\begin{theorem}\label{thm:canonical_projection}
Let $G$ be a group and let $H\trianglelefteq G$.  Then the map $\gamma:G\to G/H$ given by $\gamma(g)=gH$ is a homomorphism with $\ker(\gamma)=H$. This map is called the \textbf{canonical projection map}.
\end{theorem}

The upshot of Theorems~\ref{thm:kernel_normal} and \ref{thm:canonical_projection} is that kernels of homomorphisms are always normal and every normal subgroup is the kernel of some homomorphism. It turns out that the kernel can tell us whether $\phi$ is one-to-one.

\begin{theorem}
Let $G_1$ and $G_2$ be groups and suppose $\phi:G_1\to G_2$ is a homomorphism. Then $\phi$ is one-to-one if and only if $\ker(\phi)=\{e_1\}$, where $e_1$ is the identity in $G_1$.
\end{theorem}

\begin{remark}
Let $G_1$ and $G_2$ be groups and suppose $\phi:G_1\to G_2$ is a homomorphism. Given a generating set for $G_1$, the homomorphism $\phi$ is uniquely determined by its action on the generating set for $G_1$.  In particular, if you have a word for a group element written in terms of the generators, just apply the homomorphic property to the word to find the image of the corresponding group element.
\end{remark}

\begin{problem}\label{prob:Q8toV4}
Suppose $\phi: Q_8\to V_{4}$ is a group homomorphism satisfying $\phi(i)=h$ and $\phi(j)=v$.
\begin{enumerate}[label=\rm{(\alph*)}]
\item Find $\phi(1)$, $\phi(-1)$, $\phi(k)$, $\phi(-i)$, $\phi(-j)$, and $\phi(-k)$.
\item Find $\ker(\phi)$.
\item What well-known group is $Q_8/\ker(\phi)$ isomorphic to?
\end{enumerate}
\end{problem}

\begin{problem}
Find a non-trivial homomorphism from $\mathbb{Z}_{10}$ to $\mathbb{Z}_6$.
\end{problem}

\begin{problem}
Find all non-trivial homomorphisms from $\mathbb{Z}_3$ to $\mathbb{Z}_6$.
\end{problem}

\begin{problem}
Prove that the only homomorphism from $D_3$ to $\mathbb{Z}_3$ is the trivial homomorphism.
\end{problem}

\begin{problem}
Let $F$ be the set of all functions from $\mathbb{R}$ to $\mathbb{R}$ and let $D$ be the subset of differentiable functions on $\mathbb{R}$.  It turns out that $F$ is a group under addition of functions and $D$ is a subgroup of $F$ (you do not need to prove this). Define $\phi:D\to F$ via $\phi(f)=f'$ (where $f'$ is the derivative of $f$). Prove that $\phi$ is a homomorphism.  You may recall facts from calculus without proving them. Is $\phi$ one-to-one? Onto? 
\end{problem}

\end{section}

\begin{section}{The Isomorphism Theorems}

The next theorem is arguably the crowning achievement of the course.

\begin{theorem}[The First Isomorphism Theorem]
Let $G_1$ and $G_2$ be groups and suppose $\phi:G_1\to G_2$ is a homomorphism. Then
\[
G_1/\ker(\phi)\cong \phi(G_1).
\]
If $\phi$ is onto, then
\[
G_1/\ker(\phi)\cong G_2.
\]
\end{theorem}

\begin{problem}
Let $\phi:Q_8\to V_4$ be the homomorphism described in Problem~\ref{prob:Q8toV4}. Use the First Isomorphism Theorem to prove that $Q_8/\langle-1\rangle\cong V_4$.
\end{problem}

\begin{problem}
Define $\phi:S_n\to \mathbb{Z}_2$ via
\[
\phi(\sigma)=\begin{cases}
0, & \sigma \text{ even}\\
1, & \sigma \text{ odd}.
\end{cases}
\]
Use the First Isomorphism Theorem to prove that $S_n/A_n\cong \mathbb{Z}_2$.
\end{problem}

\begin{problem}
Use the First Isomorphism Theorem to prove that $\mathbb{Z}/6\mathbb{Z}\cong \mathbb{Z}_6$.  Attempt to draw a picture of this using Cayley diagrams.
\end{problem}

\begin{problem}
Use the First Isomorphism Theorem to prove that $(\mathbb{Z}_4\times \mathbb{Z}_2)/(\{0\}\times \mathbb{Z}_2)\cong \mathbb{Z}_4$.
\end{problem}

The next theorem is a generalization of Theorem~\ref{thm:orderImage} and follows from the First Isomorphism Theorem together with Lagrange's Theorem.

\begin{theorem}
Let $G_1$ and $G_2$ be groups and suppose $\phi:G_1\to G_2$ is a homomorphism. If $G_1$ is finite, then $|\phi(G_1)|$ divides $|G_1|$.
\end{theorem}

We finish the chapter by listing a few of the remaining isomorphism theorems.%, but we will not prove these in this course.

\begin{theorem}[The Second Isomorphism Theorem]
Let $G$ be a group with $H\leq G$ and $N\trianglelefteq G$.  Then
\begin{enumerate}[label=\rm{(\alph*)}]
\item $HN:=\{hn\mid h\in H, n\in N\}\leq G$;
\item $H\cap N\trianglelefteq H$;
\item $\displaystyle H/(H\cap N)\cong HN/N$.
\end{enumerate}
\end{theorem}

\begin{theorem}[The Third Isomorphism Theorem]
Let $G$ be a group with $H,K\trianglelefteq G$ and $K\leq H$.  Then $H/K\trianglelefteq G/K$ and
\[
G/H\cong (G/K)/(H/K).
\]
\end{theorem}

The last isomorphism theorem is sometimes called the \emph{Lattice Isomorphism Theorem} or the \emph{Correspondence Theorem}.

\begin{theorem}[The Fourth Isomorphism Theorem]

Let $G$ be a group with $N\trianglelefteq G$. Then there is a bijection from the set of subgroups of $G$ that contain $N$ onto the set of subgroups of $G/N$. In particular, every subgroup $G$ is of the form $H/N$ for some subgroup $H$ of $G$ containing $N$ (namely, its pre-image in $G$ under the canonical projection homomorphism from $G$ to $G/N$.) This bijection has the following properties: for all $H,K \subseteq G$ with  $N\subseteq H$ and $N\subseteq K$, we have
\begin{enumerate}[label=\rm{(\alph*)}]
\item $H\subset K$ if and only if $H/N \subset K/N$
\item If $H\subset K$, then $|K:H|=|K/N:H/N|$
\item $\langle H,K\rangle/N=\langle H/N,K/N\rangle$
\item $(H\cap K)/N=H/N \cap K/N$
\item $H\trianglelefteq G$ if and only if $H/N\trianglelefteq G/N$.
\end{enumerate}
\end{theorem}

\end{section}
\chapter{An Introduction to Rings}
\label{chapter:rings}

\begin{section}{Definitions and Examples}

Recall that a group is a set together with a single binary operation, which together satisfy a few modest properties. Loosely speaking, a ring is a set together with two binary operations (called addition and multiplication) that are related via a distributive property.

\begin{definition}
A \textbf{ring} $R$ is a set together with two binary operations $+$ and $\cdot$ (called \textbf{addition} and \textbf{multiplication}, respectively) satisfying the following:
\begin{enumerate}
\item[(i)] $(R,+)$ is an abelian group.
\item[(ii)] $\cdot$ is associative: $(a\cdot b)\cdot c=a\cdot (b\cdot c)$ for all $a,b,c\in R$.
\item[(iii)] The \textbf{distributive property} holds: $a\cdot (b+c)=(a\cdot b)+(a\cdot c)$ and $(a+b)\cdot c = (a\cdot c)+(b\cdot c)$ for all $a,b,c\in R$.
\end{enumerate}
\end{definition}

\begin{remark}
We make a couple comments about notation.
\begin{enumerate}[label=\rm{(\alph*)}]
\item We often write $ab$ in place of $a\cdot b$.
\item The additive inverse of the ring element $a\in R$ is denoted $-a$.
\end{enumerate}
\end{remark}

\begin{theorem}
If $R$ is a ring, then for all $a,b\in R$:
\begin{enumerate}[label=\rm{(\alph*)}]
\item $0a=a0=0$
\item $(-a)b=a(-b)=-(ab)$
\item $(-a)(-b)=ab$
\end{enumerate}
\end{theorem}

\begin{definition}
A ring $R$ is called \textbf{commutative} if multiplication is commutative.
\end{definition}

\begin{definition}
A ring $R$ is said to have an \textbf{identity} (or called a \textbf{ring with  1}) if there is an element $1\in R$ such that $1a=a 1=a$ for all $a\in R$.
\end{definition}

\begin{problem}
Justify that $\mathbb{Z}$ is a commutative ring with 1 under the usual operations of addition and multiplication. Which elements have multiplicative inverses in $\mathbb{Z}$?
\end{problem}

\begin{problem}
Justify that $\mathbb{Z}_n$ is a commutative ring with 1 under addition and multiplication mod $n$.
\end{problem}

\begin{problem}\label{prob:Z10Ring}
Consider the set $\mathbb{Z}_{10}=\{0,1,2,3,4,5,6,7,8,9\}$. Which elements have multiplicative inverses in $\mathbb{Z}_{10}$?
\end{problem}

\begin{problem}
For each of the following, find a positive integer $n$ such that the ring $\mathbb{Z}_n$ does not have the stated property.
\begin{enumerate}[label=\rm{(\alph*)}]
\item $a^2=a$ implies $a=0$ or $a=1$.
\item $ab=0$ implies $a=0$ or $b=0$.
\item $ab=ac$ and $a\neq 0$ imply $b=c$.
\end{enumerate}
\end{problem}

\begin{theorem}
If $R$ is a ring with 1, then the multiplicative identity is unique and $-a=(-1)a$.
\end{theorem}

\begin{problem}
Requiring $(R,+)$ to be a group is fairly natural, but why require $(R,+)$ to be abelian?  Suppose $R$ has a 1.  Compute $(1+1)(a+b)$ in two different ways.
\end{problem}

\begin{definition}
A ring $R$ with 1 (with $1\neq 0$) is called a \textbf{division ring} if every nonzero element in $R$ has a multiplicative inverse: if $a\in R\setminus\{0\}$, then there exists $b\in R$ such that $ab=ba=1$.
\end{definition}

\begin{definition}
A commutative division ring is called a \textbf{field}.
\end{definition}

\begin{definition}
A nonzero element $a$ in a ring $R$ is called a \textbf{zero divisor} if there is a nonzero element $b\in R$ such that either $ab=0$ or $ba=0$.
\end{definition}

\begin{problem}
Are there any zero divisors in $\mathbb{Z}_{10}$?  If so, find all of them.
\end{problem}

\begin{problem}
Are there any zero divisors in $\mathbb{Z}_5$?  If so, find all of them.
\end{problem}

\begin{problem}
Provide an example of a ring $R$ and elements $a,b\in R$ such that $ax=b$ has more than one solution.  How does this compare with groups?
\end{problem}

\begin{theorem}[Cancellation Law]\label{thm:RingCancellation}
Assume $a,b,c\in R$ such that $a$ is not a zero divisor.  If $ab=ac$, then either $a=0$ or $b=c$.
\end{theorem}

\begin{definition}
Assume $R$ is a ring with $1$ with $1\neq 0$. An element $u\in R$ is called a \textbf{unit} in $R$ if $u$ has a multiplicative inverse (i.e., there exists $v\in R$ such that $uv=vu=1$).  The set of units in $R$ is denoted $U(R)$.
\end{definition}

\begin{problem}
Consider the ring $\mathbb{Z}_{20}$.
\begin{enumerate}[label=\rm{(\alph*)}]
\item Find $U(\mathbb{Z}_{20})$.
\item Find the zero divisors of $\mathbb{Z}_{20}$.
\item Any observations?
\end{enumerate}
\end{problem}

\begin{theorem}
If $U(R)\neq\emptyset$, then $U(R)$ forms a group under multiplication.
\end{theorem}

\begin{remark}
We make a few observations.
\begin{enumerate}[label=\rm{(\alph*)}]
\item A field is a commutative ring $F$ with identity $1\neq 0$ in which every nonzero element is a unit, i.e., $U(F)=F\setminus\{0\}$.
\item Zero divisors can never be units.
\item Fields never have zero divisors.
\end{enumerate}
\end{remark}

\begin{definition}
A commutative ring with identity $1\neq 0$ is called an \textbf{integral domain} if it has no zero divisors.
\end{definition}

\begin{remark}
The Cancellation Law (Theorem~\ref{thm:RingCancellation}) holds in integral domains for any three elements.
\end{remark}

\begin{theorem}
Any finite integral domain is a field.
\end{theorem}

%\begin{proof}
%For any nonzero $a\in R$, define $f_a:R\to R$ via $f_a(x)=ax$.  If $R$ is an integral domain, the Cancellation Law forces $f_a$ to be injective.  If $R$ is finite, then $f_a$ is also surjective.  In this case, there exists $b\in R$ such that $ab=1$.
%\end{proof}

\begin{example}
Here are a few examples.  Details left as an exercise.
\begin{enumerate}[label=\rm{(\alph*)}]
\item \textbf{Zero Ring}: If $R=\{0\}$, we can turn $R$ into a ring in the obvious way.  The zero ring is a finite commutative ring with 1.  It is the only ring where the additive and multiplicative identities are equal.  The zero ring is not a division ring, not a field, and not an integral domain.
\item \textbf{Trivial Ring:} Given any abelian group $R$, we can turn $R$ into a ring by defining multiplication via $ab=0$ for all $a,b\in R$. Trivial rings are commutative rings in which every nonzero element is a zero divisor.  Hence a trivial ring is not a division ring, not a field, and not a integral domain.
\item 
%The integers $\mathbb{Z}$ form a ring under the usual operations of addition and multiplication.  
The integers form an integral domain, but $\mathbb{Z}$ is not a division ring, and hence not a field.
\item The rational numbers $\mathbb{Q}$, the real numbers $\mathbb{R}$, and the complex numbers $\mathbb{C}$ are fields under the usual operations of addition and multiplication.
\item 
%For $n\geq 1$, the set $\mathbb{Z}_n$ is a commutative ring with 1 under the operations of addition and multiplication mod $n$. 
The group of units $U(\mathbb{Z}_n)$ is the set of elements in $\mathbb{Z}_n$ that are relatively prime to $n$.  All other nonzero elements are zero divisors.  It turns out that $\mathbb{Z}_n$ forms a finite field if and only if $n$ is prime.
\item The set of even integers $2\mathbb{Z}$ forms a commutative ring under the usual operations of addition and multiplication.  However, $2\mathbb{Z}$ does not have a 1, and hence cannot be a division ring nor a field nor an integral domain.
\item \textbf{Polynomial Ring:} Fix a commutative ring $R$.  Let $R[x]$ denote the set of polynomials in the variable $x$ with coefficients in $R$.  Then $R[x]$ is a commutative ring. Moreover, $R[x]$ is a ring with 1 if and only if $R$ is a ring with 1. The units of $R[x]$ are exactly the units of $R$ (if there are any). So, $R[x]$ is never a division ring nor a field.  However, if $R$ is an integral domain, then so is $R[x]$. 
\item \textbf{Matrix Ring:} Fix a ring $R$ and let $n$ be a positive integer.  Let $M_n(R)$ be the set of $n\times n$ matrices with entries from $R$.  Then $M_n(R)$ forms a ring under ordinary matrix addition and multiplication.  If $R$ is nontrivial and $n\geq 2$, then $M_n(R)$ always has zero divisors and $M_n(R)$ is not commutative even if $R$ is.  If $R$ has a 1, then the matrix with 1's down the diagonal and 0's elsewhere is the multiplicative identity in $M_n(R)$.  In this case, the group of units is the set of invertible $n\times n$ matrices, denoted $GL_n(R)$ and called the \textbf{general linear group of degree $n$ over $R$}.
\item \textbf{Quadratic Field:} Define $\mathbb{Q}(\sqrt{2})=\{a+b\sqrt{2}\mid a,b\in\mathbb{Q}\}$.  It turns out that $\mathbb{Q}(\sqrt{2})$ is a field.  In fact, we can replace 2 with any rational number that is not a perfect square in $\mathbb{Q}$.
\item \textbf{Hamilton Quaternions:} Define $\mathbb{H}=\{a+bi+cj+dk\mid a,b,c,d\in\mathbb{R}, i,j,k\in Q_8\}$  Then $\mathbb{H}$ forms a ring, where addition is definite componentwise in $i$, $j$, and $k$ and multiplication is defined by expanding products and the simplifying using the relations of $Q_8$.  It turns out that $\mathbb{H}$ is a non-commutative ring with 1.
\end{enumerate}
\end{example}

\begin{problem}
Find an example of a ring $R$ and an element $a\in R\setminus\{0\}$ such that $a$ is neither a zero divisor nor a unit.
\end{problem}

\begin{definition}
A \textbf{subring} of a ring $R$ is a subgroup of $R$ under addition that is also closed under multiplication.
\end{definition}

\begin{remark}
The property ``is a subring" is clearly transitive. To show that a subset $S$ of a ring $R$ is a subring, it suffices to show that $S\neq \emptyset$, $S$ is closed under subtraction, and $S$ is closed under multiplication.
\end{remark}

\begin{example}
Here are a few quick examples.
\begin{enumerate}[label=\rm{(\alph*)}]
\item $\mathbb{Z}$ is a subring of $\mathbb{Q}$, which is a subring of $\mathbb{R}$, which in turn is a subring of $\mathbb{C}$.
\item $2\mathbb{Z}$ is a subring of $\mathbb{Z}$.
\item The set $\mathbb{Z}(\sqrt{2})=\{a+b\sqrt{2}\mid a,b\in\mathbb{Z}\}$ is a subring of $\mathbb{Q}(\sqrt{2})$.
\item The ring $R$ is a subring of $R[x]$ if we identify $R$ with set of constant functions.
\item The set of polynomials with zero constant term in $R[x]$ is a subring of $R[x]$.
\item $\mathbb{Z}[x]$ is a subring of $\mathbb{Q}[x]$.
\item $\mathbb{Z}_n$ is \emph{not} a subring of $\mathbb{Z}$ as the operations are different.
\end{enumerate} 
\end{example}

\begin{problem}
Consider the ring $\mathbb{Z}_{10}$ from Problem~\ref{prob:Z10Ring}. Let $S=\{0,2,4,6,8\}$.
\begin{enumerate}[label=\rm{(\alph*)}]
\item Argue that $S$ is a subring of $\mathbb{Z}_{10}$.
\item Is $S$ a ring with 1?  If so, find the multiplicative identity.  If not, explain why.
\item Is $S$ a field? Justify your answer.
\end{enumerate}
\end{problem}

\begin{problem}
Suppose $R$ is a ring and let $a\in R$.  Define $S=\{x\in R\mid ax=0\}$.  Prove that $S$ is a subring of $R$.
\end{problem}

\begin{problem}
Consider the ring $\mathbb{Z}$.  It turns out that $2\mathbb{Z}$ and $3\mathbb{Z}$ are subrings (but you don't need to prove this).  Determine whether $2\mathbb{Z}\cup 3\mathbb{Z}$ is a subring of $\mathbb{Z}$.  Justify your answer.
\end{problem}

\end{section}

\begin{section}{Ring Homomorphisms}

\begin{definition}
Let $R$ and $S$ be rings.  A \textbf{ring homomorphism} is a map $\phi:R\to S$ satisfying
\begin{enumerate}[label=\rm{(\alph*)}]
\item $\phi(a+b)=\phi(a)+\phi(b)$
\item $\phi(ab)=\phi(a)\phi(b)$
\end{enumerate}
for all $a,b\in R$. The \textbf{kernel} of $\phi$ is defined via $\ker(\phi)=\{a\in R\mid \phi(a)=0\}$.  If $\phi$ is a bijection, then $\phi$ is called an \textbf{isomorphism}, in which case, we say that $R$ and $S$ are \textbf{isomorphic rings} and write $R\cong S$.
\end{definition}

\begin{example}
\ 
\begin{enumerate}[label=\rm{(\alph*)}]
\item For $n\in \mathbb{Z}$, define $\phi_n:\mathbb{Z}\to \mathbb{Z}$ via $\phi_n(x)=nx$.  We see that $\phi_n(x+y)=n(x+y)=nx+ny=\phi_n(x)+\phi_n(y)$.  However, $\phi_n(xy)=n(xy)$ while $\phi_n(x)\phi_n(y)=(nx)(ny)=n^2xy$.  It follows that $\phi_n$ is a ring homomorphism exactly when $n\in\{0,1\}$.
\item Define $\phi:\mathbb{Q}[x]\to \mathbb{Q}$ via $\phi(p(x))=p(0)$ (called \textbf{evaluation at 0}).  It turns out that $\phi$ is a ring homomorphism, where $\ker(\phi)$ is the set of polynomials with 0 constant term.
\end{enumerate}
\end{example}

\begin{problem}
For each of the following, determine whether the given function is a ring homomorphism.  Justify your answers.
\begin{enumerate}[label=\rm{(\alph*)}]
\item Define $\phi:\mathbb{Z}_4\to \mathbb{Z}_{12}$ via $\phi(x)=3x$.
\item Define $\phi:\mathbb{Z}_{10}\to \mathbb{Z}_{10}$ via $\phi(x)=5x$.
\item Let $\displaystyle S=\left\{\begin{pmatrix}a & b\\ -b & a\end{pmatrix}\mid a, b\in \mathbb{R}\right\}$.  Define $\phi:\mathbb{C}\to S$ via $\displaystyle \phi(a+ib)=\begin{pmatrix}a & b\\ -b & a\end{pmatrix}$.
\item Let $\displaystyle T=\left\{\begin{pmatrix}a & b\\ 0 & c\end{pmatrix}\mid a, b\in \mathbb{Z}\right\}$. Define $\phi:T\to \mathbb{Z}$ via $\displaystyle \phi\left(\begin{pmatrix}a & b\\ 0 & c\end{pmatrix}\right)=a$.
\end{enumerate}
\end{problem}

\begin{theorem}\label{thm:kernelsubring}
Let $\phi:R\to S$ be a ring homomorphism.
\begin{enumerate}[label=\rm{(\alph*)}]
\item $\phi(R)$ is a subring of $S$.
\item $\ker(\phi)$ is a subring of $R$.
\end{enumerate}
\end{theorem}

\begin{problem}
Suppose $\phi:R\to S$ is a ring homomorphism such that $R$ is a ring with 1, call it $1_R$.  Prove that $\phi(1_R)$ is the multiplicative identity in $\phi(R)$ (which is a subring of $S$).  Can you think of an example of a ring homomorphism where $S$ has a multiplicative identity that is not equal to $\phi(1_R)$?
\end{problem}

Theorem~\ref{thm:kernelsubring}(b) states that the kernel of a ring homomorphism is a subring. This is analogous to the kernel of a group homomorphism being a subgroup. However, recall that the kernel of a group homomorphism is also a normal subgroup. Like the situation with groups, we can say something even stronger about the kernel of a ring homomorphism. This will lead us to the notion of an \textbf{ideal}.

\begin{theorem}
Let $\phi:R\to S$ be a ring homomorphism.  If $\alpha\in\ker(\phi)$ and $r\in R$, then $\alpha r, r\alpha\in \ker(\phi)$.  That is, $\ker(\phi)$ is closed under multiplication by elements of $R$.
\end{theorem}

\end{section}

\begin{section}{Ideals and Quotient Rings}

Recall that in the case of a homomorphism $\phi$ of groups, the cosets of $\ker(\phi)$ have the structure of a group (that happens to be isomorphic to the image of $\phi$ by the First Isomorphism Theorem).  In this case, $\ker(\phi)$ is the identity of the associated quotient group.  Moreover, recall that every kernel is a normal subgroup of the domain and every normal subgroup can be realized as the kernel of some group homomorphism.  Can we do the same sort of thing for rings?

Let $\phi:R\to S$ be a ring homomorphism with $\ker(\phi)=I$.  Note that $\phi$ is also a group homomorphism of abelian groups and the cosets of $\ker(\phi)$ are of the form $r+I$.  More specifically, if $\phi(r)=a$, then $\phi^{-1}(a)=r+I$.

These cosets naturally have the structure of a ring isomorphic to the image of $\phi$:
\begin{align}
(r+I)+(s+I) & =  (r+s)+I\\
(r+I)(s+I) & = (rs)+I
\end{align}
The reason for this is that if $\phi^{-1}(a)=X$ and $\phi^{-1}(b)=Y$, then the inverse image of $a+b$ and $ab$ are $X+Y$ and $XY$, respectively.

The corresponding ring of cosets is called the \textbf{quotient ring} of $R$ by $I=\ker(\phi)$ and is denoted by $R/I$.  The additive structure of the quotient ring $R/I$ is exactly the additive quotient group of the additive abelian group $R$ by the normal subgroup $I$ (all subgroups are normal in abelian groups).  When $I$ is the kernel of some ring homomorphism $\phi$, the additive abelian quotient group $R/I$ also has a multiplicative structure defined in (2) above, making $R/I$ into a ring.

\begin{center}
\emph{Can we make $R/I$ into a ring for any subring $I$?}
\end{center}

The answer is ``no" in general, just like in the situation with groups.  But perhaps this isn't obvious because if $I$ is an arbitrary subring of $R$, then $I$ is necessarily an additive subgroup of the abelian group $R$, which implies that $I$ is an additive normal subgroup of the group $R$.  It turns out that the multiplicative structure of $R/I$ may not be well-defined if $I$ is an arbitrary subring.

Let $I$ be an arbitrary \emph{subgroup} of the additive group $R$.  Let $r+I$ and $s+I$ be two arbitrary cosets.  In order for multiplication of the cosets to be well-defined, the product of the two cosets must be independent of choice of representatives.  Let $r+\alpha$ and $s+\beta$ be arbitrary representatives of $r+I$ and $s+I$, respectively ($\alpha,\beta\in I$), so that $r+I=(r+\alpha)+I$ and $s+I=(s+\beta)+I$.  We must have
\begin{align}
(r+\alpha)(s+\beta)+I & =rs+I.
\end{align}
This needs to be true for all possible choices of $r,s\in R$ and $\alpha, \beta\in I$.  In particular, it must be true when $r=s=0$.  In this case, we must have
\begin{align}
\alpha\beta+I & =I.
\end{align}
But this only happens when $\alpha\beta\in I$.  That is, one requirement for multiplication of cosets to be well-defined is that $I$ must be closed under multiplication, making $I$ a \emph{subring}.

Next, if we let $s=0$ and let $r$ be arbitrary, we see that we must have $r\beta\in I$ for every $r\in R$ and every $\beta\in I$.  That is, it must be the case that $I$ is closed under multiplication on the left by elements from $R$.  Similarly, letting $r=0$, we can conclude that we must have $I$ closed under multiplication on the right by elements from $R$.  

On the other hand, if $I$ is closed under multiplication on the left and on the right by elements from $R$, then it is clear that relation (4) above is satisfied.

It is easy to verify that if the multiplication of cosets defined in (2) above is well-defined, then this multiplication makes the additive quotient group $R/I$ into a ring (just check the axioms for being a ring).

We have shown that the quotient $R/I$ of the ring $R$ by a subgroup $I$ has a natural ring structure if and only if $I$ is closed under multiplication on the left and right by elements of $R$ (which also forces $I$ to be a subring).  Such subrings are called \textbf{ideals}.

\begin{definition}
Let $R$ be a ring and let $I$ be a subset of $R$.
\begin{enumerate}[label=\rm{(\alph*)}]
\item $I$ is a \textbf{left ideal} (respectively, \textbf{right ideal}) of $R$ if $I$ is a subring and $rI\subseteq I$ (respectively, $Ir\subseteq I$) for all $r\in R$.
\item $I$ is an \textbf{ideal} (or \textbf{two-sided ideal}) if $I$ is both a left and a right ideal.
\end{enumerate}
\end{definition}

Here's a summary of everything that just happened.

\begin{theorem}
Let $R$ be a ring and let $I$ be an ideal of $R$.  Then the additive quotient group $R/I$ is a ring under the binary operations:
\begin{align}
(r+I)+(s+I) & =  (r+s)+I\\
(r+I)(s+I) & = (rs)+I
\end{align}
for all $r,s\in R$.  Conversely, if $I$ is any subgroup such that the above operations are well-defined, then $I$ is an ideal of $R$.
\end{theorem}

\begin{theorem}
If $R$ a commutative ring and $I$ is an ideal of $R$, then $R/I$ is a commutative ring.
\end{theorem}

\begin{theorem}
Suppose $I$ and $J$ are ideals of the ring $R$.  Then $I\cap J$ is an ideal of $R$.
\end{theorem}

As you might expect, we have some isomorphism theorems.

\begin{theorem}[First Isomorphism Theorem for Rings]
If $\phi:R\to S$ is a ring homomorphism, then $\ker(\phi)$ is an ideal of $R$ and $R/\ker(\phi)\cong \phi(R)$.
\end{theorem}

%If $I$ and $J$ are ideals of $R$, then it is easy to verify that $I\cap J$, $I+J=\{a+b\mid a\in I, b\in J\}$, and $IJ=\{\text{finite sums of elements of the form }ab\text{ for }a\in I, b\in J\}$ are also ideals of $R$.

We also have the expected Second, Third, and Fourth Isomorphism Theorems for rings.  The next theorem tells us that a subring is an ideal if and only if it is a kernel of a ring homomorphism.

\begin{theorem}
If $I$ is any ideal of $R$, then the \textbf{natural projection} $\pi:R\to R/I$ defined via $\pi(r)=r+I$ is a surjective ring homomorphism with $\ker(\pi)=I$.
\end{theorem}

For the remainder of this section, assume that $R$ is a ring with identity $1\neq 0$.

\begin{definition}
Let $A$ be any subset of $R$. Let $(A)$ denote the smallest ideal of $R$ containing $A$, called the \textbf{ideal generated by $A$}. If $A$ consists of a single element, say $A=\{a\}$, then $(a):=(\{a\})$ is called a \textbf{principal ideal}.
\end{definition}

\begin{remark}
The following facts are easily verified.
\begin{enumerate}[label=\rm{(\alph*)}]
\item $(A)$ is the intersection of all ideals containing $A$.
\item If $R$ is commutative, then $(a)=aR:=\{ar\mid r\in R\}$.
\end{enumerate}
\end{remark}

\begin{example}\label{ex:PrincipalIdeals}
In $\mathbb{Z}$, $n\mathbb{Z}=(n)=(-n)$.  In fact, these are the only ideals in $\mathbb{Z}$ (since these are the only subgroups). So, all the ideals in $\mathbb{Z}$ are principal. If $m$ and $n$ are positive integers, then $n\mathbb{Z}\subseteq m\mathbb{Z}$ if and only if $m$ divides $n$.  Moreover, we have $(m,n)=(d)$, where $d$ is the greatest common divisor of $m$ and $n$.
\end{example}

\begin{problem}
Consider the ideal $(2,x)$ in $\mathbb{Z}[x]$. Note that $(2,x)=\{2p(x)+xq(x)\mid p(x),q(x)\in\mathbb{Z}[x]\}$.  Argue that $(2,x)$ is not a principal ideal, i.e., there is no single polynomial in $\mathbb{Z}[x]$ that we can use to generate $(2,x)$.
\end{problem}

\begin{theorem}
Assume $R$ is a commutative ring with $1\neq 0$. Let $I$ be an ideal of $R$. Then $I=R$ if and only if $I$ contains a unit.
\end{theorem}

\begin{theorem}
Assume $R$ is a commutative ring with $1\neq 0$.  Then $R$ is a field if and only if its only ideals are $(0)$ and $R$.
\end{theorem}

Loosely speaking, the previous results say that fields are ``like simple groups" (i.e, groups with no non-trivial normal subgroups).

\begin{corollary}\label{cor:HomFromField}
If $R$ is a field, then every nonzero ring homomorphism from $R$ into another ring is an injection.
\end{corollary}

\end{section}

\begin{section}{Maximal and Prime Ideals}

In this section of notes, we will study two important classes of ideals, namely \textbf{maximal} and \textbf{prime} ideals, and study the relationship between them. Throughout this entire section, we assume that all rings have a multiplicative identity $1\neq 0$. 

\begin{definition}
Assume $R$ is a commutative ring with 1. An ideal $M$ in a ring $R$ is called a \textbf{maximal ideal} if $M\neq R$ and the only ideals containing $M$ are $M$ and $R$.
\end{definition}

\begin{example}
Here are a few examples.  Checking the details is left as an exercise.
\begin{enumerate}[label=\rm{(\alph*)}]
\item In $\mathbb{Z}$, all the ideals are of the form $n\mathbb{Z}$ for $n\in\mathbb{Z}^+$.  The maximal ideals correspond to the ideals $p\mathbb{Z}$, where $p$ is prime.
\item Consider the integral domain $\mathbb{Z}[x]$.  The ideals $(x)$ (i.e., the subring containing polynomials with 0 constant term) and $(2)$ (i.e, the set of polynomials with even coefficients) are not maximal since both are contained in the proper ideal $(2,x)$.  However, as we shall see soon, $(2,x)$ is maximal in $\mathbb{Z}[x]$.
\item The zero ring has no maximal ideals.
\item Consider the abelian group $\mathbb{Q}$ under addition.  We can turn $\mathbb{Q}$ into a trivial ring by defining $ab=0$ for all $a,b\in\mathbb{Q}$.  In this case, the ideals are exactly the additive subgroups of $\mathbb{Q}$.  However, $\mathbb{Q}$ has no maximal subgroups, and so $\mathbb{Q}$ has no maximal ideals.
\end{enumerate}
\end{example}

The next result states that rings with an identity $1\neq 0$ always have maximal ideals.  It turns out that we won't need this result going forward, so we'll skip its proof.  However, it is worth noting that all known proofs make use of Zorn's Lemma (equivalent to the Axiom of Choice), which is also true for the proofs that a finitely generated group has maximal subgroups or that every vector spaces has a basis.

\begin{theorem}
In a ring with $1$, every proper ideal is contained in a maximal ideal.
\end{theorem}

For commutative rings, there is a very nice characterization about maximal ideals in terms of the structure of their quotient rings.

\begin{theorem}
Assume $R$ is a commutative ring with 1.  Then $M$ is a maximal ideal if and only if the quotient ring $R/M$ is a field.
\end{theorem}

\begin{example}
We can use the previous theorem to verify whether an ideal is maximal.
\begin{enumerate}[label=\rm{(\alph*)}]
\item Recall that $\mathbb{Z}/n\mathbb{Z}\cong \mathbb{Z}_n$ and that $\mathbb{Z}_n$ is a field if and only if $n$ is prime.  We can conclude that $n\mathbb{Z}$ is a maximal ideal precisely when $n$ is prime.
\item Define $\phi:\mathbb{Z}[x]\to\mathbb{Z}$ via $\phi(p(x))=p(0)$.  Then $\phi$ is surjective and $\ker(\phi)=(x)$.  By the First Isomorphism Theorem for Rings, we see that $\mathbb{Z}[x]/(x)\cong \mathbb{Z}$.  However, $\mathbb{Z}$ is not a field.  Hence $(x)$ is not maximal in $\mathbb{Z}[x]$.  Now, define $\psi:\mathbb{Z}\to\mathbb{Z}_2$ via $\psi(x)=x\mod 2$ and consider the composite homomorphism $\psi\circ\phi:\mathbb{Z}\to\mathbb{Z}_2$.    It is clear that $\psi\circ\phi$ is onto and the kernel of $\psi\circ\phi$ is given by $\{p(x)\in\mathbb{Z}[x]\mid p(0)\in 2\mathbb{Z}\}=(2,x)$. Again by the First Isomorphism Theorem for Rings, $\mathbb{Z}[x]/(2,x)\cong \mathbb{Z}_2$.  Since $\mathbb{Z}_2$ is a field, $(2,x)$ is a maximal ideal.
\end{enumerate}
\end{example}

\begin{definition}
Assume $R$ is a commutative ring with 1.  An ideal $P$ is called a \textbf{prime ideal} if $P\neq R$ and whenever the product $ab\in P$ for $a,b\in R$, then at least one of $a$ or $b$ is in $P$.
\end{definition}

\begin{example}
In any integral domain, the 0 ideal $(0)$ is a prime ideal. What if the ring is not an integral domain?
\end{example}

\begin{remark}
The notion of a prime ideal is a generalization of ``prime" in $\mathbb{Z}$. Suppose $n\in\mathbb{Z}^+\setminus\{1\}$ such that $n$ divides $ab$.  In this case, $n$ is guaranteed to divide either $a$ or $b$ exactly when $n$ is prime.  Now, let $n\mathbb{Z}$ be a proper ideal in $\mathbb{Z}$ with $n>1$ and suppose $ab\in \mathbb{Z}$ for $a,b\in\mathbb{Z}$. In order for $n\mathbb{Z}$ to be a prime ideal, it must be true that $n$ divides either $a$ or $b$.  However, this is only guaranteed to be true for all $a,b\in\mathbb{Z}$ when $p$ is prime.  That is, the nonzero prime ideals of $\mathbb{Z}$ are of the form $p\mathbb{Z}$, where $p$ is prime.  Note that in the case of the integers, the maximal and nonzero prime ideals are the same.
\end{remark}

\begin{theorem}
Assume $R$ is a commutative ring with 1.  Then $P$ is a prime ideal in $R$ if and only if the quotient ring $R/P$ is an integral domain.
\end{theorem}

\begin{corollary}
Assume $R$ is a commutative ring with 1.  Every maximal ideal of $R$ is a prime ideal.
\end{corollary}

\begin{example}
Recall that $\mathbb{Z}[x]/(x)\cong\mathbb{Z}$.  Since $\mathbb{Z}$ is an integral domain, it must be the case that $(x)$ is a prime ideal in $\mathbb{Z}[x]$.  However, as we saw in an earlier example, $(x)$ is not maximal in $\mathbb{Z}[x]$ since $\mathbb{Z}$ is not a field.  This shows that the converse of the previous corollary is not true.
\end{example}

\end{section}
%Appendices
\appendix
\chapter{Elements of Style for Proofs}
\label{appendix:elements_of_style}

Years of elementary school math taught us incorrectly that the answer to a math problem is just a single number, ``the right answer.''  It is time to unlearn those lessons; those days are over.  From here on out, mathematics is about discovering proofs and writing them clearly and compellingly.

The following rules apply whenever you write a proof.  I may refer to them, by number, in my comments on your homework and exams.  Keep these rules handy so that you may refer to them as you write your proofs.

\begin{enumerate}

\item \textbf{The writing process.}  Use the same writing process that you would for any writing project.
\begin{enumerate}
\item Prewriting.~This is the most mathematical step of the process. Often this step takes place on scratch paper. Figure out the mathematics: test conjectures, work out examples, try various proof techniques, etc. 
\item Writing.~When you understand the mathematics it is time to write the first draft. The draft may have extraneous information, be missing  information, be written in the wrong order, contain some minor mathematical errors, etc.
\item Revising.~Once you have a first draft, go back and revise the writing. Focus on large changes such as adding, removing, rearranging, and replacing. Fix any mathematical errors. 
\item Editing/proofreading.~At this stage you must attend to the fine details. Fix any problems with spelling, grammar, word choice, punctuation, etc. Make sure all of the mathematics is typeset correctly.
\item Publishing.~Make the final changes so that you can submit your work. You may need to fit it to a style guide (get the margins correct, add a title page, etc.), convert it to a certain file type, or print it.
\end{enumerate}

\item \textbf{The burden of communication lies on you, not on your reader.}
It is your job to explain your thoughts; it is not your reader's job to guess them from a few hints. You are trying to convince a skeptical reader who doesn't believe you, so you need to argue with airtight logic in crystal clear language;
otherwise the reader will continue to doubt.
If you didn't write something on the paper, then (a) you didn't communicate it, (b) the reader didn't learn it, and (c) the grader has to assume you didn't know it in the first place.
  
\item \textbf{Tell the reader what you're proving.} The reader doesn't necessarily know or remember what ``Theorem 2.13'' is. Even a professor grading a stack of papers might lose track from time to time. Therefore, the statement you are proving should be on the same page as the beginning of your proof. For an exam this won't be a problem, of course, but on your homework, recopy the claim you are proving. This has the additional advantage that when you study for exams by reviewing your homework, you won't have to flip back in the notes/textbook to know what you were proving.

\item \textbf{Use English words.} Although there will usually be equations or mathematical statements in your proofs, use English sentences to connect them and display their logical relationships. If you look in your notes/textbook, you'll see that each proof consists mostly of English words.

\item \textbf{Use complete sentences.} If you wrote a history essay in sentence fragments, the reader would not understand what you meant; likewise in mathematics you must use complete sentences, with verbs, to convey your logical train of thought.

Some complete sentences can be written purely in mathematical symbols, such as equations (e.g., $a^3=b^{-1}$), inequalities (e.g., $x<5$), and other relations (like $5\big|10$ or $7\in\mathbb{Z}$). These statements usually express a relationship between two mathematical \emph{objects}, like numbers or sets.  However, it is considered bad style to begin a sentence with symbols.  A common phrase to use to avoid starting a sentence with mathematical symbols is ``We see that...''

\item \textbf{Show the logical connections among your sentences.} Use phrases like ``Therefore'' or ``because'' or ``if\ldots, then\ldots'' or ``if and only if'' to connect your sentences.
  
\item \textbf{Know the difference between statements and objects.} A mathematical object is a \emph{thing}, a noun, such as a group, an element, a vector space, a number, an ordered pair, etc. Objects either exist or don't exist. Statements, on the other hand, are mathematical \emph{sentences}:  they can be true or false.

When you see or write a cluster of math symbols, be sure you know  whether it's an object (e.g., ``$x^2+3$'') or a statement (e.g., ``$x^2+3<7$''). One way to tell is that every mathematical statement includes a verb, such as $=$, $\leq$, ``divides'', etc.

\item \textbf{``$=$'' means equals.} Don't write $A=B$ unless you mean that $A$ actually equals $B$. This rule seems obvious, but there is a great temptation to be sloppy.  In calculus, for example, some people might write $f(x)=x^{2}=2x$ (which is false), when they really mean that ``if $f(x)=x^{2}$, then $f'(x)=2x$.''

\item \textbf{Don't interchange ${=}$ and ${\implies}$.} The equals sign connects two \emph{objects}, as in ``$x^2=b$''; the symbol ``$\implies$'' is an abbreviation for ``implies'' and connects two \emph{statements}, as in ``$a+b=a \implies b=0$.''  You should avoid using $\implies$ in your formal write-ups.

\item \textbf{Say exactly what you mean.} Just as the $=$ is sometimes abused, so too people sometimes write $A\in B$ when they mean $A\subseteq B$, or write $a_{ij}\in A$ when they mean that $a_{ij}$ is an entry in matrix $A$. Mathematics is a very precise language, and there is a way to say exactly what you mean; find it and use it.

\item \textbf{Don't write anything unproven.} Every statement on your paper should be something you \emph{know} to be true. The reader expects your proof to be a series of statements, each proven by the statements that came before it. If you ever need to write something you don't yet know is true, you \emph{must} preface it with words like ``assume,'' ``suppose,'' or ``if'' (if you are temporarily assuming it), or with words like ``we need to show that'' or ``we claim that'' (if it is your goal). Otherwise the reader will think they have missed part of your proof.

\item \textbf{Write strings of equalities (or inequalities) in the proper order.} When your reader sees something like
\[
A=B\leq C=D,
\]
he/she expects to understand easily why $A=B$, why $B\leq C$, and why $C=D$, and he/she expects the \emph{point} of the entire line to be the more complicated fact that $A\leq D$. For example, if you were computing the distance $d$ of the point $(12,5)$ from the origin, you could write
\[
d = \sqrt{12^2+5^2} = 13.
\]
In this string of equalities, the first equals sign is true by the Pythagorean theorem, the second is just arithmetic, and the \emph{point} is that the first item equals the last item: $d=13$.

A common error is to write strings of equations in the wrong order. For example, if you were to write ``$\sqrt{12^2+5^2}=13=d$'', your reader would understand the first equals sign, would be baffled as to how we know $d=13$, and would be utterly perplexed as to why you wanted or needed to go through $13$ to prove that $\sqrt{12^2+5^2}=d$.

\item \textbf{Avoid circularity.}  Be sure that no step in your proof makes use of the conclusion!

\item \textbf{Don't write the proof backwards.} Beginning students often attempt to write ``proofs'' like the following, which attempts to prove that $\tan^2(x)  = \sec^2(x) - 1$:
\begin{align*}
\tan^2(x) =& \sec^2(x) - 1 \\
\left(\frac{\sin(x)}{\cos(x)}\right)^2 =& \frac{1}{\cos^2(x)} - 1 \\
\frac{\sin^2(x)}{\cos^2(x)} =&  \frac{1-\cos^2(x)}{\cos^2(x)} \\
\sin^2(x) =& 1-\cos^2(x) \\
\sin^2(x) + \cos^2(x) =& 1 \\
1 =& 1
\end{align*}
Notice what has happened here:  the writer \emph{started} with the conclusion, and deduced the true statement ``$1=1$.'' In other words, he/she has proved ``If $\tan^2(x) = \sec^2(x) - 1$, then $1=1$,'' which is true but highly uninteresting.

Now this isn't a bad way of \emph{finding} a proof.
Working backwards from your goal often is a good strategy \emph{on your scratch paper},
but when it's time to \emph{write} your proof,
you have to start with the hypotheses and work to the conclusion.

\item \textbf{Be concise.} Most students err by writing their proofs too short, so that the reader can't understand their logic. It is nevertheless quite possible to be too wordy, and if you find yourself writing a full-page essay, it's probably because you don't really have a proof, but just an intuition. When you find a way to turn that intuition into a formal proof, it will be much shorter.

\item \textbf{Introduce every symbol you use.} If you use the letter ``$k$,'' the reader should know exactly what $k$ is. Good phrases for introducing symbols include ``Let $n\in \mathbb{N}$,'' ``Let $k$ be the least integer such that\ldots,'' ``For every real number $a$\ldots,'' and ``Suppose that $X$ is a counterexample.''
  
\item \textbf{Use appropriate quantifiers (once).} When you introduce a variable $x\in S$, it must be clear to your reader whether you mean ``for all $x\in S$'' or just ``for some $x\in S$.'' If you just say something like ``$y=x^2$ where $x\in S$,'' the word ``where'' doesn't indicate whether you mean ``for all'' or ``some''.

Phrases indicating the quantifier ``for all'' include ``Let $x\in S$''; ``for all $x\in S$''; ``for every $x\in S$''; ``for each $x\in S$''; etc. Phrases indicating the quantifier ``some'' (or ``there exists'') include ``for some $x\in S$''; ``there exists an $x\in S$''; ``for a suitable choice of $x\in S$''; etc.

On the other hand, don't introduce a variable more than once! Once you have said ``Let $x\in S$,'' the letter $x$ has its meaning defined. You don't \emph{need} to say ``for all $x\in S$'' again, and you definitely should \emph{not} say ``let $x\in S$'' again.

\item \textbf{Use a symbol to mean only one thing.} Once you use the letter $x$ once, its meaning is fixed for the duration of your proof. You cannot use $x$ to mean anything else.

\item \textbf{Don't ``prove by example.''}\label{proof_by_example} Most problems ask you to prove that something is true ``for all''---You \emph{cannot} prove this by giving a single example, or even a hundred. Your answer will need to be a logical argument that holds for \emph{every example there possibly could be}.

\item \textbf{Write ``Let $x=\dots$,'' not ``Let $\dots=x$.''} When you have an existing expression, say $a^{2}$, and you want to give it a new, simpler name like $b$, you should write ``Let $b=a^{2}$,'' which means, ``Let the new symbol $b$ mean $a^{2}$.''This convention makes it clear to the reader that $b$ is the brand-new symbol and $a^{2}$ is the old expression he/she already understands.

If you were to write it backwards, saying ``Let $a^{2}=b$,'' then your startled reader would ask, ``What if $a^{2}\neq b$?''
  
\item \textbf{Make your counterexamples concrete and specific.} Proofs need to be entirely general, but counterexamples should be absolutely concrete. When you provide an example or counterexample, make it as specific as possible. For a set, for example, you must name its elements, and for a function you must give its rule. Do not say things like ``$\theta$ could be one-to-one but not onto'';
instead, provide an actual function $\theta$ that \emph{is} one-to-one but not onto.
    
\item \textbf{Don't include examples in proofs.} Including an example very rarely adds anything to your proof. If your logic is sound, then it doesn't need an example to back it up. If your logic is bad, a dozen examples won't help it (see rule \ref{proof_by_example}). There are only two valid reasons to include an example in a proof: if it is a \emph{counterexample} disproving something, or if you are performing complicated manipulations in a general setting and the example is just to help the reader understand what you are saying.

\item \textbf{Use scratch paper.} Finding your proof will be a long, potentially messy process, full of false starts and dead ends. Do all that on scratch paper
until you find a real proof, and only then break out your clean paper to write your final proof carefully. \emph{Do not hand in your scratch work!}

Only sentences that actually contribute to your proof should be part of the proof. Do not just perform a ``brain dump,'' throwing everything you know onto the paper before showing the logical steps that prove the conclusion. \emph{That is what scratch paper is for.}

\end{enumerate}
\chapter{Fancy Mathematical Terms}
\label{appendix:fancy_math_terms}

Here are some important mathematical terms that you will encounter in this course and throughout your mathematical career.

\begin{enumerate}
\item
\textbf{Definition}---a precise and unambiguous description of the meaning of a mathematical term.  It characterizes the meaning of a word by giving all the properties and only those properties that must be true.
\item
\textbf{Theorem}---a mathematical statement that is proved using rigorous mathematical reasoning.  In a mathematical paper, the term theorem is often reserved for the most important results.
\item
\textbf{Lemma}---a minor result whose sole purpose is to help in proving a theorem.  It is a stepping stone on the path to proving a theorem. Very occasionally lemmas can take on a life of their own (Zorn's lemma, Urysohn's lemma, Burnside's lemma, Sperner's lemma).
\item
\textbf{Corollary}---a result in which the (usually short) proof relies heavily on a given theorem (we often say that ``this is a corollary of Theorem A'').
\item
\textbf{Proposition}---a proved and often interesting result, but generally less important than a theorem.
\item
\textbf{Conjecture}---a statement that is unproved, but is believed to be true (Collatz conjecture, Goldbach conjecture, twin prime conjecture).
\item
\textbf{Claim}---an assertion that is then proved.  It is often used like an informal lemma.
\item
\textbf{Axiom/Postulate}---a statement that is assumed to be true without proof. These are the basic building blocks from which all theorems are proved (Euclid's five postulates, Zermelo-Frankel axioms, Peano axioms).
\item
\textbf{Identity}---a mathematical expression giving the equality of two (often variable) quantities (trigonometric identities, Euler's identity).
\item
\textbf{Paradox}---a statement that can be shown, using a given set of axioms and definitions, to be both true and false. Paradoxes are often used to show the inconsistencies in a flawed theory (Russell's paradox).  The term paradox is often used informally to describe a surprising or counterintuitive result that follows from a given set of rules (Banach-Tarski paradox, Alabama paradox, Gabriel's horn).
\end{enumerate}
\chapter{Definitions in Mathematics}
\label{appendix:definitions}

It is difficult to overstate the importance of definitions in mathematics. Definitions play a different role in mathematics than they do in everyday life. 

Suppose you give your friend a piece of paper containing the definition of the rarely-used word \textbf{rodomontade}. According to the Oxford English Dictionary\footnote{http://www.oed.com/view/Entry/166837} (OED) it is:
\begin{quote}
A vainglorious brag or boast; an extravagantly boastful, arrogant, or bombastic speech or piece of writing; an arrogant act.
\end{quote}
Give your friend some time to study the definition. Then take away the paper. Ten minutes later ask her to define rodomontade. Most likely she will be able to give a reasonably accurate definition. Maybe she'd say something like, ``It is a speech or act or piece of writing created by a pompous or egotistical person who wants to show off how great they are.'' It is unlikely that she will have quoted the OED word-for-word. In everyday English that is fine---you would probably agree that your friend knows the meaning of the rodomontade. This is because most definitions are \emph{descriptive}. They describe the common usage of a word. 

Let us take a mathematical example. The OED\footnote{http://www.oed.com/view/Entry/40280}  gives this definition of \emph{continuous}.
\begin{quote}
Characterized by continuity; extending in space without interruption of substance; having no interstices or breaks; having its parts in immediate connection; connected, unbroken.
\end{quote}
Likewise, we often hear calculus students speak of a continuous function as one whose graph can be drawn ``without picking up the pencil.'' This definition is descriptive. (As we learned in calculus the picking-up-the-pencil description is not a perfect description of continuous functions.) This is not a mathematical definition. 

Mathematical definitions are \emph{prescriptive}. The definition must prescribe the exact and correct meaning  of a word. Contrast the OED's descriptive definition of continuous with the the definition of continuous found in a real analysis textbook.
\begin{quote}
A function $f:A\to \mathbb{R}$ is \textbf{continuous at a point} $c\in A$ if,  for all $\varepsilon>0$, there exists $\delta>0$ such that whenever $|x-c|<\delta$ (and $x\in A$) it follows that $|f(x)-f(c)|<\varepsilon$. If $f$ is continuous at every point in the domain $A$, then we say that $f$ is \textbf{continuous on} $A$.\footnote{This definition is taken from page 109 of Stephen Abbott's \emph{Understanding Analysis}, but the definition would be essentially the same in any modern real analysis textbook.} 
\end{quote}
In mathematics there is very little freedom in definitions. Mathematics is a deductive theory; it is impossible to state and prove theorems without clear definitions of the mathematical terms. The definition of a term must completely, accurately, and unambiguously describe the term. Each word is chosen very carefully and the order of the words is  critical. In the definition of continuity changing ``there exists'' to ``for all,'' changing the orders of quantifiers, changing $<$ to $\leq$ or $>$, or changing $\mathbb{R}$ to $\mathbb{Z}$ would completely change the meaning of the definition. 

What does this mean for you, the student? Our recommendation is that at this stage you memorize the definitions word-for-word. It is the safest way to guarantee that you have it correct. As you gain confidence and familiarity with the subject you may be ready to modify the wording. You may want to change ``for all'' to ``given any'' or you may want to change $|x-c|<\delta$ to $-\delta<x-c<\delta$ or to ``the distance between $x$ and $c$ is less than $\delta$.'' 

Of course, memorization is not enough; you must have a conceptual understanding of the term, you must see how the formal definition matches up with your conceptual understanding, and you must know how to work with the definition. It is perhaps with the first of these that descriptive definitions are useful. They are useful for building intuition and for painting the ``big picture.'' Only after days (weeks, months, years?) of experience does one get an intuitive feel for the $\varepsilon,\delta$-definition of continuity; most mathematicians have the ``picking-up-the-pencil'' definitions in their head. This is fine as long as we know that it is imperfect, and that when we prove theorems about continuous functions mathematics we use the mathematical definition. 

We end this discussion with an amusing real-life example in which a descriptive definition was not sufficient. In 2003 the German version of the game show \emph{Who wants to be a millionaire?} contained the following question: ``Every rectangle is: (a) a rhombus, (b) a trapezoid, (c) a square, (d) a parallelogram.'' 

The confused contestant decided to skip the question and left with \euro 4000. Afterward the show received letters from irate viewers. Why were the contestant and the viewers upset with this problem? Clearly a rectangle is a parallelogram, so (d) is the answer. But what about (b)? Is a rectangle a trapezoid? We would describe a trapezoid as a quadrilateral with a pair of parallel sides. But this leaves open the question: can a trapezoid have \emph{two} pairs of parallel sides or must there only be \emph{one} pair? The viewers said two pairs is allowed, the producers of the television show said it is not. This is a case in which a clear, precise, mathematical definition is required.

\end{document}