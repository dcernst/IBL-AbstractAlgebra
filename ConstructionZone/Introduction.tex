\chapter{Introduction}

\begin{section}{What is Abstract Algebra?}

Abstract algebra is the subject area of mathematics that studies algebraic structures, such as groups, rings, fields, modules, vector spaces, and algebras. This course is an introduction to abstract algebra. We will spend most of our time studying groups. Group theory is the study of symmetry, and is one of the most beautiful areas in all of mathematics. It arises in puzzles, visual arts, music, nature, the physical and life sciences, computer science, cryptography, and of course, throughout mathematics. This course will cover the basic concepts of group theory, and a special effort will be made to emphasize the intuition behind the concepts and motivate the subject matter.  In the last few weeks of the semester, we will also introduce rings and fields.

\epigraph{The mathematician does not study pure mathematics because it is useful; he studies it because he delights in it, and he delights in it because it is beautiful.}{\emph{Henri Poincar\'e}}

\end{section}

\begin{section}{An Inquiry-Based Approach}

This is not a lecture-oriented class or one in which mimicking prefabricated examples will lead you to success. You will be expected to work actively to construct your own understanding of the topics at hand with the readily available help of me and your classmates. Many of the concepts you learn and problems you work on will be new to you and ask you to stretch your thinking. You will experience \emph{frustration} and \emph{failure} before you experience \emph{understanding}. This is part of the normal learning process. If you are doing things well, you should be confused at different points in the semester. The material is too rich for a human being to completely understand it immediately. Your viability as a professional in the modern workforce depends on your ability to embrace this learning process and make it work for you.

\epigraph{Don't fear failure.  Not failure, but low aim, is the crime. In great attempts it is glorious even to fail.}{\emph{Bruce Lee}}

In order to promote a more active participation in your learning, we will incorporate ideas from an educational philosophy called inquiry-based learning (IBL).  Loosely speaking, IBL is a student-centered method of teaching mathematics that engages students in sense-making activities.  Students are given tasks requiring them to solve problems, conjecture, experiment, explore, create, and communicate.  Rather than showing facts or a clear, smooth path to a solution, the instructor guides and mentors students via well-crafted problems through an adventure in mathematical discovery.  According to \href{https://www.colorado.edu/eer/sites/default/files/attached-files/laursenrasmussencommentaryauthorversion0219.pdf}{Laursen and Rasmussen (2019)}, the Four Pillars of IBL are:
\begin{itemize}
\item Students engage deeply with coherent and meaningful mathematical tasks.
\item Students collaboratively process mathematical ideas.
\item Instructors inquire into student thinking.
\item Instructors foster equity in their design and facilitation choices.
\end{itemize}

Much of the course will be devoted to students presenting their proposed solutions or proofs on the board and a significant portion of your grade will be determined by how much mathematics you produce.  I use the word \emph{produce} because I believe that the best way to learn mathematics is by doing mathematics.  Someone cannot master a musical instrument or a martial art by simply watching, and in a similar fashion, you cannot master mathematics by simply watching; you must do mathematics!

In any act of creation, there must be room for experimentation, and thus allowance for mistakes, even failure. A key goal of our community is that we support each other---sharpening each other's thinking but also bolstering each other's confidence---so that we can make failure a productive experience. Mistakes are inevitable, and they should not be an obstacle to further progress. It's normal to struggle and be confused as you work through new material. Accepting that means you can keep working even while feeling stuck, until you overcome and reach even greater accomplishments.

\epigraph{You will become clever through your mistakes.}{\emph{German Proverb}}

Furthermore, it is important to understand that solving genuine problems is difficult and takes time.  You shouldn't expect to complete each problem in 10 minutes or less.  Sometimes, you might have to stare at the problem for an hour before even understanding how to get started.

In this course, everyone will be required to
\begin{itemize}
\item read and interact with course notes and textbook on your own;
\item write up quality solutions/proofs to assigned problems;
\item present solutions/proofs on the board to the rest of the class;
\item participate in discussions centered around a student's presented solution/proof;
\item call upon your own prodigious mental faculties to respond in flexible, thoughtful, and creative ways to problems that may seem unfamiliar on first glance.
\end{itemize}
As the semester progresses, it should become clear to you what the expectations are.

\epigraph{Tell me and I forget, teach me and I may remember, involve me and I learn.}{\emph{Benjamin Franklin}}

\end{section}

\begin{section}{Rights of the Learner}\label{sec:Rights of the Learner}
As a reader of this textbook, you have the right:
\begin{enumerate}
\item to be confused,
\item to make a mistake and to revise your thinking,
\item to speak, listen, and be heard, and
\item to enjoy doing mathematics.
\end{enumerate}

\epigraph{You may encounter many defeats, but you must not be defeated.}{\emph{Maya Angelou}}
	
\end{section}

\begin{section}{Structure of the Notes}

As you read the notes, you will be required to digest the material in a meaningful way.  It is your responsibility to read and understand new definitions and their related concepts.  However, you will be supported in this sometimes difficult endeavor. In addition, you will be asked to complete problems aimed at solidifying your understanding of the material.  Most importantly, you will be asked to make conjectures, produce counterexamples, and prove theorems.

The items labeled as \textbf{Definition} and \textbf{Example} are meant to be read and digested.  However, the items labeled as \textbf{Problem}, \textbf{Theorem}, and \textbf{Corollary} require action on your part.  Items labeled as \textbf{Problem} are sort of a mixed bag. Some Problems are computational in nature and aimed at improving your understanding of a particular concept while others ask you to provide a counterexample for a statement if it is false or to provide a proof if the statement is true. Items with the \textbf{Theorem} and \textbf{Corollary} designation are mathematical facts and the intention is for you to produce a valid proof of the given statement.  The main difference between a \textbf{Theorem} and a \textbf{Corollary} is that corollaries are typically statements that follow quickly from a previous theorem.  In general, you should expect corollaries to have very short proofs.  However, that doesn't mean that you can't produce a more lengthy yet valid proof of a corollary.

It is important to point out that there are very few examples in the notes.  This is intentional.  One of the goals of the items labeled as \textbf{Problem} is for you to produce the examples.

Lastly, there are many situations where you will want to refer to an earlier definition, problem, theorem, or corollary.  In this case, you should reference the statement by number.  For example, you might write something like, ``By Theorem~\ref{thm:order_element_divides_group_order}, we see that\ldots."

\end{section}

\begin{section}{Some Minimal Guidance}
Especially in the opening sections, it won't be clear what facts from your prior experience in mathematics we are ``allowed" to use.  Unfortunately, addressing this issue is difficult and is something we will sort out along the way.  However, in general, here are some minimal guidelines to keep in mind.  

First, there are times when we will need to do some basic algebraic manipulations.  You should feel free to do this whenever the need arises.  But you should show sufficient work along the way.  You do not need to write down justifications for basic algebraic manipulations (e.g., adding 1 to both sides of an equation, adding and subtracting the same amount on the same side of an equation, adding like terms, factoring, basic simplification, etc.).  

On the other hand, you do need to make explicit justification of the logical steps in a proof.  When necessary, you should cite a previous definition, theorem, etc. by number.

Unlike the experience many of you had writing proofs in geometry, our proofs will be written in complete sentences.  You should break sections of a proof into paragraphs and use proper grammar.  There are some pedantic conventions for doing this that I will point out along the way.  Initially, this will be an issue that most students will struggle with, but after a few weeks everyone will get the hang of it.

Ideally, you should rewrite the statements of theorems before you start the proof.  Moreover, for your sake and mine, you should label the statement with the appropriate number.  I will expect you to indicate where the proof begins by writing ``\emph{Proof.}" at the beginning.  Also, we will conclude our proofs with the standard ``proof box" (i.e., $\square$ or $\blacksquare$), which is typically right-justified.

Lastly, every time you write a proof, you need to make sure that you are making your assumptions crystal clear.  Sometimes there will be some implicit assumptions that we can omit, but at least in the beginning, you should get in the habit of stating your assumptions up front.  Typically, these statements will start off ``Assume\ldots" or ``Let\ldots".  

This should get you started.  We will discuss more as the semester progresses.  Now, go have fun and kick some butt!

\epigraph{If you want to sharpen a sword, you have to remove a little metal.}{\emph{Unknown}}

\end{section}