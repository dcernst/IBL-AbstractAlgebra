\chapter{Homomorphisms and the Isomorphism Theorems}
\label{chapter:homomorphisms}
\thispagestyle{empty}

\begin{section}{Homomorphisms}
Let $G_1$ and $G_2$ be groups. Recall that $\phi:G_1\to G_2$ is an isomorphism iff $\phi$
\begin{enumerate}[label=\rm{(\alph*)}]
\item is one-to-one, 
\item is onto, and
\item satisfies the homomorphic property.
\end{enumerate}
We say that $G_1$ is isomorphic to $G_2$ and write $G_1\cong G_2$ if such a $\phi$ exists. Loosely speaking, two groups are isomorphic if they have the ``same structure."  What if we drop the one-to-one and onto requirement?

\begin{definition}
Let $(G_1,*)$ and $(G_2,\circ)$ be groups. A function $\phi:G_1\to G_2$ is a \textbf{homomorphism} iff $\phi$ satisfies the homomorphic property:
\[
\phi(x*y)=\phi(x)\circ\phi(y)
\]
for all $x,y\in G_1$. At the risk of introducing ambiguity, we will usually omit making explicit reference to the binary operations and write the homomorphic property as
\[
\phi(xy)=\phi(x)\phi(y).
\]
\end{definition}

Group homomorphisms are analogous to linear transformations on vector spaces that one encounters in linear algebra.

Figure~\ref{fig:isoGroupTables2} captures a visual representation of the homomorphic property.  We encountered this same representation in Figure~\ref{fig:isoGroupTables}. If $\phi(x)=x'$, $\phi(y)=y'$, and $\phi(z)=z'$ while $z'=x'\circ y'$, then the only way $G_2$ may respect the structure of $G_1$ is for
\[
\phi(x*y)=\phi(z)=z'=x'\circ y'=\phi(x)\circ \phi(y).
\]

\begin{figure}
\begin{center}
\begin{tabu}{c|[2pt]ccc|c|c}
$*$                & & & & \cellcolor{green}$y$  & \\ \tabucline[2pt]{-}
                   & & & &                       & \\ \hline
\cellcolor{red}$x$ & & & & \cellcolor{orange}$z$ & \\ \hline
                   & & & &                      & \\
                   & & & &                      &
\end{tabu}
\hspace{1cm}
$\longrightarrow$
\hspace{1cm}
\begin{tabu}{c|[2pt]ccc|c|c}
$\circ$                & & & & \cellcolor{green}$y'$  & \\ \tabucline[2pt]{-}
                   & & & &                       & \\ \hline
\cellcolor{red}$x'$ & & & & \cellcolor{orange}$z'$ & \\ \hline
                   & & & &                      & \\
                   & & & &                      &
\end{tabu}
\end{center}
\caption{}\label{fig:isoGroupTables2}
\end{figure}

\begin{exercise}\label{exer:homomorphism}
Define $\phi:\mathbb{Z}_3\to D_3$ via $\phi(k)=r^k$. Prove that $\phi$ is a homomorphism and then determine whether $\phi$ is one-to-one or onto. Also, try to draw a picture of the homomorphism in terms of Cayley diagrams.
\end{exercise}

\begin{exercise}
Let $G$ and $H$ be groups. Prove that the function $\phi:G\times H\to G$ given by $\phi(g,h)=g$ is a homomorphism. This function is an example of a \textbf{projection map}.
\end{exercise}

There is always at least one homomorphism between two groups.

\begin{theorem}\label{thm:trivial_homomorphism}
Let $G_1$ and $G_2$ be groups. Define $\phi:G_1\to G_2$ via $\phi(g)=e_2$ (where $e_2$ is the identity of $G_2$).  Then $\phi$ is a homomorphism. This function is often referred to as the \textbf{trivial homomorphism} or the \textbf{$0$-map}.
\end{theorem}

Back in Section~\ref{sec:revisiting_isomorphisms}, we encountered several theorems about isomorphisms.  However, at the end of that section we remarked that some of those theorems did not require that the function be one-to-one and onto.  We collect those results here for convenience.

\begin{theorem}
Let $G_1$ and $G_2$ be groups and suppose $\phi:G_1\to G_2$ is a homomorphism.
\begin{enumerate}
\item If $e_1$ and $e_2$ are the identity elements of $G_1$ and $G_2$, respectively, then $\phi(e_1)=e_2$.
\item For all $g\in G_1$, we have $\phi(g^{-1})=[\phi(g)]^{-1}$.
\item If $H\leq G_1$, then $\phi(H)\leq G_2$, where
\[
\phi(H):=\{y\in G_2\mid \text{there exists } h\in H\text{ such that }\phi(h)=y\}. 
\]
Note that $\phi(H)$ is called the \textbf{image} of $H$. A special case is when $H=G_1$. Notice that $\phi$ is onto exactly when $\phi(G_1)=G_2$.
\end{enumerate}
\end{theorem}

The next two theorems tell us that under a homomorphism, the order of the image must divide the order of the preimage.

\begin{theorem}
Let $G_1$ and $G_2$ be groups and suppose $\phi:G_1\to G_2$ is a homomorphism. If $G_1$ is finite, then $|\phi(G_1)|$ divides $|G_1|$.
\end{theorem}

\begin{theorem}
Let $G_1$ and $G_2$ be groups and suppose $\phi:G_1\to G_2$ is a homomorphism. If $g\in G_1$ such that $|g|$ is finite, then $|\phi(g)|$ divides $|g|$.
\end{theorem}

Every homomorphism has an important subset of the domain associated with it.

\begin{definition}
Let $G_1$ and $G_2$ be groups and suppose $\phi:G_1\to G_2$ is a homomorphism.  The \textbf{kernel} of $\phi$ is defined via
\[
\ker(\phi):=\{g\in G_1\mid \phi(g)=e_2\}.
\]
\end{definition}

The kernel of a homomorphism is analogous to the null space of a linear transformation of vector spaces.  

\begin{exercise}
Identify the kernel and image for the homomorphism given in Exercise~\ref{exer:homomorphism}.
\end{exercise}

\begin{exercise}
What is the kernel of a trivial homomorphism (see Theorem~\ref{thm:trivial_homomorphism}).
\end{exercise}

\begin{theorem}\label{thm:kernel_normal}
Let $G_1$ and $G_2$ be groups and suppose $\phi:G_1\to G_2$ is a homomorphism. Then $\ker(\phi)\trianglelefteq G_1$.
\end{theorem}

It turns out that the kernel can tell us something about whether $\phi$ is one-to-one.

\begin{theorem}
Let $G_1$ and $G_2$ be groups and suppose $\phi:G_1\to G_2$ is a homomorphism. Then $\phi$ is one-to-one iff $\ker(\phi)=\{e_1\}$.
\end{theorem}

\begin{remark}
Let $G_1$ and $G_2$ be groups and suppose $\phi:G_1\to G_2$ is a homomorphism. Given a generating set for $G_1$, the homomorphism $\phi$ is uniquely determined by its action on the generating set for $G_1$.  In particular, if you have a word for a group element written in terms of the generators, just apply the homomorphic property to the word to find the image of the corresponding group element.
\end{remark}

\begin{exercise}\label{exer:Q8toV4}
Suppose $\phi: Q_8\to V_{4}$ is a group homomorphism satisfying $\phi(i)=h$ and $\phi(j)=v$.
\begin{enumerate}[label=\rm{(\alph*)}]
\item Find $\phi(1)$, $\phi(-1)$, $\phi(k)$, $\phi(-i)$, $\phi(-j)$, and $\phi(-k)$.
\item Find $\ker(\phi)$.
\item What well-known group is $Q_8/\ker(\phi)$ isomorphic to?
\end{enumerate}
\end{exercise}

\begin{exercise}
Find a non-trivial homomorphism from $\mathbb{Z}_{10}$ to $\mathbb{Z}_6$.
\end{exercise}

\begin{exercise}
Find all non-trivial homomorphisms from $\mathbb{Z}_3$ to $\mathbb{Z}_6$.
\end{exercise}

\begin{problem}
Prove that the only homomorphism from $D_3$ to $\mathbb{Z}_3$ is the trivial homomorphism.
\end{problem}

\begin{exercise}
Let $F$ be the set of all functions from $\mathbb{R}$ to $\mathbb{R}$ and let $D$ be the subset of differentiable functions on $\mathbb{R}$.  It turns out that $F$ is a group under addition of functions and $D$ is a subgroup of $F$ (you do not need to prove this). Define $\phi:D\to F$ via $\phi(f)=f'$ (where $f'$ is the derivative of $f$). Prove that $\phi$ is a homomorphism.  You may recall facts from calculus without proving them. Is $\phi$ one-to-one? Onto? 
\end{exercise}

\end{section}

\begin{section}{The Isomorphism Theorems}

We begin with a theorem.

\begin{theorem}\label{thm:canonical_projection}
Let $G$ be a group and let $H\trianglelefteq G$.  Then the map $\gamma:G\to G/H$ given by $\gamma(g)=gH$ is a homomorphism with $\ker(\gamma)=H$. This map is called the \textbf{canonical projection map}.
\end{theorem}

The upshot of Theorems~\ref{thm:kernel_normal} and \ref{thm:canonical_projection} is that kernels of homomorphisms are always normal and every normal subgroup is the kernel of some homomorphism.

The next theorem is arguably the crowning achievement of the course.

\begin{theorem}[The First Isomorphism Theorem]
Let $G_1$ and $G_2$ be groups and suppose $\phi:G_1\to G_2$ is a homomorphism. Then
\[
G_1/\ker(\phi)\cong \phi(G_1).
\]
If $\phi$ is onto, then
\[
G_1/\ker(\phi)\cong G_2.
\]
\end{theorem}

\begin{exercise}
Let $\phi:Q_8\to V_4$ be the homomorphism described in Exercise~\ref{exer:Q8toV4}. Use the First Isomorphism Theorem to prove that $Q_8/\langle-1\rangle\cong V_4$.
\end{exercise}

\begin{exercise}
Define $\phi:S_n\to \mathbb{Z}_2$ via
\[
\phi(\sigma)=\begin{cases}
0, & \sigma \text{ even}\\
1, & \sigma \text{ odd}.
\end{cases}
\]
Use the First Isomorphism Theorem to prove that $S_n/A_n\cong \mathbb{Z}_2$.
\end{exercise}

\begin{exercise}
Use the First Isomorphism Theorem to prove that $\mathbb{Z}/6\mathbb{Z}\cong \mathbb{Z}_6$.  Attempt to draw a picture of this using Cayley diagrams.
\end{exercise}

\begin{exercise}
Use the First Isomorphism Theorem to prove that $(\mathbb{Z}_4\times \mathbb{Z}_2)/(\{0\}\times \mathbb{Z}_2)\cong \mathbb{Z}_4$.
\end{exercise}

We finish the chapter by listing a few of the remaining isomorphism theorems, but we won't prove these in this course.

\begin{theorem}[The Second Isomorphism Theorem]
Let $G$ be a group with $H\leq G$ and $N\trianglelefteq G$.  Then
\begin{enumerate}
\item $HN:=\{hn\mid h\in H, n\in N\}\leq G$;
\item $H\cap N\trianglelefteq H$;
\item $\displaystyle H/H\cap N\cong HN/N$.
\end{enumerate}
\end{theorem}

\begin{theorem}[The Third Isomorphism Theorem]
Let $G$ be a group with $H,K\trianglelefteq G$ and $K\leq H$.  Then
\[
G/H\cong (G/K)/(H/K).
\]
\end{theorem}

\end{section}